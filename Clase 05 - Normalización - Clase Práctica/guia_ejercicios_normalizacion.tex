\documentclass[12pt]{article}

\usepackage[utf8]{inputenc}
\usepackage[T1]{fontenc}
\usepackage{lmodern}
\usepackage[spanish]{babel}
\usepackage{booktabs}
\usepackage{amsmath}
\usepackage{forest}
\usepackage{float}
\usepackage{listings}
\usepackage{xcolor}
\usepackage{tikz}
\usepackage{array}
\usepackage{multirow}

\definecolor{codegreen}{rgb}{0,0.6,0}
\definecolor{codegray}{rgb}{0.5,0.5,0.5}
\definecolor{codepurple}{rgb}{0.58,0,0.82}
\definecolor{backcolour}{rgb}{0.95,0.95,0.92}

\lstdefinestyle{mystyle}{
    backgroundcolor=\color{backcolour},   
    commentstyle=\color{codegreen},
    keywordstyle=\color{magenta},
    numberstyle=\tiny\color{codegray},
    stringstyle=\color{codepurple},
    basicstyle=\ttfamily\footnotesize,
    breakatwhitespace=false,         
    breaklines=true,                 
    captionpos=b,                    
    keepspaces=true,                 
    numbers=left,                    
    numbersep=5pt,                  
    showspaces=false,                
    showstringspaces=false,
    showtabs=false,                  
    tabsize=2
}

\lstset{style=mystyle}

\sloppy
\setlength{\parindent}{0pt}

\begin{document}

\begin{center}
  {\LARGE \textbf{Guía de Ejercicios \\ Normalización de Bases de Datos}}\\[0.5em]
  {Gestión y Arquitectura de Datos, Universidad de San Andrés}
\end{center}

Si encuentran algún error en el documento o hay alguna duda, mandenmé un mail a rodriguezf@udesa.edu.ar y lo revisamos.

\section{Ejercicios}

A continuación se presentan 10 ejercicios de normalización, ordenados por dificultad creciente. Para cada ejercicio:
\begin{itemize}
    \item Identifique las dependencias funcionales
    \item Determine la forma normal actual
    \item Normalice hasta alcanzar la 3FN
\end{itemize}

Intenten resolver los ejercicios por su cuenta. Las respuestas se encuentran al final del documento.

\subsection{Ejercicio 1: Club de Tenis}
Se tiene la siguiente tabla para un club de tenis:
\vspace{0.5em}
\begin{center}
\begin{tabular}{|l|l|l|l|l|}
\hline
ID\_Socio & Nombre & Teléfono & Cancha & Horario\_Reserva \\
\hline
\end{tabular}
\end{center}

Donde:
\begin{itemize}
    \item Cada socio tiene un ID único, nombre y teléfono
    \item Un socio puede reservar múltiples canchas en diferentes horarios
    \item Una cancha solo puede ser reservada por un socio en un horario específico
\end{itemize}

\subsection{Ejercicio 2: Biblioteca Musical}
Para una biblioteca de música se almacena:
\vspace{0.5em}
\begin{center}
\begin{tabular}{|l|l|l|l|l|l|}
\hline
ID\_Canción & Título & Artista & Álbum & Año\_Álbum & Género \\
\hline
\end{tabular}
\end{center}

Donde:
\begin{itemize}
    \item Cada canción tiene un ID único
    \item Un artista puede tener múltiples álbumes
    \item Un álbum pertenece a un único artista y tiene un año de lanzamiento
    \item Una canción pertenece a un único álbum
\end{itemize}

\subsection{Ejercicio 3: Torneo de Fútbol}
Se tiene los siguientes datos de una base de datos de un torneo de fútbol:

\vspace{0.5em}

\begin{minipage}[t]{0.48\textwidth}
\begin{itemize}
    \item ID\_Partido
    \item Equipo\_Local
    \item Ciudad\_Local
    \item Equipo\_Visitante
\end{itemize}
\end{minipage}
\hfill
\begin{minipage}[t]{0.48\textwidth}
\begin{itemize}
    \item Ciudad\_Visitante
    \item Fecha
    \item Resultado
\end{itemize}
\end{minipage}

\vspace{1em}

Donde:
\begin{itemize}
    \item Cada partido tiene un ID único
    \item Cada equipo pertenece a una única ciudad
    \item Un partido se juega entre dos equipos diferentes en una fecha específica
    \item El resultado registra el marcador final
\end{itemize}

\subsection{Ejercicio 4: Empresa de Ventas}
Para una empresa se registran las ventas en los siguientes datos:

\vspace{0.5em}

\begin{minipage}[t]{0.48\textwidth}
\begin{itemize}
    \item ID\_Venta
    \item Vendedor
    \item Departamento
    \item Supervisor
\end{itemize}
\end{minipage}
\begin{minipage}[t]{0.48\textwidth}
\begin{itemize}
    \item Producto
    \item Cantidad
    \item Precio\_Unit
\end{itemize}
\end{minipage}

\vspace{1em}

Donde:
\begin{itemize}
    \item Cada venta tiene un ID único
    \item Cada vendedor pertenece a un único departamento
    \item Cada departamento tiene un único supervisor
    \item Cada producto tiene un precio unitario fijo
\end{itemize}

\subsection{Ejercicio 5: Local de Computadoras}
Se tiene los siguientes datos de una base de datos de un local de computadoras:

\vspace{0.5em}

\begin{minipage}[t]{0.48\textwidth}
\begin{itemize}
    \item ID\_PC
    \item Marca
    \item Modelo
    \item Procesador
\end{itemize}
\end{minipage}
\begin{minipage}[t]{0.48\textwidth}
\begin{itemize}
    \item Velocidad\_Proc
    \item RAM
    \item Precio
\end{itemize}
\end{minipage}

\vspace{1em}

Donde:
\begin{itemize}
    \item Cada PC tiene un ID único
    \item Cada modelo de PC pertenece a una única marca
    \item Un procesador tiene una velocidad específica
    \item El precio depende del modelo
\end{itemize}

\subsection{Ejercicio 6: Sistema de Cursos}
Para un sistema educativo se tiene los siguientes datos:
\vspace{0.5em}

\begin{minipage}[t]{0.48\textwidth}
\begin{itemize}
    \item ID\_Curso 
    \item Nombre\_Curso
    \item Profesor
    \item Departamento\_Prof
\end{itemize}
\end{minipage}
\hfill
\begin{minipage}[t]{0.48\textwidth}
\begin{itemize}
    \item Aula
    \item Capacidad\_Aula
    \item Horario
\end{itemize}
\end{minipage}

\vspace{1em}

Donde:
\begin{itemize}
    \item Cada curso tiene un ID único y nombre
    \item Cada profesor pertenece a un único departamento
    \item Cada aula tiene una capacidad fija
    \item Un curso se dicta en un aula específica en un horario específico
\end{itemize}

\subsection{Ejercicio 7: Biblioteca de Películas}
Se tiene los siguientes datos de una base de datos de una biblioteca de películas:

\vspace{0.5em}

\begin{minipage}[t]{0.48\textwidth}
\begin{itemize}
    \item ID\_Película
    \item Título
    \item Director
    \item País\_Director
\end{itemize}
\end{minipage}
\hfill
\begin{minipage}[t]{0.48\textwidth}
\begin{itemize}
    \item Año
    \item Género
    \item Duración
    \item Clasificación
\end{itemize}
\end{minipage}

\vspace{1em}

Donde:
\begin{itemize}
    \item Cada película tiene un ID único
    \item Cada director es de un país específico
    \item Una película tiene un único director
    \item La clasificación depende del contenido de la película
\end{itemize}

\subsection{Ejercicio 8: Gestión de Proyectos}
Para gestionar proyectos se usa los siguientes datos:

\vspace{0.5em}

\begin{minipage}[t]{0.48\textwidth}
\begin{itemize}
    \item ID\_Proyecto
    \item Nombre\_Proyecto
    \item Líder
    \item Departamento\_Líder
\end{itemize}
\end{minipage}
\hfill
\begin{minipage}[t]{0.48\textwidth}
\begin{itemize}
    \item Presupuesto
    \item Fecha\_Inicio
    \item Cliente
    \item Ciudad\_Cliente
\end{itemize}
\end{minipage}

\vspace{1em}

Donde:
\begin{itemize}
    \item Cada proyecto tiene un ID único
    \item Cada líder pertenece a un departamento
    \item Cada cliente está en una ciudad específica
    \item Un proyecto tiene un único líder y un único cliente
\end{itemize}

\subsection{Ejercicio 9: Local de Instrumentos}
Para un local de música se registra:

\vspace{0.5em}

\begin{minipage}[t]{0.48\textwidth}
\begin{itemize}
    \item ID\_Instrumento
    \item Tipo
    \item Marca
    \item País\_Fabricación
\end{itemize}
\end{minipage}
\hfill
\begin{minipage}[t]{0.48\textwidth}
\begin{itemize}
    \item Modelo
    \item Material
    \item Precio
    \item Stock
\end{itemize}
\end{minipage}

\vspace{1em}

Donde:
\begin{itemize}
    \item Cada instrumento tiene un ID único
    \item Cada modelo pertenece a una marca específica
    \item Cada marca tiene un país de fabricación principal
    \item El precio y stock son específicos para cada modelo
\end{itemize}

\subsection{Ejercicio 10: Sistema de Hospital}
Se tiene los siguientes datos de una base de datos de un sistema de hospital:

\vspace{0.5em}

\begin{minipage}[t]{0.48\textwidth}
\begin{itemize}
    \item ID\_Consulta
    \item Paciente
    \item Obra\_Social
    \item Doctor
    \item Especialidad
\end{itemize}
\end{minipage}
\hfill
\begin{minipage}[t]{0.48\textwidth}
\begin{itemize}
    \item Consultorio
    \item Fecha
    \item Diagnóstico
    \item Tratamiento
\end{itemize}
\end{minipage}

\vspace{1em}
Donde:
\begin{itemize}
    \item Cada consulta tiene un ID único
    \item Cada paciente tiene una obra social
    \item Cada doctor tiene una especialidad
    \item Cada consultorio está asignado a una especialidad
    \item Una consulta genera un diagnóstico y un tratamiento
\end{itemize}

\newpage

\section{Anexo: Respuestas}

\subsection{Respuesta Ejercicio 1: Club de Tenis}

Dependencias Funcionales:
\begin{itemize}
    \item ID\_Socio → Nombre, Teléfono
    \item Cancha, Horario\_Reserva → ID\_Socio
\end{itemize}

La tabla no está en 2FN porque hay dependencias parciales. Normalización:

\textbf{Socios (1FN → 2FN, 3FN)}
\begin{center}
\begin{tabular}{|l|l|l|}
\hline
ID\_Socio & Nombre & Teléfono \\
\hline
\end{tabular}
\end{center}

\textbf{Reservas (2FN → 3FN)}
\begin{center}
\begin{tabular}{|l|l|l|}
\hline
ID\_Socio & Cancha & Horario\_Reserva \\
\hline
\end{tabular}
\end{center}

\subsection{Respuesta Ejercicio 2: Biblioteca Musical}

Dependencias Funcionales:
\begin{itemize}
    \item ID\_Canción → Título, Álbum, Artista, Género
    \item Álbum → Artista, Año\_Álbum
\end{itemize}

\textbf{Artistas (3FN)}
\begin{center}
\begin{tabular}{|l|l|}
\hline
ID\_Artista & Nombre\_Artista \\
\hline
\end{tabular}
\end{center}

\textbf{Álbumes (3FN)}
\begin{center}
\begin{tabular}{|l|l|l|}
\hline
ID\_Álbum & ID\_Artista & Año\_Álbum \\
\hline
\end{tabular}
\end{center}

\textbf{Canciones (3FN)}
\begin{center}
\begin{tabular}{|l|l|l|l|}
\hline
ID\_Canción & Título & ID\_Álbum & Género \\
\hline
\end{tabular}
\end{center}

\subsection{Respuesta Ejercicio 3: Torneo de Fútbol}

Dependencias Funcionales:
\begin{itemize}
    \item ID\_Partido → Equipo\_Local, Equipo\_Visitante, Fecha, Resultado
    \item Equipo\_Local → Ciudad\_Local
    \item Equipo\_Visitante → Ciudad\_Visitante
\end{itemize}

\textbf{Equipos (3FN)}
\begin{center}
\begin{tabular}{|l|l|}
\hline
ID\_Equipo & Ciudad \\
\hline
\end{tabular}
\end{center}

\textbf{Partidos (3FN)}
\begin{center}
\begin{tabular}{|l|l|l|l|l|}
\hline
ID\_Partido & ID\_Local & ID\_Visitante & Fecha & Resultado \\
\hline
\end{tabular}
\end{center}

\subsection{Respuesta Ejercicio 4: Empresa de Ventas}

Dependencias Funcionales:
\begin{itemize}
    \item ID\_Venta → Vendedor, Producto, Cantidad
    \item Vendedor → Departamento
    \item Departamento → Supervisor
    \item Producto → Precio\_Unit
\end{itemize}

\textbf{Departamentos (3FN)}
\begin{center}
\begin{tabular}{|l|l|}
\hline
ID\_Departamento & Supervisor \\
\hline
\end{tabular}
\end{center}

\textbf{Vendedores (3FN)}
\begin{center}
\begin{tabular}{|l|l|}
\hline
ID\_Vendedor & ID\_Departamento \\
\hline
\end{tabular}
\end{center}

\textbf{Productos (3FN)}
\begin{center}
\begin{tabular}{|l|l|}
\hline
ID\_Producto & Precio\_Unit \\
\hline
\end{tabular}
\end{center}

\textbf{Ventas (3FN)}
\begin{center}
\begin{tabular}{|l|l|l|l|}
\hline
ID\_Venta & ID\_Vendedor & ID\_Producto & Cantidad \\
\hline
\end{tabular}
\end{center}

\subsection{Respuesta Ejercicio 5: Local de Computadoras}

Dependencias Funcionales:
\begin{itemize}
    \item ID\_PC → Marca, Modelo, Procesador, RAM
    \item Modelo → Marca, Precio
    \item Procesador → Velocidad\_Proc
\end{itemize}

\textbf{Marcas (3FN)}
\begin{center}
\begin{tabular}{|l|l|}
\hline
ID\_Marca & Nombre\_Marca \\
\hline
\end{tabular}
\end{center}

\textbf{Modelos (3FN)}
\begin{center}
\begin{tabular}{|l|l|l|}
\hline
ID\_Modelo & ID\_Marca & Precio \\
\hline
\end{tabular}
\end{center}

\textbf{Procesadores (3FN)}
\begin{center}
\begin{tabular}{|l|l|}
\hline
ID\_Procesador & Velocidad\_Proc \\
\hline
\end{tabular}
\end{center}

\textbf{Computadoras (3FN)}
\begin{center}
\begin{tabular}{|l|l|l|l|}
\hline
ID\_PC & ID\_Modelo & ID\_Procesador & RAM \\
\hline
\end{tabular}
\end{center}

\subsection{Respuesta Ejercicio 6: Sistema de Cursos}

Dependencias Funcionales:
\begin{itemize}
    \item ID\_Curso → Nombre\_Curso, Profesor, Aula, Horario
    \item Profesor → Departamento\_Prof
    \item Aula → Capacidad\_Aula
\end{itemize}

\textbf{Profesores (3FN)}
\begin{center}
\begin{tabular}{|l|l|}
\hline
ID\_Profesor & Departamento\_Prof \\
\hline
\end{tabular}
\end{center}

\textbf{Aulas (3FN)}
\begin{center}
\begin{tabular}{|l|l|}
\hline
ID\_Aula & Capacidad\_Aula \\
\hline
\end{tabular}
\end{center}

\textbf{Cursos (3FN)}
\begin{center}
\begin{tabular}{|l|l|l|l|l|}
\hline
ID\_Curso & Nombre\_Curso & ID\_Profesor & ID\_Aula & Horario \\
\hline
\end{tabular}
\end{center}

\subsection{Respuesta Ejercicio 7: Biblioteca de Películas}

Dependencias Funcionales:
\begin{itemize}
    \item ID\_Película → Título, Director, Año, Género, Duración, Clasificación
    \item Director → País\_Director
\end{itemize}

\textbf{Directores (3FN)}
\begin{center}
\begin{tabular}{|l|l|}
\hline
ID\_Director & País\_Director \\
\hline
\end{tabular}
\end{center}

\textbf{Películas (3FN)}
\begin{center}
\begin{tabular}{|l|l|l|l|l|l|l|}
\hline
ID\_Película & Título & ID\_Director & Año & Género & Duración & Clasificación \\
\hline
\end{tabular}
\end{center}

\subsection{Respuesta Ejercicio 8: Gestión de Proyectos}

Dependencias Funcionales:
\begin{itemize}
    \item ID\_Proyecto → Nombre\_Proyecto, Líder, Presupuesto, Fecha\_Inicio, Cliente
    \item Líder → Departamento\_Líder
    \item Cliente → Ciudad\_Cliente
\end{itemize}

\textbf{Líderes (3FN)}
\begin{center}
\begin{tabular}{|l|l|}
\hline
ID\_Líder & Departamento\_Líder \\
\hline
\end{tabular}
\end{center}

\textbf{Clientes (3FN)}
\begin{center}
\begin{tabular}{|l|l|}
\hline
ID\_Cliente & Ciudad\_Cliente \\
\hline
\end{tabular}
\end{center}

\textbf{Proyectos (3FN)}
\begin{center}
\begin{tabular}{|l|l|l|l|l|l|}
\hline
ID\_Proyecto & Nombre\_Proyecto & ID\_Líder & Presupuesto & Fecha\_Inicio & ID\_Cliente \\
\hline
\end{tabular}
\end{center}

\subsection{Respuesta Ejercicio 9: Local de Instrumentos}

Dependencias Funcionales:
\begin{itemize}
    \item ID\_Instrumento → Tipo, Marca, Modelo, Material
    \item Marca → País\_Fabricación
    \item Modelo → Precio, Stock
\end{itemize}

\textbf{Marcas (3FN)}
\begin{center}
\begin{tabular}{|l|l|}
\hline
ID\_Marca & País\_Fabricación \\
\hline
\end{tabular}
\end{center}

\textbf{Modelos (3FN)}
\begin{center}
\begin{tabular}{|l|l|l|}
\hline
ID\_Modelo & Precio & Stock \\
\hline
\end{tabular}
\end{center}

\textbf{Instrumentos (3FN)}
\begin{center}
\begin{tabular}{|l|l|l|l|l|}
\hline
ID\_Instrumento & Tipo & ID\_Marca & ID\_Modelo & Material \\
\hline
\end{tabular}
\end{center}

\subsection{Respuesta Ejercicio 10: Sistema de Hospital}

Dependencias Funcionales:
\begin{itemize}
    \item ID\_Consulta → Paciente, Doctor, Consultorio, Fecha, Diagnóstico, Tratamiento
    \item Paciente → Obra\_Social
    \item Doctor → Especialidad
    \item Consultorio → Especialidad
\end{itemize}

\textbf{Pacientes (3FN)}
\begin{center}
\begin{tabular}{|l|l|}
\hline
ID\_Paciente & Obra\_Social \\
\hline
\end{tabular}
\end{center}

\textbf{Doctores (3FN)}
\begin{center}
\begin{tabular}{|l|l|}
\hline
ID\_Doctor & Especialidad \\
\hline
\end{tabular}
\end{center}

\textbf{Consultorios (3FN)}
\begin{center}
\begin{tabular}{|l|l|}
\hline
ID\_Consultorio & Especialidad \\
\hline
\end{tabular}
\end{center}

\textbf{Consultas (3FN)}
{\small
\begin{center}
\begin{tabular}{|l|l|l|l|l|l|l|}
\hline
ID\_Consulta & ID\_Paciente & ID\_Doctor & ID\_Consultorio & Fecha & Diagnóstico & Tratamiento \\
\hline
\end{tabular}
\end{center}
}
\end{document}
