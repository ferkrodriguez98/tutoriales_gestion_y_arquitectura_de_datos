% Diapositivas DER - Basado en apunte y template UdeSA

\documentclass{beamer}
\usetheme{metropolis}

\usepackage[spanish]{babel}
\usepackage[utf8]{inputenc}
\usepackage{graphicx}
\usepackage{tikz}
\usepackage{xcolor}
\usepackage{amsmath}
\usepackage{fontawesome5}

% Definición de estilos para entidades (copiado del apunte)
\usetikzlibrary{positioning,shapes.multipart,calc,arrows,shapes.geometric}
\tikzset{
    entity/.style={
        rectangle split,
        rectangle split parts=2,
        draw,
        align=center,
        minimum height=1cm
    },
    arrow/.style={
        ->,
        >=latex
    }
}

% Colores y configuración del template
\definecolor{primary}{RGB}{46, 204, 113}
\definecolor{secondary}{RGB}{52, 152, 219}
\definecolor{accent}{RGB}{231, 76, 60}
\definecolor{background}{RGB}{236, 240, 241}
\setbeamercolor{normal text}{fg=black,bg=background}
\setbeamercolor{structure}{fg=primary}
\setbeamercolor{alerted text}{fg=accent}

% Título
\title{Diagramas Entidad-Relación (DER)}
\author{Gestión y Arquitectura de Datos}
\date{}

\begin{document}

% Portada
\begin{frame}
    \titlepage
    \begin{tikzpicture}[remember picture,overlay]
        \node[anchor=south west,inner sep=30pt] at (current page.south west) {
            \includegraphics[height=1cm]{../misc/UdeSA.png}
        };
    \end{tikzpicture}
\end{frame}

% Conceptos basicos
\begin{frame}{Conceptos basicos}
    \begin{columns}[c]
        \column{0.7\textwidth}
            \begin{center}
                Un DER es una herramienta para el modelado de datos que permite representar entidades relevantes de un sistema y sus interrelaciones
            \end{center}
        \column{0.3\textwidth}
            \begin{center}
                {\fontsize{40pt}{48pt}\selectfont\faDatabase}
            \end{center}
        
    \end{columns}
\end{frame}

\begin{frame}{Elementos básicos - Entidad}
    \begin{center}
        \begin{tikzpicture}[node distance=4cm]
            % Entidades
            \node[entity] (cliente) {Cliente\nodepart{second}\phantom{\textbf{id\_cliente (PK)}}\\ \phantom{nombre}\\ \phantom{email}\\ \phantom{telefono}};
        \end{tikzpicture}
    \end{center}
\end{frame}

\begin{frame}{Elementos básicos - Atributos}
    \begin{center}
        \begin{tikzpicture}[node distance=4cm]
            % Entidades
            \node[entity] (cliente) {Cliente\nodepart{second}\textbf{id\_cliente (PK)}\\ nombre\\ email\\ telefono};
        \end{tikzpicture}
    \end{center}
\end{frame}

\begin{frame}{Elementos básicos - Relación}
    \begin{center}
        \begin{tikzpicture}[node distance=4cm]
            % Entidades
            \node[entity] (cliente) {Cliente\nodepart{second}\textbf{id\_cliente (PK)}\\ nombre\\ email\\ telefono};
            \node[entity] (pedido) [right=of cliente] {Pedido\nodepart{second}\textbf{id\_pedido (PK)}\\ \underline{id\_cliente (FK)}\\ fecha\\ total\\ estado};
            
            % Conexión con cardinalidades y flecha FK->PK
            \coordinate (fk_cliente) at ($(pedido.west)+(0,0.1)$);
            \coordinate (pk_cliente) at ($(cliente.east)+(0,0.4)$);
            \draw[-latex] (fk_cliente) to[out=180,in=0] 
                node[pos=0.1,above] {\phantom{1..N}} 
                node[pos=0.9,above] {\phantom{1..1}} 
                (pk_cliente);
        \end{tikzpicture}
    \end{center}
\end{frame}

\begin{frame}{Elementos básicos - Cardinalidad}
    \begin{center}
        \begin{tikzpicture}[node distance=4cm]
            % Entidades
            \node[entity] (cliente) {Cliente\nodepart{second}\textbf{id\_cliente (PK)}\\ nombre\\ email\\ telefono};
            \node[entity] (pedido) [right=of cliente] {Pedido\nodepart{second}\textbf{id\_pedido (PK)}\\ \underline{id\_cliente (FK)}\\ fecha\\ total\\ estado};
            
            % Conexión con cardinalidades y flecha FK->PK
            \coordinate (fk_cliente) at ($(pedido.west)+(0,0.1)$);
            \coordinate (pk_cliente) at ($(cliente.east)+(0,0.4)$);
            \draw[-latex] (fk_cliente) to[out=180,in=0] 
                node[pos=0.1,above] {1..N} 
                node[pos=0.9,above] {1..1} 
                (pk_cliente);
        \end{tikzpicture}
    \end{center}
\end{frame}

\begin{frame}{Tipos de relaciones}
    \begin{itemize}
        \item Relacion Uno a Uno (1..1)
        \item Relacion Uno a Muchos (1..N)
        \item Relacion Muchos a Muchos (N..M)
        \item Herencia (Tipo-Subtipo)
        \item Autoreferencial
    \end{itemize}
\end{frame}

\begin{frame}{Relación Uno a Uno}
    \begin{center}
        \begin{tikzpicture}[node distance=4cm]
            \node[entity] (persona) {Persona\nodepart{second}\textbf{id\_persona (PK)}\\ nombre\\ apellido\\ fecha\_nacimiento};
            \node[entity] (pasaporte) [right=of persona] {Pasaporte\nodepart{second}\textbf{id\_pasaporte (PK)}\\ \underline{id\_persona (FK)}\\ numero\\ fecha\_vencimiento};
        
            \coordinate (fk_persona) at ($(pasaporte.west)+(0,-0.1)$);
            \coordinate (pk_persona) at ($(persona.east)+(0,0.4)$);
            \draw[-latex] (fk_persona) to[out=180,in=0] 
                node[pos=0.1,above] {1..1} 
                node[pos=0.9,above] {1..1} 
                (pk_persona);
        \end{tikzpicture}
    \end{center}
\end{frame}

% Relación Uno a Muchos
\begin{frame}{Relación Uno a Muchos}
    \begin{center}
    \begin{tikzpicture}[node distance=4cm]
        \node[entity] (cliente) {Cliente\nodepart{second}\textbf{id\_cliente (PK)}\\ nombre\\ email\\ telefono};
        \node[entity] (pedido) [right=of cliente] {Pedido\nodepart{second}\textbf{id\_pedido (PK)}\\ \underline{id\_cliente (FK)}\\ fecha\\ total\\ estado};
        \coordinate (fk_cliente) at ($(pedido.west)+(0,0.1)$);
        \coordinate (pk_cliente) at ($(cliente.east)+(0,0.4)$);
        \draw[-latex] (fk_cliente) to[out=180,in=0] node[pos=0.1,above] {1..N} node[pos=0.9,above] {1..1} (pk_cliente);
    \end{tikzpicture}
    \end{center}
\end{frame}

% Relación Muchos a Muchos
\begin{frame}{Relación Muchos a Muchos}
    \begin{center}
    \resizebox{0.95\textwidth}{!}{
    \begin{tikzpicture}[node distance=2cm]
        \node[entity] (pedido) {Pedido\nodepart{second}\textbf{id\_pedido (PK)}\\ fecha\\ total};
        \node[entity] (detalle) [right=of pedido] {Detalle\_Pedido\nodepart{second}\textbf{id\_detalle (PK)}\\ \underline{id\_pedido (FK)}\\ \underline{id\_producto (FK)}\\ cantidad\\ precio\_unitario};
        \node[entity] (producto) [right=of detalle] {Producto\nodepart{second}\textbf{id\_producto (PK)}\\ nombre\\ precio\\ stock};
        \coordinate (fk_pedido) at ($(detalle.west)+(0,0.1)$);
        \coordinate (pk_pedido) at ($(pedido.east)+(0,0.1)$);
        \draw[-latex] (fk_pedido) to[out=180,in=0] node[pos=0.2,above] {1..N} node[pos=0.8,above] {1..1} (pk_pedido);
        \coordinate (fk_producto) at ($(detalle.east)+(0,-0.5)$);
        \coordinate (pk_producto) at ($(producto.west)+(0,0.5)$);
        \draw[-latex] (fk_producto) to[out=0,in=180] node[pos=0.2,above] {N..1} node[pos=0.8,above] {1..1} (pk_producto);
    \end{tikzpicture}
    }
    \end{center}
\end{frame}

% Herencia (Subtipo-Supertipo)
\begin{frame}{Herencia (Subtipo-Supertipo)}
    \begin{center}
    \resizebox{0.95\textwidth}{!}{
    \begin{tikzpicture}[node distance=4cm]
        \node[entity] (usuario) {Usuario\nodepart{second}\textbf{id\_usuario (PK)}\\ nombre\\ email\\ password};
        \node[entity] (cliente) [below left=2cm of usuario] {Cliente\nodepart{second}\underline{\textbf{id\_usuario (PK,FK)}}\\ direccion\\ telefono\\ nivel\_membresia};
        \node[entity] (empleado) [below right=2cm of usuario] {Empleado\nodepart{second}\underline{\textbf{id\_usuario (PK,FK)}}\\ salario\\ departamento\\ fecha\_ingreso};
        \coordinate (pk_usuario) at ($(usuario.south)+(0,-0.1)$);
        \coordinate (fk_cliente) at ($(cliente.north)+(0,0.1)$);
        \coordinate (fk_empleado) at ($(empleado.north)+(0,0.1)$);
        \coordinate (union) at ($(fk_cliente)!0.5!(fk_empleado)$);
        \draw[arrow] (union) -- (pk_usuario);
        \draw (fk_cliente) -- (union);
        \draw (fk_empleado) -- (union);
    \end{tikzpicture}
    }
    \end{center}
\end{frame}

% Relación Autoreferencial
\begin{frame}{Relación Autoreferencial}
    \begin{center}
    \begin{tikzpicture}[node distance=4cm]
        \node[entity] (empleado) {Empleado\nodepart{second}\textbf{id\_empleado (PK)}\\ \underline{id\_supervisor (FK)}\\ nombre\\ cargo\\ departamento};
        \coordinate (salida) at ($(empleado.north)+(0,0)$);         
        \coordinate (p1) at ($(salida)+(0,1)$);                      
        \coordinate (p2) at ($(p1)+(3.5,0)$);                          
        \coordinate (p3) at ($(p2)+(0,-2.7)$);                         
        \coordinate (entrada) at ($(empleado.east)+(0,0)$);           
        \coordinate (p4) at ($(entrada)+(0.8,0)$);                    
        \draw[-latex, thick] (salida) -- (p1) -- (p2) -- (p3) -- (p4) -- (entrada);
        \node at ($(p2)+(-3,0.4)$) {0..N};
        \node at ($(p4)+(-0.4,0.3)$) {0..1};
    \end{tikzpicture}
    \end{center}
\end{frame}

% Consideraciones importantes (una sola slide, ítems secuenciales)
\begin{frame}{Consideraciones importantes}
    \begin{itemize}
        \item<1-> \textbf{Claves Primarias (PK)}: Identificador único, mostrado en negrita.
        \item<2-> \textbf{Claves Foráneas (FK)}: Referencias a PKs, mostradas subrayadas.
        \item<3-> \textbf{Cardinalidades}: Siempre indicar en ambos extremos (min..max).
        \item<4-> \textbf{Herencia}: Las PKs se heredan como PK,FK en las tablas hijas.
        \item<5-> \textbf{Relaciones N:M}: Requieren tabla pivot con sus propias FKs.
    \end{itemize}
\end{frame}

\begin{frame}{Terminamos}
    \begin{center}
        \Large{\textbf{¿Dudas?\\¿Consultas?}}
    \end{center}
    \begin{tikzpicture}[remember picture,overlay]
        \node[anchor=south,inner sep=30pt] at (current page.south) {
            \includegraphics[height=1cm]{../misc/UdeSA.png}
        };
    \end{tikzpicture}
\end{frame}

\end{document}
