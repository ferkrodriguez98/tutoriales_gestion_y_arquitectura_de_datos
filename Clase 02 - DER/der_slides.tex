% Diapositivas DER - Basado en apunte y template UdeSA

\documentclass{beamer}
\usetheme{metropolis}

\usepackage[spanish]{babel}
\usepackage[utf8]{inputenc}
\usepackage{graphicx}
\usepackage{tikz}
\usepackage{xcolor}
\usepackage{amsmath}
\usepackage{fontawesome5}
\usepackage{soul}

\setul{0.5ex}{0.3ex}

% Definición de estilos para entidades (copiado del apunte)
\usetikzlibrary{positioning,shapes.multipart,calc,arrows,shapes.geometric}
\tikzset{
    entity/.style={
        rectangle split,
        rectangle split parts=2,
        draw,
        align=center,
        minimum height=1cm
    },
    arrow/.style={
        ->,
        >=latex
    }
}

% Colores y configuración del template
\definecolor{primary}{RGB}{46, 204, 113}
\definecolor{secondary}{RGB}{52, 152, 219}
\definecolor{accent}{RGB}{231, 76, 60}
\definecolor{background}{RGB}{236, 240, 241}
\setbeamercolor{normal text}{fg=black,bg=background}
\setbeamercolor{structure}{fg=primary}
\setbeamercolor{alerted text}{fg=accent}

% Título
\title{Diagramas Entidad-Relación (DER)}
\author{Gestión y Arquitectura de Datos}
\date{}

\begin{document}

% Portada
\begin{frame}
    \titlepage
    \begin{tikzpicture}[remember picture,overlay]
        \node[anchor=south west,inner sep=30pt] at (current page.south west) {
            \includegraphics[height=1cm]{../misc/UdeSA.png}
        };
    \end{tikzpicture}
\end{frame}

% Conceptos basicos
\begin{frame}{Conceptos basicos}
    \begin{columns}[c]
        \column{0.7\textwidth}
            \begin{center}
                Un DER es una herramienta para el modelado de datos que permite representar entidades relevantes de un sistema y sus interrelaciones
            \end{center}
        \column{0.3\textwidth}
            \begin{center}
                {\fontsize{40pt}{48pt}\selectfont\faDatabase}
            \end{center}
        
    \end{columns}
\end{frame}

\begin{frame}{Elementos básicos - Entidad}
    \begin{center}
        \begin{tikzpicture}[node distance=4cm]
            % Entidades
            \node[entity] (cliente) {Cliente\nodepart{second}\phantom{\textbf{id\_cliente (PK)}}\\ \phantom{nombre}\\ \phantom{email}\\ \phantom{telefono}};
        \end{tikzpicture}
    \end{center}
\end{frame}

\begin{frame}{Elementos básicos - Atributos}
    \begin{center}
        \begin{tikzpicture}[node distance=4cm]
            % Entidades
            \node[entity] (cliente) {Cliente\nodepart{second}\textbf{id\_cliente (PK)}\\ nombre\\ email\\ telefono};
        \end{tikzpicture}
    \end{center}
\end{frame}

\begin{frame}{Elementos básicos - Relación}
    \begin{center}
        \begin{tikzpicture}[node distance=4cm]
            % Entidades
            \node[entity] (cliente) {Cliente\nodepart{second}\textbf{id\_cliente (PK)}\\ nombre\\ email\\ telefono};
            \node[entity] (pedido) [right=of cliente] {Pedido\nodepart{second}\textbf{id\_pedido (PK)}\\ \underline{id\_cliente (FK)}\\ fecha\\ total\\ estado};
            
            % Conexión con cardinalidades y flecha FK->PK
            \coordinate (fk_cliente) at ($(pedido.west)+(0,0.1)$);
            \coordinate (pk_cliente) at ($(cliente.east)+(0,0.4)$);
            \draw[-latex] (fk_cliente) to[out=180,in=0] 
                node[pos=0.1,above] {\phantom{1..N}} 
                node[pos=0.9,above] {\phantom{1..1}} 
                (pk_cliente);
        \end{tikzpicture}
    \end{center}
\end{frame}

\begin{frame}{Elementos básicos - Cardinalidad}
    \begin{center}
        \begin{tikzpicture}[node distance=4cm]
            % Entidades
            \node[entity] (cliente) {Cliente\nodepart{second}\textbf{id\_cliente (PK)}\\ nombre\\ email\\ telefono};
            \node[entity] (pedido) [right=of cliente] {Pedido\nodepart{second}\textbf{id\_pedido (PK)}\\ \underline{id\_cliente (FK)}\\ fecha\\ total\\ estado};
            
            % Conexión con cardinalidades y flecha FK->PK
            \coordinate (fk_cliente) at ($(pedido.west)+(0,0.1)$);
            \coordinate (pk_cliente) at ($(cliente.east)+(0,0.4)$);
            \draw[-latex] (fk_cliente) to[out=180,in=0] 
                node[pos=0.1,above] {1..N} 
                node[pos=0.9,above] {1..1} 
                (pk_cliente);
        \end{tikzpicture}
    \end{center}
\end{frame}

\begin{frame}{Tipos de relaciones}
    \begin{itemize}
        \item Relacion Uno a Uno (1..1)
        \item Relacion Uno a Muchos (1..N)
        \item Relacion Muchos a Muchos (N..M)
        \item Herencia (Tipo-Subtipo)
        \item Autoreferencial
    \end{itemize}
\end{frame}

\begin{frame}{Relación Uno a Uno}
    \begin{center}
        \begin{tikzpicture}[node distance=4cm]
            \node[entity] (persona) {Persona\nodepart{second}\textbf{id\_persona (PK)}\\ nombre\\ apellido\\ fecha\_nacimiento};
            \node[entity] (pasaporte) [right=of persona] {Pasaporte\nodepart{second}\textbf{id\_pasaporte (PK)}\\ \underline{id\_persona (FK)}\\ numero\\ fecha\_vencimiento};
        
            \coordinate (fk_persona) at ($(pasaporte.west)+(0,-0.1)$);
            \coordinate (pk_persona) at ($(persona.east)+(0,0.4)$);
            \draw[-latex] (fk_persona) to[out=180,in=0] 
                node[pos=0.1,above] {1..1} 
                node[pos=0.9,above] {1..1} 
                (pk_persona);
        \end{tikzpicture}
    \end{center}
\end{frame}

% Relación Uno a Muchos
\begin{frame}{Relación Uno a Muchos}
    \begin{center}
    \begin{tikzpicture}[node distance=4cm]
        \node[entity] (cliente) {Cliente\nodepart{second}\textbf{id\_cliente (PK)}\\ nombre\\ email\\ telefono};
        \node[entity] (pedido) [right=of cliente] {Pedido\nodepart{second}\textbf{id\_pedido (PK)}\\ \underline{id\_cliente (FK)}\\ fecha\\ total\\ estado};
        \coordinate (fk_cliente) at ($(pedido.west)+(0,0.1)$);
        \coordinate (pk_cliente) at ($(cliente.east)+(0,0.4)$);
        \draw[-latex] (fk_cliente) to[out=180,in=0] node[pos=0.1,above] {1..N} node[pos=0.9,above] {1..1} (pk_cliente);
    \end{tikzpicture}
    \end{center}
\end{frame}

% Relación Muchos a Muchos
\begin{frame}{Relación Muchos a Muchos}
    \begin{center}
    \resizebox{0.95\textwidth}{!}{
    \begin{tikzpicture}[node distance=2cm]
        \node[entity] (pedido) {Pedido\nodepart{second}\textbf{id\_pedido (PK)}\\ fecha\\ total};
        \node[entity] (detalle) [right=of pedido] {Detalle\_Pedido\nodepart{second}\textbf{id\_detalle (PK)}\\ \underline{id\_pedido (FK)}\\ \underline{id\_producto (FK)}\\ cantidad\\ precio\_unitario};
        \node[entity] (producto) [right=of detalle] {Producto\nodepart{second}\textbf{id\_producto (PK)}\\ nombre\\ precio\\ stock};
        \coordinate (fk_pedido) at ($(detalle.west)+(0,0.1)$);
        \coordinate (pk_pedido) at ($(pedido.east)+(0,0.1)$);
        \draw[-latex] (fk_pedido) to[out=180,in=0] node[pos=0.2,above] {1..N} node[pos=0.8,above] {1..1} (pk_pedido);
        \coordinate (fk_producto) at ($(detalle.east)+(0,-0.5)$);
        \coordinate (pk_producto) at ($(producto.west)+(0,0.5)$);
        \draw[-latex] (fk_producto) to[out=0,in=180] node[pos=0.2,above] {N..1} node[pos=0.8,above] {1..1} (pk_producto);
    \end{tikzpicture}
    }
    \end{center}
\end{frame}

% Herencia (Subtipo-Supertipo)
\begin{frame}{Herencia (Subtipo-Supertipo)}
    \begin{center}
    \resizebox{0.95\textwidth}{!}{
    \begin{tikzpicture}[node distance=4cm]
        \node[entity] (usuario) {Usuario\nodepart{second}\textbf{id\_usuario (PK)}\\ nombre\\ email\\ password};
        \node[entity] (cliente) [below left=2cm of usuario] {Cliente\nodepart{second}\underline{\textbf{id\_usuario (PK,FK)}}\\ direccion\\ telefono\\ nivel\_membresia};
        \node[entity] (empleado) [below right=2cm of usuario] {Empleado\nodepart{second}\underline{\textbf{id\_usuario (PK,FK)}}\\ salario\\ departamento\\ fecha\_ingreso};
        \coordinate (pk_usuario) at ($(usuario.south)+(0,-0.1)$);
        \coordinate (fk_cliente) at ($(cliente.north)+(0,0.1)$);
        \coordinate (fk_empleado) at ($(empleado.north)+(0,0.1)$);
        \coordinate (union) at ($(fk_cliente)!0.5!(fk_empleado)$);
        \draw[arrow] (union) -- (pk_usuario);
        \draw (fk_cliente) -- (union);
        \draw (fk_empleado) -- (union);
    \end{tikzpicture}
    }
    \end{center}
\end{frame}

% Relación Autoreferencial
\begin{frame}{Relación Autoreferencial}
    \begin{center}
    \begin{tikzpicture}[node distance=4cm]
        \node[entity] (empleado) {Empleado\nodepart{second}\textbf{id\_empleado (PK)}\\ \underline{id\_supervisor (FK)}\\ nombre\\ cargo\\ departamento};
        \coordinate (salida) at ($(empleado.north)+(0,0)$);         
        \coordinate (p1) at ($(salida)+(0,1)$);                      
        \coordinate (p2) at ($(p1)+(3.5,0)$);                          
        \coordinate (p3) at ($(p2)+(0,-2.7)$);                         
        \coordinate (entrada) at ($(empleado.east)+(0,0)$);           
        \coordinate (p4) at ($(entrada)+(0.8,0)$);                    
        \draw[-latex, thick] (salida) -- (p1) -- (p2) -- (p3) -- (p4) -- (entrada);
        \node at ($(p2)+(-3,0.4)$) {0..N};
        \node at ($(p4)+(-0.4,0.3)$) {0..1};
    \end{tikzpicture}
    \end{center}
\end{frame}

% Consideraciones importantes (una sola slide, ítems secuenciales)
\begin{frame}{Consideraciones importantes}
    \begin{itemize}
        \item<1-> \textbf{Claves Primarias (PK)}: Identificador único, mostrado en negrita.
        \item<2-> \textbf{Claves Foráneas (FK)}: Referencias a PKs, mostradas subrayadas.
        \item<3-> \textbf{Cardinalidades}: Siempre indicar en ambos extremos (min..max).
        \item<4-> \textbf{Herencia}: Las PKs se heredan como PK,FK en las tablas hijas.
        \item<5-> \textbf{Relaciones N:M}: Requieren tabla pivot con sus propias FKs.
    \end{itemize}
\end{frame}

% Ejemplo práctico 1 - Enunciado
\begin{frame}{Ejemplo 1: Sistema de Mascotas y Veterinarios}
    \begin{center}
        \small
        Imaginate que tenés que diseñar un sistema para una veterinaria. Los dueños llevan sus mascotas a consultas con veterinarios. Cada mascota tiene un dueño, y un dueño puede tener varias mascotas. Los veterinarios atienden a las mascotas en consultas. Del dueño tenés que guardar su nombre, teléfono y dirección. De la mascota tenés que guardar su nombre, especie y edad. De la consulta tenés que guardar la fecha, el diagnóstico y el tratamiento. De los veterinarios tenés que guardar su nombre, especialidad y matrícula.
    \end{center}
\end{frame}

\begin{frame}{Ejemplo 1: Sistema de Mascotas y Veterinarios}
    \begin{center}
        \small
        Imaginate que tenés que diseñar un sistema para una \setulcolor{red}\ul{veterinaria}. Los \setulcolor{red}\ul{dueños} llevan sus mascotas a consultas con veterinarios. Cada \setulcolor{red}\ul{mascota} tiene un \setulcolor{red}\ul{dueño}, y un dueño puede tener varias mascotas. Los \setulcolor{red}\ul{veterinarios} atienden a las mascotas en \setulcolor{red}\ul{consultas}. Del dueño tenés que guardar su \setulcolor{blue}\ul{nombre}, \setulcolor{blue}\ul{teléfono} y \setulcolor{blue}\ul{dirección}. De la mascota tenés que guardar su \setulcolor{blue}\ul{nombre}, \setulcolor{blue}\ul{especie} y \setulcolor{blue}\ul{edad}. De la consulta tenés que guardar la \setulcolor{blue}\ul{fecha}, el \setulcolor{blue}\ul{diagnóstico} y el \setulcolor{blue}\ul{tratamiento}. De los veterinarios tenés que guardar su \setulcolor{blue}\ul{nombre}, \setulcolor{blue}\ul{especialidad} y \setulcolor{blue}\ul{matrícula}.
    \end{center}
\end{frame}

% Ejemplo práctico 1 - Diagrama
\begin{frame}{Ejemplo 1: Sistema de Mascotas y Veterinarios}
    \begin{center}
    \resizebox{0.95\textwidth}{!}{
    \begin{tikzpicture}[node distance=3cm]
        \node[entity] (dueno) {Dueño\nodepart{second}\textbf{id\_dueno (PK)}\\ nombre\\ telefono\\ direccion};
        \node[entity] (mascota) [right=of dueno] {Mascota\nodepart{second}\textbf{id\_mascota (PK)}\\ \underline{id\_dueno (FK)}\\ nombre\\ especie\\ edad};
        \node[entity] (consulta) [right=of mascota] {Consulta\nodepart{second}\textbf{id\_consulta (PK)}\\ \underline{id\_mascota (FK)}\\ \underline{id\_veterinario (FK)}\\ fecha\\ diagnostico\\ tratamiento};
        \node[entity] (veterinario) [right=of consulta] {Veterinario\nodepart{second}\textbf{id\_veterinario (PK)}\\ nombre\\ especialidad\\ matricula};
        
        % Conexiones
        \coordinate (fk_dueno) at ($(mascota.west)+(0,0.1)$);
        \coordinate (pk_dueno) at ($(dueno.east)+(0,0.1)$);
        \draw[-latex] (fk_dueno) to[out=180,in=0] 
            node[pos=0.2,above] {1..N} 
            node[pos=0.8,above] {1..1} 
            (pk_dueno);
            
        \coordinate (fk_mascota) at ($(consulta.west)+(0,0.1)$);
        \coordinate (pk_mascota) at ($(mascota.east)+(0,0.1)$);
        \draw[-latex] (fk_mascota) to[out=180,in=0] 
            node[pos=0.2,above] {1..N} 
            node[pos=0.8,above] {1..1} 
            (pk_mascota);
            
        \coordinate (fk_veterinario) at ($(consulta.east)+(0,-0.5)$);
        \coordinate (pk_veterinario) at ($(veterinario.west)+(0,0.5)$);
        \draw[-latex] (fk_veterinario) to[out=0,in=180] 
            node[pos=0.2,above] {1..N} 
            node[pos=0.8,above] {1..1} 
            (pk_veterinario);
    \end{tikzpicture}
    }
    \end{center}
\end{frame}

% Ejemplo 2 - Enunciado
\begin{frame}{Ejemplo 2: Sistema de Eventos y Organizadores}
    \begin{center}
        \small
        Tenés que diseñar un sistema para gestionar eventos. Los organizadores crean eventos, y los participantes se inscriben a estos eventos. Cada evento tiene un organizador principal, pero puede tener varios organizadores colaboradores. De los organizadores tenés que guardar su nombre, email y teléfono. De los eventos tenés que guardar su nombre, fecha, lugar y capacidad. De las inscripciones tenés que guardar la fecha de inscripción y el estado. De los participantes tenés que guardar su nombre, email y teléfono.
    \end{center}
\end{frame}

\begin{frame}{Ejemplo 2: Sistema de Eventos y Organizadores}
    \begin{center}
        \small
        Tenés que diseñar un sistema para gestionar eventos. Los \setulcolor{red}\ul{organizadores} crean \setulcolor{red}\ul{eventos}, y los \setulcolor{red}\ul{participantes} se inscriben a estos eventos. Cada evento tiene un organizador principal, pero puede tener varios organizadores colaboradores. De los organizadores tenés que guardar su \setulcolor{blue}\ul{nombre}, \setulcolor{blue}\ul{email} y \setulcolor{blue}\ul{teléfono}. De los eventos tenés que guardar su \setulcolor{blue}\ul{nombre}, \setulcolor{blue}\ul{fecha}, \setulcolor{blue}\ul{lugar} y \setulcolor{blue}\ul{capacidad}. De las \setulcolor{red}\ul{inscripciones} tenés que guardar la \setulcolor{blue}\ul{fecha de inscripción} y el \setulcolor{blue}\ul{estado}. De los participantes tenés que guardar su \setulcolor{blue}\ul{nombre}, \setulcolor{blue}\ul{email} y \setulcolor{blue}\ul{teléfono}.
    \end{center}
\end{frame}

% Ejemplo 2 - Diagrama
\begin{frame}{Ejemplo 2: Sistema de Eventos y Organizadores}
    \begin{center}
    \resizebox{0.95\textwidth}{!}{
    \begin{tikzpicture}[node distance=2.5cm]
        \node[entity] (organizador) {Organizador\nodepart{second}\textbf{id\_organizador (PK)}\\ nombre\\ email\\ telefono};
        \node[entity] (evento) [right=of organizador] {Evento\nodepart{second}\textbf{id\_evento (PK)}\\ \underline{id\_organizador (FK)}\\ nombre\\ fecha\\ lugar\\ capacidad};
        \node[entity] (inscripcion) [right=of evento] {Inscripcion\nodepart{second}\textbf{id\_inscripcion (PK)}\\ \underline{id\_evento (FK)}\\ \underline{id\_participante (FK)}\\ fecha\_inscripcion\\ estado};
        \node[entity] (participante) [right=of inscripcion] {Participante\nodepart{second}\textbf{id\_participante (PK)}\\ nombre\\ email\\ telefono};
        
        % Conexiones
        \coordinate (fk_organizador) at ($(evento.west)+(0,0.1)$);
        \coordinate (pk_organizador) at ($(organizador.east)+(0,0.1)$);
        \draw[-latex] (fk_organizador) to[out=180,in=0] 
            node[pos=0.2,above] {1..N} 
            node[pos=0.8,above] {1..1} 
            (pk_organizador);
            
        \coordinate (fk_evento) at ($(inscripcion.west)+(0,0.1)$);
        \coordinate (pk_evento) at ($(evento.east)+(0,0.1)$);
        \draw[-latex] (fk_evento) to[out=180,in=0] 
            node[pos=0.2,above] {1..N} 
            node[pos=0.8,above] {1..1} 
            (pk_evento);
            
        \coordinate (fk_participante) at ($(inscripcion.east)+(0,-0.5)$);
        \coordinate (pk_participante) at ($(participante.west)+(0,0.5)$);
        \draw[-latex] (fk_participante) to[out=0,in=180] 
            node[pos=0.2,above] {1..N} 
            node[pos=0.8,above] {1..1} 
            (pk_participante);
    \end{tikzpicture}
    }
    \end{center}
\end{frame}

% Ejemplo 3 - Enunciado
\begin{frame}{Ejemplo 3: Sistema de Cursos Online}
    \begin{center}
        Pensá en una plataforma de cursos online donde hay personas que se encargan de armar y dictar los cursos, y otras que se suman como estudiantes. Cada curso está pensado y gestionado por un instructor, tiene su propio nombre, una descripción que lo presenta y un precio que lo diferencia. Los cursos se componen de varias lecciones, cada una con su propio contenido, un título que la identifica y una duración que indica cuánto tiempo lleva completarla. Los estudiantes pueden anotarse en los cursos que les interesan y, a medida que avanzan, ir marcando qué lecciones ya terminaron. En el sistema, cada vez que alguien se suma a un curso queda registrado cuándo lo hizo, y también se guarda cuándo completó cada lección. Tanto los instructores como los estudiantes tienen su información personal y de contacto, y en el caso de los instructores, también se sabe en qué área se especializan. Los estudiantes, además, tienen registrada la fecha en que se unieron a la plataforma.
        \small
    \end{center}
\end{frame}

% Ejemplo 3 - Enunciado
\begin{frame}{Ejemplo 3: Sistema de Cursos Online}
    \begin{center}
        Pensá en una plataforma de cursos online donde hay personas que se encargan de armar y dictar los cursos, y otras que se suman como estudiantes. Cada \setulcolor{red}\ul{curso} está pensado y gestionado por un \setulcolor{red}\ul{instructor}, tiene su propio \setulcolor{blue}\ul{nombre}, una \setulcolor{blue}\ul{descripción} que lo presenta y un \setulcolor{blue}\ul{precio} que lo diferencia. Los cursos se componen de varias \setulcolor{red}\ul{lecciones}, cada una con su propio \setulcolor{blue}\ul{contenido}, un \setulcolor{blue}\ul{título} que la identifica y una \setulcolor{blue}\ul{duración} que indica cuánto tiempo lleva completarla. Los \setulcolor{red}\ul{estudiantes} pueden \setulcolor{red}\ul{anotarse} en los cursos que les interesan y, a medida que \setulcolor{red}\ul{avanzan}, ir marcando qué lecciones ya terminaron. En el sistema, cada vez que alguien se suma a un curso \setulcolor{blue}\ul{queda registrado cuándo lo hizo}, y también se guarda \setulcolor{blue}{cuándo completó} cada lección. Tanto los instructores como los estudiantes tienen su \setulcolor{blue}\ul{información personal y de contacto}, y en el caso de los instructores, también se sabe en qué \setulcolor{blue}{área se especializan}. Los estudiantes, además, tienen registrada la \setulcolor{blue}\ul{fecha} en que se unieron a la plataforma.
        \small
    \end{center}
\end{frame}

% Ejemplo 3 - Diagrama
\begin{frame}{Ejemplo 3: Sistema de Cursos Online}
    \begin{center}
    \resizebox{0.95\textwidth}{!}{
    \begin{tikzpicture}[node distance=2cm]
        \node[entity] (instructor) {Instructor\nodepart{second}\textbf{id\_instructor (PK)}\\ nombre\\ email\\ especialidad};
        \node[entity] (curso) [right=of instructor] {Curso\nodepart{second}\textbf{id\_curso (PK)}\\ \underline{id\_instructor (FK)}\\ titulo\\ descripcion\\ precio};
        \node[entity] (leccion) [right=of curso] {Leccion\nodepart{second}\textbf{id\_leccion (PK)}\\ \underline{id\_curso (FK)}\\ titulo\\ contenido\\ duracion};
        \node[entity] (inscripcion) [below=of curso] {Inscripcion\nodepart{second}\textbf{id\_inscripcion (PK)}\\ \underline{id\_curso (FK)}\\ \underline{id\_estudiante (FK)}\\ fecha\_inscripcion};
        \node[entity] (estudiante) [right=of inscripcion] {Estudiante\nodepart{second}\textbf{id\_estudiante (PK)}\\ nombre\\ email\\ fecha\_registro};
        \node[entity] (progreso) [right=of estudiante] {Progreso\nodepart{second}\textbf{id\_progreso (PK)}\\ \underline{id\_leccion (FK)}\\ \underline{id\_estudiante (FK)}\\ fecha\_completado};
        
        % Conexiones instructor-curso
        \coordinate (fk_instructor) at ($(curso.west)+(0,0.1)$);
        \coordinate (pk_instructor) at ($(instructor.east)+(0,0.1)$);
        \draw[-latex] (fk_instructor) to[out=180,in=0] 
            node[pos=0.2,above] {1..N} 
            node[pos=0.8,above] {1..1} 
            (pk_instructor);
            
        % Conexiones curso-leccion
        \coordinate (fk_curso) at ($(leccion.west)+(0,0.1)$);
        \coordinate (pk_curso) at ($(curso.east)+(0,0.1)$);
        \draw[-latex] (fk_curso) to[out=180,in=0] 
            node[pos=0.2,above] {1..N} 
            node[pos=0.8,above] {1..1} 
            (pk_curso);
            
        % Conexiones inscripcion-curso
        \coordinate (fk_curso_insc) at ($(inscripcion.north)+(0,0.1)$);
        \coordinate (pk_curso_insc) at ($(curso.south)+(0,-0.1)$);
        \draw[-latex] (fk_curso_insc) to[out=90,in=-90] 
            node[pos=0.2,right] {1..N} 
            node[pos=0.8,right] {1..1} 
            (pk_curso_insc);
            
        % Conexiones inscripcion-estudiante
        \coordinate (fk_estudiante_insc) at ($(inscripcion.east)+(0,-0.5)$);
        \coordinate (pk_estudiante_insc) at ($(estudiante.west)+(0,0.5)$);
        \draw[-latex] (fk_estudiante_insc) to[out=0,in=180] 
            node[pos=0.2,above] {1..N} 
            node[pos=0.8,above] {1..1} 
            (pk_estudiante_insc);
            
        % Conexiones progreso-leccion
        \coordinate (fk_leccion_prog) at ($(progreso.west)+(0,0.1)$);
        \coordinate (pk_leccion_prog) at ($(leccion.east)+(0,-0.1)$);
        \draw[-latex] (fk_leccion_prog) to[out=180,in=0] 
            node[pos=0.2,right] {1..N} 
            node[pos=0.8,right] {1..1} 
            (pk_leccion_prog);
            
        % Conexiones progreso-estudiante
        \coordinate (fk_estudiante_prog) at ($(progreso.west)+(0,-0.5)$);
        \coordinate (pk_estudiante_prog) at ($(estudiante.east)+(0,0.5)$);
        \draw[-latex] (fk_estudiante_prog) to[out=90,in=-90] 
            node[pos=0.3,below] {1..N} 
            node[pos=0.7,below] {1..1} 
            (pk_estudiante_prog);
    \end{tikzpicture}
    }
    \end{center}
\end{frame}

% Ejemplo 4 - Enunciado
\begin{frame}{Ejemplo 4: Sistema de Gestión de Inventario}
    \begin{center}
        \small
        Queremos modelar un sistema para una tienda que maneja inventario. La tienda tiene distintas \setulcolor{red}\ul{categorías}, y de cada categoría vamos a guardar su \setulcolor{blue}\ul{nombre} y \setulcolor{blue}\ul{descripción}. Cada \setulcolor{red}\ul{producto} pertenece a una categoría, y de cada producto necesitamos guardar su \setulcolor{blue}\ul{nombre}, \setulcolor{blue}\ul{precio} y \setulcolor{blue}\ul{stock}. Los \setulcolor{red}\ul{proveedores} suministran productos; de cada proveedor guardamos su \setulcolor{blue}\ul{nombre}, \setulcolor{blue}\ul{teléfono} y \setulcolor{blue}\ul{dirección}. También necesitamos registrar los \setulcolor{red}\ul{suministros}, donde para cada suministro se guarda la \setulcolor{blue}\ul{fecha} y la \setulcolor{blue}\ul{cantidad} de productos suministrados. Por otro lado, hay \setulcolor{red}\ul{ventas}: de cada venta guardamos la \setulcolor{blue}\ul{fecha} y el \setulcolor{blue}\ul{total}, y \setulcolor{red}\ul{para cada producto vendido} en una venta registramos la \setulcolor{blue}\ul{cantidad} y el \setulcolor{blue}\ul{precio unitario}.
    \end{center}
\end{frame}

% Ejemplo 4 - Diagrama
\begin{frame}{Ejemplo 4: Sistema de Gestión de Inventario}
    \begin{center}
    \resizebox{0.95\textwidth}{!}{
    \begin{tikzpicture}[node distance=2cm]
        \node[entity] (categoria) {Categoria\nodepart{second}\textbf{id\_categoria (PK)}\\ nombre\\ descripcion};
        \node[entity] (producto) [right=of categoria] {Producto\nodepart{second}\textbf{id\_producto (PK)}\\ \underline{id\_categoria (FK)}\\ nombre\\ precio\\ stock};
        \node[entity] (suministro) [right=of producto] {Suministro\nodepart{second}\textbf{id\_suministro (PK)}\\ \underline{id\_proveedor (FK)}\\ \underline{id\_producto (FK)}\\ fecha\\ cantidad};
        \node[entity] (proveedor) [right=of suministro] {Proveedor\nodepart{second}\textbf{id\_proveedor (PK)}\\ nombre\\ telefono\\ direccion};
        \node[entity] (venta) [below=of producto] {Venta\nodepart{second}\textbf{id\_venta (PK)}\\ fecha\\ total};
        \node[entity] (detalle) [right=of venta] {DetalleVenta\nodepart{second}\textbf{id\_detalle (PK)}\\ \underline{id\_venta (FK)}\\ \underline{id\_producto (FK)}\\ cantidad\\ precio\_unitario};
        
        % Conexiones categoria-producto
        \coordinate (fk_categoria) at ($(producto.west)+(0,0.1)$);
        \coordinate (pk_categoria) at ($(categoria.east)+(0,0.1)$);
        \draw[-latex] (fk_categoria) to[out=180,in=0] 
            node[pos=0.2,above] {1..N} 
            node[pos=0.8,above] {1..1} 
            (pk_categoria);
            
        % Conexiones producto-suministro
        \coordinate (fk_producto_sup) at ($(suministro.west)+(0,0.1)$);
        \coordinate (pk_producto_sup) at ($(producto.east)+(0,0.1)$);
        \draw[-latex] (fk_producto_sup) to[out=180,in=0] 
            node[pos=0.2,above] {1..N} 
            node[pos=0.8,above] {1..1} 
            (pk_producto_sup);
            
        % Conexiones suministro-proveedor
        \coordinate (fk_proveedor) at ($(suministro.east)+(0,0.1)$);
        \coordinate (pk_proveedor) at ($(proveedor.west)+(0,0.1)$);
        \draw[-latex] (fk_proveedor) to[out=0,in=180] 
            node[pos=0.2,above] {1..N} 
            node[pos=0.8,above] {1..1} 
            (pk_proveedor);
            
        % Conexiones venta-detalle
        \coordinate (fk_venta) at ($(detalle.west)+(0,0.1)$);
        \coordinate (pk_venta) at ($(venta.east)+(0,0.1)$);
        \draw[-latex] (fk_venta) to[out=180,in=0] 
            node[pos=0.2,above] {1..N} 
            node[pos=0.8,above] {1..1} 
            (pk_venta);
            
        % Conexiones detalle-producto
        \coordinate (fk_producto_det) at ($(detalle.west)+(0,-0.5)$);
        \coordinate (pk_producto_det) at ($(producto.east)+(0,0.7)$);
        \draw[-latex] (fk_producto_det) to[out=125,in=40] 
            node[pos=0.3,left] {1..N} 
            node[pos=0.6,left] {1..1} 
            (pk_producto_det);
    \end{tikzpicture}
    }
    \end{center}
\end{frame}

\begin{frame}{Terminamos}
    \begin{center}
        \Large{\textbf{¿Dudas?\\¿Consultas?}}
    \end{center}
    \begin{tikzpicture}[remember picture,overlay]
        \node[anchor=south,inner sep=30pt] at (current page.south) {
            \includegraphics[height=1cm]{../misc/UdeSA.png}
        };
    \end{tikzpicture}
\end{frame}

\end{document}
