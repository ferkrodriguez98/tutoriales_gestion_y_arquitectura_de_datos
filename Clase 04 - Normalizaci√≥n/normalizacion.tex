\documentclass[12pt]{article}

\usepackage[utf8]{inputenc}
\usepackage[T1]{fontenc}
\usepackage{lmodern}
\usepackage[spanish]{babel}
\usepackage{booktabs}
\usepackage{amsmath}
\usepackage{forest}
\usepackage{float}
\usepackage{listings}
\usepackage{xcolor}
\usepackage{tikz}

\definecolor{codegreen}{rgb}{0,0.6,0}
\definecolor{codegray}{rgb}{0.5,0.5,0.5}
\definecolor{codepurple}{rgb}{0.58,0,0.82}
\definecolor{backcolour}{rgb}{0.95,0.95,0.92}

\lstdefinestyle{mystyle}{
    backgroundcolor=\color{backcolour},   
    commentstyle=\color{codegreen},
    keywordstyle=\color{magenta},
    numberstyle=\tiny\color{codegray},
    stringstyle=\color{codepurple},
    basicstyle=\ttfamily\footnotesize,
    breakatwhitespace=false,         
    breaklines=true,                 
    captionpos=b,                    
    keepspaces=true,                 
    numbers=left,                    
    numbersep=5pt,                  
    showspaces=false,                
    showstringspaces=false,
    showtabs=false,                  
    tabsize=2
}

\lstset{style=mystyle}

\sloppy
\setlength{\parindent}{0pt}

\begin{document}

\begin{center}
  {\LARGE \textbf{Normalización de Bases de Datos}}\\[0.5em]
  {Gestión y Arquitectura de Datos, Universidad de San Andrés}
\end{center}

Si encuentran algún error en el documento o hay alguna duda, mandenmé un mail a rodriguezf@udesa.edu.ar y lo revisamos.

\section{Introducción}
La normalización es un proceso de diseño de bases de datos que reduce la redundancia y las anomalías de actualización. Se aplica a través de una serie de formas normales, cada una construyendo sobre la anterior.

\section{Primera Forma Normal (1FN)}
Una tabla está en Primera Forma Normal si:
\begin{itemize}
    \item Cada columna contiene valores atómicos (no divisibles)
    \item No hay grupos repetitivos
    \item Existe una clave primaria
\end{itemize}

\subsection{Ejemplo - Violación de 1FN}
Tabla: Estudiantes (No normalizada)
\begin{center}
\begin{tabular}{llll}
\toprule
ID & Nombre & Teléfonos & Materias \\
\midrule
1 & Juan & 1234567, 7654321 & Matemática, Física \\
2 & María & 2345678 & Química, Biología, Física \\
\bottomrule
\end{tabular}
\end{center}

\subsection{Solución en 1FN}

Tabla: Estudiantes
\begin{center}
\begin{tabular}{lll}
\toprule
ID & Nombre & Teléfono \\
\midrule
1 & Juan & 1234567 \\
1 & Juan & 7654321 \\
2 & María & 2345678 \\
\bottomrule
\end{tabular}
\end{center}

Tabla: EstudiantesMaterias
\begin{center}
\begin{tabular}{ll}
\toprule
ID\_Estudiante & Materia \\
\midrule
1 & Matemática \\
1 & Física \\
2 & Química \\
2 & Biología \\
2 & Física \\
\bottomrule
\end{tabular}
\end{center}

\section{Segunda Forma Normal (2FN)}
Una tabla está en Segunda Forma Normal si:
\begin{itemize}
    \item Está en 1FN
    \item Todos los atributos no clave dependen completamente de la clave primaria
\end{itemize}

\subsection{Ejemplo - Violación de 2FN}
Tabla: VentasProductos (en 1FN pero no en 2FN)
\begin{center}
\begin{tabular}{lllll}
\toprule
ID\_Venta & ID\_Producto & Cantidad & Precio\_Producto & Categoría\_Producto \\
\midrule
1 & P1 & 2 & 100 & Electrónica \\
1 & P2 & 1 & 200 & Hogar \\
2 & P1 & 3 & 100 & Electrónica \\
\bottomrule
\end{tabular}
\end{center}

Aquí, Precio\_Producto y Categoría\_Producto solo dependen de ID\_Producto, no de la clave completa (ID\_Venta, ID\_Producto).

\subsection{Solución en 2FN}

Tabla: Ventas
\begin{center}
\begin{tabular}{lll}
\toprule
ID\_Venta & ID\_Producto & Cantidad \\
\midrule
1 & P1 & 2 \\
1 & P2 & 1 \\
2 & P1 & 3 \\
\bottomrule
\end{tabular}
\end{center}

Tabla: Productos
\begin{center}
\begin{tabular}{lll}
\toprule
ID\_Producto & Precio & Categoría \\
\midrule
P1 & 100 & Electrónica \\
P2 & 200 & Hogar \\
\bottomrule
\end{tabular}
\end{center}

\section{Tercera Forma Normal (3FN)}
Una tabla está en Tercera Forma Normal si:
\begin{itemize}
    \item Está en 2FN
    \item No hay dependencias transitivas (atributos no clave que dependen de otros atributos no clave)
\end{itemize}

\subsection{Ejemplo - Violación de 3FN}
Tabla: Empleados (en 2FN pero no en 3FN)
\begin{center}
\begin{tabular}{llll}
\toprule
ID\_Empleado & Departamento & ID\_Jefe & Nombre\_Jefe \\
\midrule
1 & Ventas & J1 & Ana López \\
2 & IT & J2 & Pedro Gómez \\
3 & Ventas & J1 & Ana López \\
\bottomrule
\end{tabular}
\end{center}

Aquí, Nombre\_Jefe depende transitivamente de ID\_Empleado a través de ID\_Jefe.

\subsection{Solución en 3FN}

Tabla: Empleados
\begin{center}
\begin{tabular}{lll}
\toprule
ID\_Empleado & Departamento & ID\_Jefe \\
\midrule
1 & Ventas & J1 \\
2 & IT & J2 \\
3 & Ventas & J1 \\
\bottomrule
\end{tabular}
\end{center}

Tabla: Jefes
\begin{center}
\begin{tabular}{ll}
\toprule
ID\_Jefe & Nombre \\
\midrule
J1 & Ana López \\
J2 & Pedro Gómez \\
\bottomrule
\end{tabular}
\end{center}

\section{Ejemplo - TravelPro}

\subsection{Consigna}
La empresa TravelPro gestiona reservas de clientes en hoteles internacionales. Actualmente, la información está almacenada en una única tabla con la siguiente estructura:

\vspace{1em}

\begin{minipage}[t]{0.48\textwidth}
\begin{itemize}
    \item IDCliente
    \item ClienteNombre
    \item ClienteEmail *
    \item IDHotel
    \item HotelNombre
    \item IdCiudadHotel
    \item CiudadHotel
    \item IdPaisHotel
\end{itemize}
\end{minipage}
\hfill
\begin{minipage}[t]{0.48\textwidth}
\begin{itemize}
    \item NombrePaisHotel
    \item FechaCheckIn
    \item FechaCheckOut
    \item HabitaciónTipo
    \item IdGerenteHotel
    \item NombreGerenteHotel
    \item IdiomasGerenteHotel **
    \item NivelIdiomaGerenteHotel ***
\end{itemize}
\end{minipage}

\vspace{2em}

{\small * ClienteEmail: un cliente puede tener varios emails} \\
{\small ** IdiomasGerenteHotel: cada gerente puede hablar varios idiomas} \\
{\small *** NivelIdiomaGerenteHotel: indica el nivel de cada idioma que habla el gerente} \\

Consideraciones:
\begin{itemize}
    \item Un cliente puede alojarse varias veces en distintos hoteles (en distintas fechas)
    \item Cada hotel está ubicado en una única ciudad, y cada ciudad pertenece a un país
    \item Cada hotel tiene un único gerente asignado
    \item Cada reserva corresponde a un cliente, un hotel y queda registrada con su fecha de check-in y check-out
\end{itemize}

\subsection{Análisis de la Clave Primaria}
Para la tabla de reservas de TravelPro, la clave primaria mínima debe ser:
\begin{itemize}
    \item IDCliente (identifica al cliente)
    \item IDHotel (identifica el hotel)
    \item FechaCheckIn (identifica cuándo comienza la reserva)
\end{itemize}

Esta combinación es necesaria porque:
\begin{itemize}
    \item Un cliente puede tener múltiples reservas
    \item Un hotel puede tener múltiples reservas
    \item Un cliente puede reservar en el mismo hotel en diferentes fechas
\end{itemize}

\subsection{Normalización a 3FN}

\subsubsection{1FN - Eliminar atributos multivaluados}
Tabla: Clientes\_Emails
\begin{center}
\begin{tabular}{ll}
\toprule
IDCliente & ClienteEmail \\
\midrule
1 & cliente1@email.com \\
1 & cliente1@alternativo.com \\
2 & cliente2@email.com \\
\bottomrule
\end{tabular}
\end{center}

Tabla: Gerentes\_Idiomas
\begin{center}
\begin{tabular}{lll}
\toprule
IdGerenteHotel & Idioma & NivelIdioma \\
\midrule
G1 & Inglés & Avanzado \\
G1 & Español & Intermedio \\
G2 & Inglés & Básico \\
\bottomrule
\end{tabular}
\end{center}

\subsubsection{2FN - Eliminar dependencias parciales}
Tabla: Hoteles
\begin{center}
\begin{tabular}{lllll}
\toprule
IDHotel & HotelNombre & IdCiudadHotel & IdGerenteHotel & NombreGerenteHotel \\
\midrule
H1 & Hotel A & C1 & G1 & Juan Pérez \\
H2 & Hotel B & C2 & G2 & María López \\
\bottomrule
\end{tabular}
\end{center}

Tabla: Ciudades
\begin{center}
\begin{tabular}{lll}
\toprule
IdCiudadHotel & CiudadHotel & IdPaisHotel \\
\midrule
C1 & Buenos Aires & P1 \\
C2 & Madrid & P2 \\
\bottomrule
\end{tabular}
\end{center}

Tabla: Paises
\begin{center}
\begin{tabular}{ll}
\toprule
IdPaisHotel & NombrePaisHotel \\
\midrule
P1 & Argentina \\
P2 & España \\
\bottomrule
\end{tabular}
\end{center}

\subsubsection{3FN - Eliminar dependencias transitivas}
Tabla: Reservas
\begin{center}
\begin{tabular}{lllll}
\toprule
IDCliente & IDHotel & FechaCheckIn & FechaCheckOut & HabitaciónTipo \\
\midrule
1 & H1 & 2024-01-01 & 2024-01-05 & Simple \\
1 & H2 & 2024-02-01 & 2024-02-03 & Doble \\
2 & H1 & 2024-03-01 & 2024-03-10 & Suite \\
\bottomrule
\end{tabular}
\end{center}

Tabla: Clientes
\begin{center}
\begin{tabular}{ll}
\toprule
IDCliente & ClienteNombre \\
\midrule
1 & Juan García \\
2 & María López \\
\bottomrule
\end{tabular}
\end{center}

\subsection{Explicación de la Normalización}
\begin{itemize}
    \item Se separaron los emails de clientes en una tabla aparte (1FN)
    \item Se separaron los idiomas de los gerentes en una tabla aparte (1FN)
    \item Se extrajo la información de hoteles, ciudades y países en tablas separadas (2FN)
    \item Se eliminaron las dependencias transitivas separando la información de clientes (3FN)
\end{itemize}

\section{Ejercicio Práctico}
Normalizar la siguiente tabla hasta 3FN:
\begin{center}
\begin{tabular}{lllllll}
\toprule
ID\_Pedido & Cliente & Ciudad\_Cliente & Producto & Categoría & Precio & Stock \\
\midrule
1 & Juan & CABA & Laptop & Electrónica & 1000 & 50 \\
2 & María & CABA & Mouse & Electrónica & 20 & 100 \\
3 & Juan & CABA & Mouse & Electrónica & 20 & 100 \\
\bottomrule
\end{tabular}
\end{center}

\subsection{Solución}

\subsubsection{1FN - Eliminar atributos multivaluados}
Ya está en 1FN (valores atómicos)

\subsubsection{2FN - Eliminar dependencias parciales}

Tabla: Pedidos
\begin{center}
\begin{tabular}{lll}
\toprule
ID\_Pedido & ID\_Cliente & ID\_Producto \\
\midrule
1 & C1 & P1 \\
2 & C2 & P2 \\
3 & C1 & P2 \\
\bottomrule
\end{tabular}
\end{center}

Tabla: Productos
\begin{center}
\begin{tabular}{llll}
\toprule
ID\_Producto & Nombre & Categoría & Precio \\
\midrule
P1 & Laptop & Electrónica & 1000 \\
P2 & Mouse & Electrónica & 20 \\
\bottomrule
\end{tabular}
\end{center}

\subsubsection{3FN - Eliminar dependencias transitivas}

Tabla: Pedidos (igual que en 2FN)
\begin{center}
\begin{tabular}{lll}
\toprule
ID\_Pedido & ID\_Cliente & ID\_Producto \\
\midrule
1 & C1 & P1 \\
2 & C2 & P2 \\
3 & C1 & P2 \\
\bottomrule
\end{tabular}
\end{center}

Tabla: Productos
\begin{center}
\begin{tabular}{lll}
\toprule
ID\_Producto & Nombre & ID\_Categoría \\
\midrule
P1 & Laptop & CAT1 \\
P2 & Mouse & CAT1 \\
\bottomrule
\end{tabular}
\end{center}

Tabla: Categorías
\begin{center}
\begin{tabular}{ll}
\toprule
ID\_Categoría & Nombre \\
\midrule
CAT1 & Electrónica \\
\bottomrule
\end{tabular}
\end{center}

Tabla: Clientes
\begin{center}
\begin{tabular}{lll}
\toprule
ID\_Cliente & Nombre & Ciudad \\
\midrule
C1 & Juan & CABA \\
C2 & María & CABA \\
\bottomrule
\end{tabular}
\end{center}

\end{document}
