% Diapositivas Normalizacion - Basado en template UdeSA

\documentclass{beamer}
\usetheme{metropolis}

\usepackage[spanish]{babel}
\usepackage[utf8]{inputenc}
\usepackage{graphicx}
\usepackage{tikz}
\usepackage{xcolor}
\usepackage{amsmath}
\usepackage{fontawesome5}

\usetikzlibrary{positioning,shapes.multipart,calc,arrows,shapes.geometric}

% Colores y configuración del template
\definecolor{primary}{RGB}{46, 204, 113}
\definecolor{secondary}{RGB}{52, 152, 219}
\definecolor{accent}{RGB}{231, 76, 60}
\definecolor{background}{RGB}{236, 240, 241}
\setbeamercolor{normal text}{fg=black,bg=background}
\setbeamercolor{structure}{fg=primary}
\setbeamercolor{alerted text}{fg=accent}

% Título
\title{Normalización de Bases de Datos}
\author{Gestión y Arquitectura de Datos}
\date{}

\begin{document}

% Portada
\begin{frame}
    \titlepage
    \begin{tikzpicture}[remember picture,overlay]
        \node[anchor=south west,inner sep=30pt] at (current page.south west) {
            \includegraphics[height=1cm]{../misc/UdeSA.png}
        };
    \end{tikzpicture}
\end{frame}

% Introducción
\begin{frame}{¿Qué es la normalización?}
    \begin{center}
        Proceso sistemático que busca:\\
        \vspace{0.6cm}
        \begin{tabular}{ccc}
            \begin{minipage}{2.4cm}
                \centering
                {\Large \faClone}\\[0.2cm]
                \textbf{Reducir redundancia}
            \end{minipage} &
            \begin{minipage}{2.4cm}
                \centering
                {\Large \faTools}\\[0.2cm]
                \textbf{Evitar anomalías}
            \end{minipage} &
            \begin{minipage}{2.4cm}
                \centering
                {\Large \faDatabase}\\[0.2cm]
                \textbf{Mejorar integridad}
            \end{minipage}
        \end{tabular}
    \end{center}
\end{frame}

% Formas normales overview
\begin{frame}{Formas normales}
    \begin{itemize}
        \item<1-> Primera Forma Normal (1FN): valores atómicos, sin grupos repetitivos.
        \item<2-> Segunda Forma Normal (2FN): en 1FN y sin dependencias parciales respecto de la clave.
        \item<3-> Tercera Forma Normal (3FN): en 2FN y sin dependencias transitivas entre atributos no clave.
    \end{itemize}
\end{frame}

% 1FN definición
\begin{frame}{Primera Forma Normal (1FN)}
    Una tabla está en 1FN si:
    \begin{itemize}
        \item Cada columna contiene \textbf{un único valor atómico}.
        \item No existen grupos de columnas repetitivas.
        \item Cuenta con una clave primaria definida.
    \end{itemize}
\end{frame}

% 1FN ejemplo
\begin{frame}{Ejemplo de violacion de 1FN}
    \small
    \begin{center}
    \begin{tabular}{cccc}
        \hline
        ID & Nombre & Telefonos & Materias \\
        \hline
        1 & Juan & 1234567, 7654321 & Matemática, Física \\
        2 & Maria & 2345678 & Química, Biología, Física \\
        \hline
    \end{tabular}
    \end{center}
\end{frame}

% 1FN solución
\begin{frame}{Cómo llevar a 1FN}
    \small
    \begin{columns}
        \column{0.5\textwidth}
            \begin{center}
            \textbf{Estudiantes}\\[0.1cm]
            \begin{tabular}{ccc}
                \hline
                ID & Nombre & Teléfono \\
                \hline
                1 & Juan & 1234567 \\
                1 & Juan & 7654321 \\
                2 & Maria & 2345678 \\
                \hline
            \end{tabular}
            \end{center}
        \column{0.5\textwidth}
            \begin{center}
            \textbf{EstudiantesMaterias}\\[0.1cm]
            \begin{tabular}{cc}
                \hline
                ID\_Estudiante & Materia \\
                \hline
                1 & Matemática \\
                1 & Física \\
                2 & Química \\
                2 & Biología \\
                2 & Física \\
                \hline
            \end{tabular}
            \end{center}
    \end{columns}
\end{frame}

% 2FN definición
\begin{frame}{Segunda Forma Normal (2FN)}
    Requisitos:
    \begin{itemize}
        \item Estar en 1FN.
        \item Todos los atributos no clave dependen \textbf{de toda la clave primaria}, no de una parte de ella.
    \end{itemize}
\end{frame}

% 2FN ejemplo
\begin{frame}{Ejemplo de violación de 2FN}
    \small
    \begin{center}
    \begin{tabular}{ccccc}
        \hline
        ID\_Venta & ID\_Producto & Cantidad & Precio & Categoría \\
        \hline
        1 & P1 & 2 & 100 & Electrónica \\
        1 & P2 & 1 & 200 & Hogar \\
        2 & P1 & 3 & 100 & Electrónica \\
        \hline
    \end{tabular}
    \end{center}

    \vspace{0.4cm}
    \begin{center}
    \begin{itemize}
        \item Cantidad depende de \textit{ID\_Venta} y \textit{ID\_Producto}
        \item Precio y Categoría dependen de \textit{ID\_Producto}
    \end{itemize}
    \end{center}
\end{frame}

% 2FN solucion
\begin{frame}{Como llevar a 2FN}
    \begin{center}
        \textbf{Ventas}\\[0.1cm]
        \small
        \begin{tabular}{ccc}
            \hline
            ID\_Venta & ID\_Producto & Cantidad \\
            \hline
            1 & P1 & 2 \\
            1 & P2 & 1 \\
            2 & P1 & 3 \\
            \hline
        \end{tabular}
        \vspace{0.5cm}

        \textbf{Productos}\\[0.1cm]
        \small
        \begin{tabular}{ccc}
            \hline
            ID\_Producto & Precio & Categoría \\
            \hline
            P1 & 100 & Electrónica \\
            P2 & 200 & Hogar \\
            \hline
        \end{tabular}
    \end{center}
\end{frame}

% 3FN definición
\begin{frame}{Tercera Forma Normal (3FN)}
    Requisitos:
    \begin{itemize}
        \item Estar en 2FN.
        \item No existen \textbf{dependencias transitivas}: ningún atributo no clave depende de otro atributo no clave.
    \end{itemize}
\end{frame}

% 3FN ejemplo
\begin{frame}{Ejemplo de violación de 3FN}
    \small
    \begin{center}
    \begin{tabular}{cccc}
        \hline
        ID\_Empleado & Departamento & ID\_Jefe & Nombre\_Jefe \\
        \hline
        1 & Ventas & J1 & Ana López \\
        2 & IT & J2 & Pedro Gómez \\
        3 & Ventas & J1 & Ana López \\
        \hline
    \end{tabular}
    \end{center}
    \vspace{0.4cm}
    \begin{center}
        \begin{itemize}
            \item Nombre\_Jefe depende transitivamente de ID\_Empleado vía ID\_Jefe.
        \end{itemize}
    \end{center}
\end{frame}

% 3FN solución
\begin{frame}{Cómo llevar a 3FN}
    \begin{columns}
        \column{0.55\textwidth}
            \begin{center}
            \textbf{Empleados}\\[0.1cm]
            \small
            \begin{tabular}{ccc}
                \hline
                ID\_Empleado & Departamento & ID\_Jefe \\
                \hline
                1 & Ventas & J1 \\
                2 & IT & J2 \\
                3 & Ventas & J1 \\
                \hline
            \end{tabular}
            \end{center}
        \column{0.45\textwidth}
            \begin{center}
            \textbf{Jefes}\\[0.1cm]
            \small
            \begin{tabular}{cc}
                \hline
                ID\_Jefe & Nombre \\
                \hline
                J1 & Ana Lopez \\
                J2 & Pedro Gomez \\
                \hline
            \end{tabular}
            \end{center}
    \end{columns}
\end{frame}

% Consigna
\begin{frame}{TravelPro - Consigna}
    \small
    La empresa \textbf{TravelPro} gestiona reservas de clientes en hoteles internacionales. Actualmente toda la información se almacena en una única tabla, lo cual genera redundancia y anomalías. \\
    \vspace{0.4cm}
    \textbf{Consideraciones:}
    \begin{itemize}
        \item Un cliente puede alojarse varias veces en distintos hoteles.
        \item Cada hotel pertenece a una única ciudad y cada ciudad a un país.
        \item Cada hotel tiene un único gerente asignado.
        \item Cada reserva se identifica por cliente, hotel y fecha de check-in.
    \end{itemize}
\end{frame}

% Datos originales
\begin{frame}{TravelPro - Datos originales}
    \small
    \begin{columns}
        \column{0.5\textwidth}
            \begin{itemize}
                \item IDCliente
                \item ClienteNombre
                \item ClienteEmail (multivaluado)
                \item IDHotel
                \item HotelNombre
                \item IdCiudadHotel
                \item CiudadHotel
                \item IdPaisHotel
            \end{itemize}
        \column{0.5\textwidth}
            \begin{itemize}
                \item NombrePaisHotel
                \item FechaCheckIn / FechaCheckOut
                \item HabitacionTipo
                \item IdGerenteHotel
                \item NombreGerenteHotel
                \item IdiomasGerenteHotel (multivaluado)
                \item NivelIdiomaGerenteHotel
            \end{itemize}
    \end{columns}
\end{frame}

% Clave primaria
\begin{frame}{TravelPro - Clave primaria}
    \small
    Para identificar unívocamente cada reserva se necesita la combinación:
    \begin{center}
        \textbf{IDCliente \quad + \quad IDHotel \quad + \quad FechaCheckIn}
    \end{center}
    \vspace{0.4cm}
    \begin{itemize}
        \item Un mismo cliente puede mantener varias reservas.
        \item Un hotel tiene muchas reservas.
        \item El mismo cliente puede reservar el mismo hotel en fechas distintas.
    \end{itemize}
\end{frame}

% Paso 1 1FN
\begin{frame}{TravelPro - Paso 1: llevar a 1FN}
    \begin{center}
    \textbf{Clientes\_Emails}\\[0.1cm]
    \begin{tabular}{cc}
        \hline
        IDCliente & ClienteEmail \\
        \hline
        1 & cliente1@email.com \\
        1 & cliente1@alternativo.com \\
        2 & cliente2@email.com \\
        \hline
    \end{tabular}
    \vspace{0.6cm}
    \textbf{\\Gerentes\_Idiomas}\\[0.1cm]
    \begin{tabular}{ccc}
        \hline
        IdGerenteHotel & Idioma & Nivel \\
        \hline
        G1 & Ingles & Avanzado \\
        G1 & Espanol & Intermedio \\
        G2 & Ingles & Basico \\
        \hline
    \end{tabular}
    \end{center}
\end{frame}

% Paso 2 2FN
\begin{frame}{TravelPro - Paso 2: llevar a 2FN}
    \begin{center}
    \textbf{Hoteles}\\[0.1cm]
    \resizebox{\textwidth}{!}{
    \begin{tabular}{ccccc}
        \hline
        IDHotel & HotelNombre & IdCiudadHotel & IdGerenteHotel & NombreGerenteHotel \\
        \hline
        H1 & Hotel A & C1 & G1 & Juan Perez \\
        H2 & Hotel B & C2 & G2 & Maria Lopez \\
        \hline
    \end{tabular}
    }
    \vspace{0.4cm}

    \textbf{Ciudades}\\[0.1cm]
    \begin{tabular}{ccc}
        \hline
        IdCiudadHotel & CiudadHotel & IdPaisHotel \\
        \hline
        C1 & Buenos Aires & P1 \\
        C2 & Madrid & P2 \\
        \hline
    \end{tabular}
    \vspace{0.4cm}

    \textbf{Paises}\\[0.1cm]
    \begin{tabular}{cc}
        \hline
        IdPaisHotel & NombrePaisHotel \\
        \hline
        P1 & Argentina \\
        P2 & España \\
        \hline
    \end{tabular}
    \end{center}
\end{frame}

% Paso 3 3FN
\begin{frame}{TravelPro - Paso 3: llevar a 3FN}
    \begin{center}
    \textbf{Reservas}\\[0.1cm]
    \resizebox{\textwidth}{!}{
    \begin{tabular}{ccccc}
        \hline
        IDCliente & IDHotel & FechaCheckIn & FechaCheckOut & HabitacionTipo \\
        \hline
        1 & H1 & 2024-01-01 & 2024-01-05 & Simple \\
        1 & H2 & 2024-02-01 & 2024-02-03 & Doble \\
        2 & H1 & 2024-03-01 & 2024-03-10 & Suite \\
        \hline
    \end{tabular}
    }
    \vspace{0.6cm}
    \textbf{\\Clientes}\\[0.1cm]
    \begin{tabular}{cc}
        \hline
        IDCliente & ClienteNombre \\
        \hline
        1 & Juan Garcia \\
        2 & Maria Lopez \\
        \hline
    \end{tabular}
    \end{center}
\end{frame}

% Resumen pasos
\begin{frame}{Pasos para normalizar}
    \begin{enumerate}
        \item Identificar la clave primaria.
        \item Eliminar atributos multivaluados (1FN).
        \item Dividir dependencias parciales (2FN).
        \item Eliminar dependencias transitivas (3FN).
    \end{enumerate}
\end{frame}

% Ejercicio práctico
\begin{frame}{Ejercicio práctico - Ventas de productos}
    \small
    \begin{center}
    \begin{tabular}{ccccccc}
        \hline
        ID\_Pedido & Cliente & Ciudad & Producto & Categoría & Precio & Stock \\
        \hline
        1 & Juan & CABA & Laptop & Electrónica & 1000 & 50 \\
        2 & Maria & CABA & Mouse & Electrónica & 20 & 100 \\
        3 & Juan & CABA & Mouse & Electrónica & 20 & 100 \\
        \hline
    \end{tabular}
    \end{center}
\end{frame}

% Paso 2FN
\begin{frame}{Ejercicio práctico - Paso 2: 2FN}
    \small
    \begin{center}
    \textbf{Pedidos}\\[0.1cm]
    \begin{tabular}{ccc}
        \hline
        ID\_Pedido & ID\_Cliente & ID\_Producto \\
        \hline
        1 & C1 & P1 \\
        2 & C2 & P2 \\
        3 & C1 & P2 \\
        \hline
    \end{tabular}
    \vspace{0.6cm}
    \textbf{\\Productos}\\[0.1cm]
    \begin{tabular}{cccc}
        \hline
        ID\_Producto & Nombre & Categoria & Precio \\
        \hline
        P1 & Laptop & Electrónica & 1000 \\
        P2 & Mouse & Electrónica & 20 \\
        \hline
    \end{tabular}
    \end{center}
\end{frame}

% Paso 3FN (1)
\begin{frame}{Ejercicio práctico - Paso 3: 3FN (1)}
    \small
    \begin{center}
    \textbf{Productos}\\[0.1cm]
    \begin{tabular}{ccc}
        \hline
        ID\_Producto & Nombre & ID\_Categoria \\
        \hline
        P1 & Laptop & CAT1 \\
        P2 & Mouse & CAT1 \\
        \hline
    \end{tabular}
    \vspace{0.6cm}
    \textbf{\\Categorias}\\[0.1cm]
    \begin{tabular}{cc}
        \hline
        ID\_Categoria & Nombre \\
        \hline
        CAT1 & Electrónica \\
        \hline
    \end{tabular}
    \end{center}
\end{frame}

% Paso 3FN (2)
\begin{frame}{Ejercicio práctico - Paso 3: 3FN (2)}
    \small
    \begin{center}
    \textbf{Pedidos}\\[0.1cm]
    \begin{tabular}{ccc}
        \hline
        ID\_Pedido & ID\_Cliente & ID\_Producto \\
        \hline
        1 & C1 & P1 \\
        2 & C2 & P2 \\
        3 & C1 & P2 \\
        \hline
    \end{tabular}
    \vspace{0.6cm}
    \textbf{\\Clientes}\\[0.1cm]
    \begin{tabular}{ccc}
        \hline
        ID\_Cliente & Nombre & Ciudad \\
        \hline
        C1 & Juan & CABA \\
        C2 & Maria & CABA \\
        \hline
    \end{tabular}
    \end{center}
\end{frame}

% Cierre
\begin{frame}{Terminamos}
    \begin{center}
        \Large{\textbf{¿Dudas?\\¿Consultas?}}
    \end{center}
    \begin{tikzpicture}[remember picture,overlay]
        \node[anchor=south,inner sep=30pt] at (current page.south) {
            \includegraphics[height=1cm]{../misc/UdeSA.png}
        };
    \end{tikzpicture}
\end{frame}

\end{document}
