\documentclass[12pt]{article}

\usepackage[utf8]{inputenc}
\usepackage[T1]{fontenc}
\usepackage{lmodern}
\usepackage[spanish]{babel}
\usepackage{booktabs}
\usepackage{amsmath}
\usepackage{forest}
\usepackage{float}
\usepackage{listings}
\usepackage{xcolor}
\usepackage{tikz}

\definecolor{codegreen}{rgb}{0,0.6,0}
\definecolor{codegray}{rgb}{0.5,0.5,0.5}
\definecolor{codepurple}{rgb}{0.58,0,0.82}
\definecolor{backcolour}{rgb}{0.95,0.95,0.92}

\lstdefinestyle{mystyle}{
    backgroundcolor=\color{backcolour},   
    commentstyle=\color{codegreen},
    keywordstyle=\color{magenta},
    numberstyle=\tiny\color{codegray},
    stringstyle=\color{codepurple},
    basicstyle=\ttfamily\footnotesize,
    breakatwhitespace=false,         
    breaklines=true,                 
    captionpos=b,                    
    keepspaces=true,                 
    numbers=left,                    
    numbersep=5pt,                  
    showspaces=false,                
    showstringspaces=false,
    showtabs=false,                  
    tabsize=2
}

\lstset{style=mystyle}

\sloppy
\setlength{\parindent}{0pt}

\begin{document}

\begin{center}
  {\LARGE \textbf{Introducción a SQL}}\\[0.5em]
  {Gestión y Arquitectura de Datos, Universidad de San Andrés}
\end{center}

Si encuentran algún error en el documento o hay alguna duda, mandenmé un mail a rodriguezf@udesa.edu.ar y lo revisamos.

\section{¿Qué es SQL?}
SQL (Structured Query Language) es el lenguaje estándar para interactuar con bases de datos relacionales. Es un lenguaje declarativo, lo que significa que especificamos qué queremos obtener en lugar de cómo obtenerlo.

\vspace{1em}

Las principales operaciones que podemos realizar con SQL son:
\begin{itemize}
    \item Consultar datos (SELECT)
    \item Insertar datos (INSERT)
    \item Actualizar datos (UPDATE)
    \item Eliminar datos (DELETE)
    \item Crear y modificar estructuras de datos (CREATE, ALTER, DROP)
\end{itemize}

\section{Estructura Básica de una Consulta}
La estructura más básica de una consulta SQL es:

\begin{lstlisting}[language=SQL]
SELECT columna1, columna2
FROM tabla
WHERE condicion;
\end{lstlisting}

Resultado ejemplo:
\begin{center}
\begin{tabular}{ll}
\toprule
columna1 & columna2 \\
\midrule
valor1 & valor2 \\
valor3 & valor4 \\
\bottomrule
\end{tabular}
\end{center}

Donde:
\begin{itemize}
    \item SELECT especifica qué columnas queremos ver
    \item FROM indica de qué tabla obtenemos los datos
    \item WHERE (opcional) filtra los resultados según una condición
\end{itemize}

\section{Operadores y Funciones Básicas}
\subsection{Operadores de Comparación}
Los operadores de comparación nos van a servir para poder comparar valores entre dos cosas. Estos son:
\begin{itemize}
    \item = (igual)
    \item < (menor que)
    \item > (mayor que)
    \item <= (menor o igual que)
    \item >= (mayor o igual que)
    \item <> o != (distinto)
\end{itemize}

\vspace{1em}

Ejemplos:
\begin{lstlisting}[language=SQL]
-- Buscar salarios mayores a 50000
SELECT nombre, apellido, salario
FROM empleados
WHERE salario > 50000;
\end{lstlisting}

\begin{center}
\begin{tabular}{ccc}
\toprule
nombre & apellido & salario \\
\midrule
Juan & Pérez & 55000 \\
\bottomrule
\end{tabular}
\end{center}

\begin{lstlisting}[language=SQL]
-- Buscar apellidos distintos a Perez
SELECT nombre, apellido
FROM empleados
WHERE apellido <> 'Perez';
\end{lstlisting}

\begin{center}
\begin{tabular}{cc}
\toprule
nombre & apellido \\
\midrule
Juan & García \\
María & López \\
\bottomrule
\end{tabular}
\end{center}

\subsection{Operadores Lógicos}
Los operadores lógicos nos van a servir para poder combinar condiciones. Estos son:
\begin{itemize}
    \item AND
    \item OR
    \item NOT
\end{itemize}

\vspace{1em}

Ejemplos:
\begin{lstlisting}[language=SQL]
-- Buscar empleados con salario mayor a 50000 y que trabajen en el departamento 1
SELECT nombre, apellido, salario
FROM empleados
WHERE salario > 50000 AND nombre = 'Juan';
\end{lstlisting}

\begin{center}

\begin{tabular}{ccc}
\toprule
nombre & apellido & salario \\
\midrule
Juan & García & 55000 \\
\bottomrule
\end{tabular}
\end{center}

\begin{lstlisting}[language=SQL]
-- Buscar empleados con salario exactamente 42000 y que no sean de la ciudad de Buenos Aires
SELECT nombre, apellido, salario, ciudad
FROM empleados
WHERE salario = 42000 AND ciudad <> 'Buenos Aires';
\end{lstlisting}

\begin{center}
\begin{tabular}{cccc}
\toprule
nombre & apellido & salario & ciudad \\
\midrule
Marcos & Gómez & 42000 & Córdoba \\
\bottomrule
\end{tabular}
\end{center}

\subsection{Operador IN}
El operador IN se usa para buscar valores en una lista de valores que le damos.

\vspace{1em}

Ejemplos:
\begin{lstlisting}[language=SQL]
-- Buscar empleados que tengan de apellido Garcia o Lopez
SELECT nombre, apellido
FROM empleados
WHERE apellido IN ('Garcia', 'Lopez');
\end{lstlisting}

\begin{center}
\begin{tabular}{cc}
\toprule
nombre & apellido \\
\midrule
Juan & García \\
María & López \\
\bottomrule
\end{tabular}
\end{center}

\subsection{Operador BETWEEN}
El operador BETWEEN se usa para buscar valores entre dos valores que le damos.

\vspace{1em}

Ejemplos:
\begin{lstlisting}[language=SQL]
-- Buscar empleados con salario entre 40000 y 50000
SELECT nombre, apellido, salario
FROM empleados
WHERE salario BETWEEN 40000 AND 50000;
\end{lstlisting}

\begin{center}
\begin{tabular}{ccc}
\toprule
nombre & apellido & salario \\
\midrule
Marcos & Gómez & 42000 \\
\bottomrule
\end{tabular}
\end{center}

\subsection{Operador LIKE}
El operador LIKE se usa para buscar patrones en texto. Tenemos dos opciones, el \% y el \_. El \% representa cualquier cantidad de caracteres, mientras que el \_ representa un único caracter.

Ejemplos:
\begin{lstlisting}[language=SQL]
-- Busca apellidos que terminan en G
SELECT nombre, apellido
FROM empleados
WHERE apellido LIKE '%G';
\end{lstlisting}

\begin{center}
\begin{tabular}{cc}
\toprule
nombre & apellido \\
\midrule
Juan & Rodriguez \\
\bottomrule
\end{tabular}
\end{center}

\begin{lstlisting}[language=SQL]
-- Busca apellidos que tienen G como segundo caracter
SELECT nombre, apellido
FROM empleados
WHERE apellido LIKE '_arcia';
\end{lstlisting}

\begin{center}
\begin{tabular}{cc}
\toprule
nombre & apellido \\
\midrule
Maria & Garcia \\
\bottomrule
\end{tabular}
\end{center}

\begin{lstlisting}[language=SQL]
-- Busca apellidos que empiezan con G
SELECT nombre, apellido
FROM empleados
WHERE apellido LIKE 'G%';
\end{lstlisting}

\begin{center}
\begin{tabular}{cc}
\toprule
nombre & apellido \\
\midrule
Maria & Garcia \\
Carlos & Gonzalez \\
\bottomrule
\end{tabular}
\end{center}

\begin{lstlisting}[language=SQL]
-- Busca apellidos que contienen G en cualquier posicion
SELECT nombre, apellido
FROM empleados
WHERE apellido LIKE '%G%';
\end{lstlisting}

\begin{center}
\begin{tabular}{cc}
\toprule
nombre & apellido \\
\midrule
Juan & Rodriguez \\
Maria & Garcia \\
Carlos & Gonzalez \\
\bottomrule
\end{tabular}
\end{center}

\textbf{Nota:} No en todos los motores de SQL el \%G\% devolvería Rodríguez dado que es una g minúscula y no una G mayúscula como estamos pidiendo. En la mayoría sí lo devolvería.

\subsection{Valores NULL}
NULL representa la ausencia de un valor. Para trabajar con NULL usamos:
\begin{itemize}
    \item IS NULL
    \item IS NOT NULL
\end{itemize}

Ejemplo:
\begin{lstlisting}[language=SQL]
SELECT nombre, telefono
FROM empleados
WHERE telefono IS NULL;
\end{lstlisting}

\begin{center}
\begin{tabular}{cc}
\toprule
nombre & telefono \\
\midrule
Carlos & NULL \\
\bottomrule
\end{tabular}
\end{center}

\section{Ejemplos Prácticos}
\subsection{Seleccionar Todos los Registros}
\begin{lstlisting}[language=SQL]
SELECT *
FROM empleados;
\end{lstlisting}

\begin{center}
\begin{tabular}{llll}
\toprule
id & nombre & apellido & salario \\
\midrule
1 & Juan & Pérez & 50000 \\
2 & María & García & 60000 \\
3 & Carlos & López & 45000 \\
\bottomrule
\end{tabular}
\end{center}

\textbf{Nota:} Evitar usar SELECT *, en cambio es recomendable especificar las columnas que necesitamos. Esto es para evitar problemas de performance en bases de datos gigantes.

\subsection{Filtrar Registros}
\begin{lstlisting}[language=SQL]
SELECT nombre, apellido, salario
FROM empleados
WHERE salario > 50000;
\end{lstlisting}

\begin{center}
\begin{tabular}{lll}
\toprule
nombre & apellido & salario \\
\midrule
María & García & 60000 \\
\bottomrule
\end{tabular}
\end{center}

\subsection{Ordenar Resultados}
\begin{lstlisting}[language=SQL]
SELECT nombre, apellido, edad
FROM empleados
ORDER BY edad DESC;
\end{lstlisting}

\begin{center}
\begin{tabular}{lll}
\toprule
nombre & apellido & edad \\
\midrule
Carlos & López & 45 \\
María & García & 35 \\
Juan & Pérez & 30 \\
\bottomrule
\end{tabular}
\end{center}

\section{Alias}
Los alias nos permiten dar nombres temporales a tablas o columnas para hacer las consultas más legibles.

\subsection{Alias de Tablas}
\begin{lstlisting}[language=SQL]
SELECT e.nombre, d.nombre_departamento
FROM empleados AS e
JOIN departamentos AS d ON e.id_departamento = d.id;
\end{lstlisting}

\begin{center}
\begin{tabular}{cc}
\toprule
nombre & nombre\_departamento \\
\midrule
Juan & Ventas \\
María & IT \\
Carlos & Ventas \\
\bottomrule
\end{tabular}
\end{center}

\subsection{Alias de Columnas}
\begin{lstlisting}[language=SQL]
SELECT 
    nombre AS nombre_empleado,
    salario AS sueldo_mensual
FROM empleados;
\end{lstlisting}

\begin{center}
\begin{tabular}{cc}
\toprule
nombre\_empleado & sueldo\_mensual \\
\midrule
Juan & 50000 \\
María & 60000 \\
\bottomrule
\end{tabular}
\end{center}

\section{Funciones de Agregación}
Las funciones de agregación nos permiten realizar cálculos sobre grupos de registros.

\subsection{COUNT()}
Cuenta el número de registros que cumplen con una condición.

\begin{lstlisting}[language=SQL]
SELECT COUNT(*) as total_empleados
FROM empleados;
\end{lstlisting}

\begin{center}
\begin{tabular}{c}
\toprule
total\_empleados \\
\midrule
3 \\
\bottomrule
\end{tabular}
\end{center}

\subsection{SUM()}
Suma los valores de una columna numérica.

\begin{lstlisting}[language=SQL]
SELECT SUM(salario) as total_salarios
FROM empleados;
\end{lstlisting}

\begin{center}
\begin{tabular}{c}
\toprule
total\_salarios \\
\midrule
155000 \\
\bottomrule
\end{tabular}
\end{center}

\subsection{AVG()}
Calcula el promedio de los valores en una columna.

\begin{lstlisting}[language=SQL]
SELECT AVG(salario) as salario_promedio
FROM empleados;
\end{lstlisting}

\begin{center}
\begin{tabular}{c}
\toprule
salario\_promedio \\
\midrule
51666.67 \\
\bottomrule
\end{tabular}
\end{center}

\subsection{MAX()}
Encuentra el valor máximo en una columna.

\begin{lstlisting}[language=SQL]
SELECT MAX(salario) as salario_maximo
FROM empleados;
\end{lstlisting}

\begin{center}
\begin{tabular}{c}
\toprule
salario\_maximo \\
\midrule
60000 \\
\bottomrule
\end{tabular}
\end{center}

\subsection{MIN()}
Encuentra el valor mínimo en una columna.

\begin{lstlisting}[language=SQL]
SELECT MIN(salario) as salario_minimo
FROM empleados;
\end{lstlisting}

\begin{center}
\begin{tabular}{c}
\toprule
salario\_minimo \\
\midrule
45000 \\
\bottomrule
\end{tabular}
\end{center}

\subsection{Agrupación con GROUP BY}
Podemos combinar estas funciones con GROUP BY para obtener estadísticas por grupo:

\begin{lstlisting}[language=SQL]
SELECT 
    departamento,
    COUNT(*) as cantidad_empleados,
    AVG(salario) as salario_promedio
FROM empleados
GROUP BY departamento;
\end{lstlisting}

\begin{center}
\begin{tabular}{ccc}
\toprule
departamento & cantidad\_empleados & salario\_promedio \\
\midrule
Ventas & 2 & 55000 \\
IT & 1 & 45000 \\
\bottomrule
\end{tabular}
\end{center}

\begin{lstlisting}[language=SQL]
SELECT nombre, edad
FROM empleados
WHERE edad IS NOT NULL AND nombre LIKE '%er%'
\end{lstlisting}

\begin{center}
\begin{tabular}{cc}
\toprule
nombre & edad \\
\midrule
Fermín & 26 \\
Hernán & 42 \\
\bottomrule
\end{tabular}
\end{center}

\section{Joins}
Los JOINs nos permiten combinar filas de dos o más tablas basándonos en una condición de relación entre ellas.

\subsection{INNER JOIN}
Devuelve solo los registros que coinciden en ambas tablas. Ejemplo:
\begin{lstlisting}[language=SQL]
SELECT e.nombre, d.nombre_departamento
FROM empleados e
INNER JOIN departamentos d ON e.id_departamento = d.id;
\end{lstlisting}

\begin{center}
\begin{tabular}{cc}
\toprule
nombre & nombre\_departamento \\
\midrule
Juan & Ventas \\
María & IT \\
Carlos & Ventas \\
\bottomrule
\end{tabular}
\end{center}

\subsection{LEFT JOIN}
Devuelve todos los registros de la tabla izquierda y los coincidentes de la derecha. Ejemplo:
\begin{lstlisting}[language=SQL]
SELECT e.nombre, d.nombre_departamento
FROM empleados e
LEFT JOIN departamentos d ON e.id_departamento = d.id;
\end{lstlisting}

\begin{center}
\begin{tabular}{cc}
\toprule
nombre & nombre\_departamento \\
\midrule
Juan & Ventas \\
María & IT \\
Carlos & Ventas \\
Pedro & NULL \\
\bottomrule
\end{tabular}
\end{center}

\subsection{RIGHT JOIN}
Devuelve todos los registros de la tabla derecha y los coincidentes de la izquierda. Ejemplo:
\begin{lstlisting}[language=SQL]
SELECT e.nombre, d.nombre_departamento
FROM empleados e
RIGHT JOIN departamentos d ON e.id_departamento = d.id;
\end{lstlisting}

\begin{center}
\begin{tabular}{cc}
\toprule
nombre & nombre\_departamento \\
\midrule
Juan & Ventas \\
María & IT \\
Carlos & Ventas \\
NULL & Marketing \\
\bottomrule
\end{tabular}
\end{center}

\subsection{FULL JOIN}
Devuelve todos los registros cuando hay coincidencia en cualquiera de las tablas. Ejemplo:
\begin{lstlisting}[language=SQL]
SELECT e.nombre, d.nombre_departamento
FROM empleados e
FULL JOIN departamentos d ON e.id_departamento = d.id;
\end{lstlisting}

\begin{center}
\begin{tabular}{cc}
\toprule
nombre & nombre\_departamento \\
\midrule
Juan & Ventas \\
María & IT \\
Carlos & Ventas \\
Pedro & NULL \\
NULL & Marketing \\
\bottomrule
\end{tabular}
\end{center}

\end{document}
