\documentclass[12pt]{article}

\usepackage[utf8]{inputenc}
\usepackage[T1]{fontenc}
\usepackage{lmodern}
\usepackage[spanish]{babel}
\usepackage{booktabs}
\usepackage{amsmath}
\usepackage{forest}
\usepackage{float}
\usepackage{listings}
\usepackage{xcolor}
\usepackage{tikz}
\usepackage{tabularx}
\usepackage{enumitem}
\usetikzlibrary{shapes,arrows,positioning,calc}

\definecolor{codegreen}{rgb}{0,0.6,0}
\definecolor{codegray}{rgb}{0.5,0.5,0.5}
\definecolor{codepurple}{rgb}{0.58,0,0.82}
\definecolor{backcolour}{rgb}{0.95,0.95,0.92}

\lstdefinestyle{mystyle}{
    backgroundcolor=\color{backcolour},   
    commentstyle=\color{codegreen},
    keywordstyle=\color{magenta},
    numberstyle=\tiny\color{codegray},
    stringstyle=\color{codepurple},
    basicstyle=\ttfamily\footnotesize,
    breakatwhitespace=false,         
    breaklines=true,                 
    captionpos=b,                    
    keepspaces=true,                 
    numbers=left,                    
    numbersep=5pt,                  
    showspaces=false,                
    showstringspaces=false,
    showtabs=false,                  
    tabsize=2
}

\lstset{style=mystyle}

\sloppy
\setlength{\parindent}{0pt}

\begin{document}

\begin{center}
  {\LARGE \textbf{Guía de Preguntas - Gestión de Datos}}\\[0.5em]
  {Gestión y Arquitectura de Datos, Universidad de San Andrés}
\end{center}

Si encuentran algún error en el documento o hay alguna duda, mandenmé un mail a rodriguezf@udesa.edu.ar y lo revisamos.

\section*{Preguntas}

\begin{enumerate}[label=\arabic*.]

\item En el contexto de calidad de datos, ¿cuál de las siguientes afirmaciones sobre la dimensión de ``Consistencia'' es correcta?
\begin{enumerate}
    \item Significa que los datos son coherentes a través de diferentes sistemas y representaciones
    \item Se refiere únicamente a que los datos estén completos
    \item Implica que los datos sean coherentes solo dentro de una misma base de datos
    \item Solo se aplica a datos numéricos
\end{enumerate}

\item Una empresa implementa una arquitectura de datos moderna. ¿Cuál de las siguientes combinaciones de componentes sería la más apropiada para procesamiento en tiempo real y batch?
\begin{enumerate}
    \item Data Lake + Data Warehouse
    \item Solo Data Warehouse
    \item Lambda Architecture + Stream Processing
    \item Data Mart + Batch Processing
\end{enumerate}

\item En el rol de Data Steward, ¿cuál de las siguientes NO es una responsabilidad principal?
\begin{enumerate}
    \item Mantener los metadatos actualizados
    \item Asegurar la calidad de los datos
    \item Verificar el cumplimiento de políticas de datos
    \item Definir la estrategia general de datos de la empresa
\end{enumerate}

\item En el ciclo de vida de los datos, ¿cuál es el objetivo principal de la fase de archivado y eliminación?
\begin{enumerate}
    \item Garantizar que los datos se conserven solo el tiempo necesario y se eliminen de forma segura
    \item Mejorar la visualización de los datos
    \item Aumentar la velocidad de procesamiento en tiempo real
    \item Automatizar la captura de datos desde sensores
\end{enumerate}

\item En el ciclo de vida de los datos, ¿qué fase debería incluir necesariamente validación de calidad?
\begin{enumerate}
    \item Solo en la fase de Archivado
    \item En la fase de Procesamiento (ETL)
    \item Solo en la fase de Análisis
    \item Solo en la fase de Creación
\end{enumerate}

\item Para implementar una política efectiva de gobierno de datos, ¿qué combinación de elementos es más crítica?
\begin{enumerate}
    \item Solo tecnología y herramientas
    \item Únicamente automatización
    \item Roles definidos + Políticas claras + Procesos documentados
    \item Solo documentación técnica
\end{enumerate}

\item ¿Cuál de las siguientes dimensiones de calidad de datos se enfoca en la frecuencia de actualización de la información?
\begin{enumerate}
    \item Precisión
    \item Actualidad
    \item Unicidad
    \item Accesibilidad
\end{enumerate}

\item En el contexto de arquitectura de datos, ¿qué tipo de procesamiento sería más adecuado para analizar datos históricos de ventas mensuales?
\begin{enumerate}
    \item Procesamiento en streaming
    \item Procesamiento por lotes (Batch)
    \item Procesamiento en tiempo real
    \item Procesamiento híbrido
\end{enumerate}

\item ¿Cuál de las siguientes métricas de calidad de datos mide la frecuencia de duplicados en un conjunto de datos?
\begin{enumerate}
    \item Tasa de error
    \item Tasa de duplicación
    \item Tasa de actualización
    \item Tasa de completitud
\end{enumerate}

\item En el ciclo de vida de los datos, ¿qué fase incluye principalmente la transformación y limpieza de datos?
\begin{enumerate}
    \item Creación y Captura
    \item Almacenamiento
    \item Procesamiento
    \item Análisis
\end{enumerate}

\item ¿Qué rol en el gobierno de datos es responsable de mantener los datos actualizados y consistentes a nivel operativo?
\item ¿Cuáles son las tres dimensiones de calidad de datos más críticas para garantizar la confiabilidad de un sistema de información?
\item ¿En qué se diferencia un Data Lake de un Data Warehouse en términos de almacenamiento y uso?
\item ¿Por qué es importante implementar políticas de retención de datos en una organización?
\item ¿Cuál es la diferencia entre procesamiento por lotes y procesamiento en streaming, y cuándo se usa cada uno?

\end{enumerate}

\newpage
\section*{Respuestas}

\begin{enumerate}[label=\arabic*.]

\item En el contexto de calidad de datos, ¿cuál de las siguientes afirmaciones sobre la dimensión de ``Consistencia'' es correcta?
\begin{enumerate}
    \item \textbf{Significa que los datos son coherentes a través de diferentes sistemas y representaciones}
    \item Se refiere únicamente a que los datos estén completos
    \item Implica que los datos sean coherentes solo dentro de una misma base de datos
    \item Solo se aplica a datos numéricos
\end{enumerate}
La consistencia NO es completitud (b) ni se limita a una base (c). Requiere coherencia entre múltiples sistemas y representaciones, no solo datos numéricos (d).

\item Una empresa implementa una arquitectura de datos moderna. ¿Cuál de las siguientes combinaciones de componentes sería la más apropiada para procesamiento en tiempo real y batch?
\begin{enumerate}
    \item Data Lake + Data Warehouse
    \item Solo Data Warehouse
    \item \textbf{Lambda Architecture + Stream Processing}
    \item Data Mart + Batch Processing
\end{enumerate}
Data Lake/Warehouse (a) o solo Warehouse (b) NO manejan tiempo real eficientemente. Lambda Architecture sí combina ambos procesamientos, no solo batch (d).

\item En el rol de Data Steward, ¿cuál de las siguientes NO es una responsabilidad principal?
\begin{enumerate}
    \item Mantener los metadatos actualizados
    \item Asegurar la calidad de los datos
    \item Verificar el cumplimiento de políticas de datos
    \item \textbf{Definir la estrategia general de datos de la empresa}
\end{enumerate}
Data Steward SÍ hace metadatos (a), calidad (b) y cumplimiento (c). Estrategia general (d) es responsabilidad del CDO, no del Steward.

\item En el ciclo de vida de los datos, ¿cuál es el objetivo principal de la fase de archivado y eliminación?
\begin{enumerate}
    \item \textbf{Garantizar que los datos se conserven solo el tiempo necesario y se eliminen de forma segura}
    \item Mejorar la visualizacion de los datos
    \item Aumentar la velocidad de procesamiento en tiempo real
    \item Automatizar la captura de datos desde sensores
\end{enumerate}
El objetivo de archivado y eliminacion es asegurarse de que los datos no se guarden mas tiempo del necesario y que se borren de manera segura. No tiene que ver con visualizacion, velocidad ni captura.

\item En el ciclo de vida de los datos, ¿qué fase debería incluir necesariamente validación de calidad?
\begin{enumerate}
    \item Solo en la fase de Archivado
    \item \textbf{En la fase de Procesamiento (ETL)}
    \item Solo en la fase de Análisis
    \item Solo en la fase de Creación
\end{enumerate}
Archivado (a), Análisis (c) y Creación (d) NO transforman datos. ETL es donde se valida y limpia antes del uso final.

\item Para implementar una política efectiva de gobierno de datos, ¿qué combinación de elementos es más crítica?
\begin{enumerate}
    \item Solo tecnología y herramientas
    \item Únicamente automatización
    \item \textbf{Roles definidos + Políticas claras + Procesos documentados}
    \item Solo documentación técnica
\end{enumerate}
Solo tecnología (a), automatización (b) o documentación (d) NO bastan. Gobierno necesita roles, políticas y procesos integrados para funcionar.

\item ¿Cuál de las siguientes dimensiones de calidad de datos se enfoca en la frecuencia de actualización de la información?
\begin{enumerate}
    \item Precisión
    \item \textbf{Actualidad}
    \item Unicidad
    \item Accesibilidad
\end{enumerate}
La actualidad se refiere a qué tan recientes son los datos. Precisión (a) es exactitud, unicidad (c) es ausencia de duplicados, accesibilidad (d) es facilidad de acceso.

\item En el contexto de arquitectura de datos, ¿qué tipo de procesamiento sería más adecuado para analizar datos históricos de ventas mensuales?
\begin{enumerate}
    \item Procesamiento en streaming
    \item \textbf{Procesamiento por lotes (Batch)}
    \item Procesamiento en tiempo real
    \item Procesamiento híbrido
\end{enumerate}
Datos históricos mensuales no requieren tiempo real. Batch es ideal para grandes volúmenes sin urgencia inmediata. Streaming (a) y tiempo real (c) son para datos inmediatos.

\item ¿Cuál de las siguientes métricas de calidad de datos mide la frecuencia de duplicados en un conjunto de datos?
\begin{enumerate}
    \item Tasa de error
    \item \textbf{Tasa de duplicación}
    \item Tasa de actualización
    \item Tasa de completitud
\end{enumerate}
La tasa de duplicación mide específicamente la frecuencia de registros duplicados. Tasa de error (a) mide errores generales, actualización (c) mide frecuencia de cambios, completitud (d) no es una métrica estándar.

\item En el ciclo de vida de los datos, ¿qué fase incluye principalmente la transformación y limpieza de datos?
\begin{enumerate}
    \item Creación y Captura
    \item Almacenamiento
    \item \textbf{Procesamiento}
    \item Análisis
\end{enumerate}
El procesamiento incluye ETL, limpieza y validación. Creación (a) es captura inicial, almacenamiento (b) es guardado, análisis (d) es uso final.

\item ¿Qué rol en el gobierno de datos es responsable de mantener los datos actualizados y consistentes a nivel operativo?

El Data Steward es el responsable operativo que mantiene los datos actualizados y consistentes, a diferencia del CDO que define estrategia o el Data Owner que tiene responsabilidad general.

\item ¿Cuáles son las tres dimensiones de calidad de datos más críticas para garantizar la confiabilidad de un sistema de información?

Precisión asegura exactitud, completitud evita datos faltantes, y consistencia garantiza coherencia entre sistemas.

\item ¿En qué se diferencia un Data Lake de un Data Warehouse en términos de almacenamiento y uso?

Data Lake almacena datos en formato original (estructurados y no estructurados) para análisis exploratorio. Data Warehouse almacena datos estructurados y procesados para análisis histórico y reportes.

\item ¿Por qué es importante implementar políticas de retención de datos en una organización?

Evita almacenar datos innecesarios, cumple leyes de privacidad y reduce riesgos de seguridad.

\item ¿Cuál es la diferencia entre procesamiento por lotes y procesamiento en streaming, y cuándo se usa cada uno?

Batch procesa grandes volúmenes sin urgencia (datos históricos). Streaming procesa datos inmediatamente (sensores, transacciones). Batch es más eficiente para análisis, streaming para tiempo real.

\end{enumerate}

\end{document}
