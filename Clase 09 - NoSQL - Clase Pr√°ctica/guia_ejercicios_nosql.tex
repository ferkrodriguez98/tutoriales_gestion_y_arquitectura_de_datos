\documentclass[12pt]{article}

\usepackage[utf8]{inputenc}
\usepackage[T1]{fontenc}
\usepackage{lmodern}
\usepackage[spanish]{babel}
\usepackage{booktabs}
\usepackage{amsmath}
\usepackage{forest}
\usepackage{float}
\usepackage{listings}
\usepackage{xcolor}
\usepackage{tikz}

\definecolor{codegreen}{rgb}{0,0.6,0}
\definecolor{codegray}{rgb}{0.5,0.5,0.5}
\definecolor{codepurple}{rgb}{0.58,0,0.82}
\definecolor{backcolour}{rgb}{0.95,0.95,0.92}

\lstdefinestyle{mystyle}{
    backgroundcolor=\color{backcolour},   
    commentstyle=\color{codegreen},
    keywordstyle=\color{magenta},
    numberstyle=\tiny\color{codegray},
    stringstyle=\color{codepurple},
    basicstyle=\ttfamily\footnotesize,
    breakatwhitespace=false,         
    breaklines=true,                 
    captionpos=b,                    
    keepspaces=true,                 
    numbers=left,                    
    numbersep=5pt,                  
    showspaces=false,                
    showstringspaces=false,
    showtabs=false,                  
    tabsize=2
}

\lstset{style=mystyle}

\sloppy
\setlength{\parindent}{0pt}

\begin{document}

\begin{center}
  {\LARGE \textbf{Guía de Ejercicios \\ Selección de Bases de Datos NoSQL}}\\[0.5em]
  {Gestión y Arquitectura de Datos, Universidad de San Andrés}
\end{center}

Si encuentran algún error en el documento o hay alguna duda, mandenmé un mail a rodriguezf@udesa.edu.ar y lo revisamos.

\section{Ejercicios}

A continuación se presentan 10 casos de uso reales donde se debe decidir qué tipo(s) de base(s) de datos NoSQL utilizar. Para cada caso:
\begin{itemize}
    \item Analice los requerimientos del sistema
    \item Identifique las características clave que influyen en la elección
    \item Seleccione la(s) tecnología(s) más apropiada(s)
    \item Justifique su elección
\end{itemize}

Intenten resolver los casos por su cuenta. Las respuestas se encuentran al final del documento.

\subsection{Caso 1: E-commerce Básico}
Una pequeña tienda online está comenzando su negocio. Necesitan almacenar un catálogo de productos con descripciones y categorías y tener carritos de compra temporales. Tambien necesitan saber el historial de pedidos por usuario. La tienda espera tener unos 1,000 productos y 5,000 usuarios en su primer año.

\subsection{Caso 2: Red Social de Fotografías}
Una startup está desarrollando una red social centrada en compartir fotos. Necesitan almacenar fotos y sus metadatos, manejar relaciones de ``seguir'' entre usuarios, gestionar likes y comentarios y tener un feed personalizado para cada usuario. La red social espera tener unos 10,000 usuarios y 100,000 fotos en su primer año.

\subsection{Caso 3: Sistema de Monitoreo Industrial}
Una fábrica necesita monitorear sus máquinas en tiempo real. Los sensores envían datos cada segundo y se necesita almacenar históricos por 6 meses. La fábrica tiene 100 máquinas, cada una con 5 sensores diferentes.

\subsection{Caso 4: Plataforma de Streaming de Videos}
Un servicio de streaming necesita un catálogo de videos con metadatos, un sistema de recomendaciones, un historial de visualización por usuario, preferencias y configuraciones de usuario y caché de videos populares. El servicio espera manejar millones de usuarios concurrentes.

\subsection{Caso 5: Sistema de Reservas de Hotel}
Una cadena hotelera internacional necesita un sistema para gestionar disponibilidad en tiempo real, manejar reservas y cancelaciones, perfiles de clientes con preferencias, historial de estadías, sistema de fidelización y reportes de ocupación. La cadena tiene 50 hoteles en 20 países diferentes.

\subsection{Caso 6: Sistema de Logística}
Una empresa de logística necesita un sistema para rastrear ubicación de paquetes en tiempo real, optimizar rutas de entrega, gestionar flota de vehículos, historial de entregas y análisis de eficiencia. La empresa maneja 1,000 entregas diarias con 100 vehículos en 5 ciudades.

\subsection{Caso 7: Plataforma de Juegos Online}
Un estudio de videojuegos necesita una plataforma para perfiles de jugadores y estadísticas, rankings y logros, inventarios de items in-game, chat en tiempo real, matchmaking y transacciones de items entre jugadores. Esperan picos de 100,000 jugadores simultáneos.

\subsection{Caso 8: Sistema de Salud}
Un hospital necesita modernizar su sistema para historias clínicas electrónicas, gestión de citas y turnos, resultados de estudios (incluyendo imágenes), prescripciones médicas, seguimiento de tratamientos e integración con otros centros médicos. El hospital atiende 2,000 pacientes diarios y necesita acceso 24/7 a los datos.

\subsection{Caso 9: Plataforma IoT Smart City}
Una ciudad inteligente necesita gestionar datos de sensores de tráfico, monitoreo de calidad del aire, control de semáforos, sistema de parking inteligente, consumo energético y análisis predictivo para servicios públicos. La ciudad tiene 1 millón de habitantes y miles de sensores distribuidos.

\subsection{Caso 10: Sistema Financiero}
Un banco digital necesita manejar transacciones en tiempo real, detección de fraude, historial de operaciones, perfiles de clientes, análisis de riesgo crediticio, reportes regulatorios e integración con otros bancos. El banco tiene 500,000 clientes y procesa millones de transacciones diarias.

\newpage
\section{Anexo: Respuestas}

\subsection{Respuesta Caso 1: E-commerce Básico}

\textbf{MongoDB (Base de Datos Documental)} como base principal:
\begin{itemize}
    \item Ideal para el catálogo de productos por su esquema flexible
    \item Permite almacenar productos con diferentes atributos y categorías
    \item Bueno para el historial de pedidos con documentos anidados
\end{itemize}

\textbf{Redis (Base de Datos Clave-Valor)} para:
\begin{itemize}
    \item Carritos de compra temporales
    \item Caché de productos populares
    \item Sesiones de usuario
\end{itemize}

Justificación: Para un e-commerce pequeño, esta combinación ofrece:
\begin{itemize}
    \item Flexibilidad para el crecimiento del catálogo
    \item Alta velocidad para operaciones frecuentes
    \item Simplicidad de implementación
    \item Buena relación costo-beneficio
\end{itemize}

\subsection{Respuesta Caso 2: Red Social de Fotografías}

\textbf{MongoDB (Base de Datos Documental)} para:
\begin{itemize}
    \item Perfiles de usuario
    \item Metadatos de fotos
    \item Comentarios y likes
\end{itemize}

\textbf{Neo4j (Base de Datos de Grafos)} para:
\begin{itemize}
    \item Relaciones de "seguir"
    \item Recomendaciones de usuarios
    \item Análisis de redes sociales
\end{itemize}

\textbf{Redis (Base de Datos Clave-Valor)} para:
\begin{itemize}
    \item Caché de feeds
    \item Contadores de likes
    \item Sesiones activas
\end{itemize}

Justificación: Esta combinación permite:
\begin{itemize}
    \item Escalabilidad para el crecimiento rápido
    \item Optimización de consultas sociales
    \item Alta velocidad en operaciones frecuentes
\end{itemize}

\subsection{Respuesta Caso 3: Sistema de Monitoreo Industrial}

\textbf{Cassandra (Base de Datos Columnar)} para:
\begin{itemize}
    \item Almacenamiento de datos de sensores
    \item Series temporales
    \item Alta disponibilidad
\end{itemize}

\textbf{Redis (Base de Datos Clave-Valor)} para:
\begin{itemize}
    \item Análisis en tiempo real
    \item Caché de últimas lecturas
    \item Alertas temporales
\end{itemize}

Justificación:
\begin{itemize}
    \item Cassandra es excelente para series temporales y escrituras intensivas
    \item Permite escalabilidad horizontal para históricos
    \item Redis complementa con análisis en tiempo real
\end{itemize}

\subsection{Respuesta Caso 4: Plataforma de Streaming}

\textbf{MongoDB (Base de Datos Documental)} para:
\begin{itemize}
    \item Catálogo de videos y metadatos
    \item Perfiles de usuario
    \item Historial de visualización
\end{itemize}

\textbf{Redis (Base de Datos Clave-Valor)} para:
\begin{itemize}
    \item Caché de videos populares
    \item Sesiones de usuario
    \item Estado de reproducción
\end{itemize}

\textbf{Neo4j (Base de Datos de Grafos)} para:
\begin{itemize}
    \item Sistema de recomendaciones
    \item Análisis de preferencias
\end{itemize}

Justificación: Esta arquitectura permite:
\begin{itemize}
    \item Alta disponibilidad
    \item Escalabilidad horizontal
    \item Recomendaciones personalizadas eficientes
\end{itemize}

\subsection{Respuesta Caso 5: Sistema de Reservas de Hotel}

\textbf{MongoDB (Base de Datos Documental)} para:
\begin{itemize}
    \item Perfiles de clientes
    \item Detalles de hoteles
    \item Historial de estadías
\end{itemize}

\textbf{Redis (Base de Datos Clave-Valor)} para:
\begin{itemize}
    \item Disponibilidad en tiempo real
    \item Bloqueo temporal de habitaciones
    \item Caché de búsquedas frecuentes
\end{itemize}

Justificación:
\begin{itemize}
    \item Permite consistencia eventual entre regiones
    \item Alta disponibilidad para reservas
    \item Buen manejo de datos distribuidos
\end{itemize}

\subsection{Respuesta Caso 6: Sistema de Logística}

\textbf{MongoDB (Base de Datos Documental)} para:
\begin{itemize}
    \item Información de paquetes
    \item Detalles de vehículos
    \item Historial de entregas
\end{itemize}

\textbf{Redis (Base de Datos Clave-Valor)} para:
\begin{itemize}
    \item Ubicación en tiempo real
    \item Estado de entregas
    \item Caché de rutas
\end{itemize}

\textbf{Neo4j (Base de Datos de Grafos)} para:
\begin{itemize}
    \item Optimización de rutas
    \item Análisis de red logística
\end{itemize}

Justificación:
\begin{itemize}
    \item Permite tracking en tiempo real
    \item Optimización eficiente de rutas
    \item Análisis histórico de eficiencia
\end{itemize}

\subsection{Respuesta Caso 7: Plataforma de Juegos Online}

\textbf{MongoDB (Base de Datos Documental)} para:
\begin{itemize}
    \item Perfiles de jugadores
    \item Inventarios
    \item Estadísticas
\end{itemize}

\textbf{Redis (Base de Datos Clave-Valor)} para:
\begin{itemize}
    \item Chat en tiempo real
    \item Estado de partidas
    \item Rankings temporales
    \item Matchmaking
\end{itemize}

\textbf{Neo4j (Base de Datos de Grafos)} para:
\begin{itemize}
    \item Relaciones entre jugadores
    \item Análisis de comportamiento
\end{itemize}

Justificación:
\begin{itemize}
    \item Alta velocidad para operaciones en tiempo real
    \item Escalabilidad para picos de usuarios
    \item Análisis social para matchmaking
\end{itemize}

\subsection{Respuesta Caso 8: Sistema de Salud}

\textbf{MongoDB (Base de Datos Documental)} para:
\begin{itemize}
    \item Historias clínicas
    \item Resultados de estudios
    \item Prescripciones
\end{itemize}

\textbf{Redis (Base de Datos Clave-Valor)} para:
\begin{itemize}
    \item Gestión de turnos
    \item Caché de datos frecuentes
    \item Estado de pacientes
\end{itemize}

\textbf{Neo4j (Base de Datos de Grafos)} para:
\begin{itemize}
    \item Relaciones entre diagnósticos
    \item Análisis de tratamientos
    \item Seguimiento de derivaciones
\end{itemize}

Justificación:
\begin{itemize}
    \item Alta disponibilidad 24/7
    \item Seguridad y consistencia de datos
    \item Análisis complejo de historias clínicas
\end{itemize}

\subsection{Respuesta Caso 9: Plataforma IoT Smart City}

\textbf{Cassandra (Base de Datos Columnar)} para:
\begin{itemize}
    \item Datos de sensores
    \item Métricas ambientales
    \item Históricos de tráfico
\end{itemize}

\textbf{Redis (Base de Datos Clave-Valor)} para:
\begin{itemize}
    \item Estado actual de sensores
    \item Control de semáforos
    \item Alertas en tiempo real
\end{itemize}

\textbf{Neo4j (Base de Datos de Grafos)} para:
\begin{itemize}
    \item Análisis de patrones de tráfico
    \item Optimización de rutas
    \item Relaciones entre servicios
\end{itemize}

Justificación:
\begin{itemize}
    \item Manejo eficiente de grandes volúmenes de datos
    \item Alta velocidad para decisiones en tiempo real
    \item Capacidad de análisis predictivo
\end{itemize}

\subsection{Respuesta Caso 10: Sistema Financiero}

\textbf{MongoDB (Base de Datos Documental)} para:
\begin{itemize}
    \item Perfiles de clientes
    \item Historial de transacciones
    \item Documentación
\end{itemize}

\textbf{Redis (Base de Datos Clave-Valor)} para:
\begin{itemize}
    \item Procesamiento de transacciones
    \item Detección de fraude en tiempo real
    \item Caché de datos frecuentes
\end{itemize}

\textbf{Cassandra (Base de Datos Columnar)} para:
\begin{itemize}
    \item Logs de auditoría
    \item Históricos de operaciones
    \item Reportes regulatorios
\end{itemize}

Justificación:
\begin{itemize}
    \item Garantiza consistencia en transacciones
    \item Alta disponibilidad y tolerancia a fallos
    \item Cumplimiento regulatorio
    \item Escalabilidad para millones de operaciones
\end{itemize}

\end{document} 