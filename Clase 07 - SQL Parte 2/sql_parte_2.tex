\documentclass[12pt]{article}

\usepackage[utf8]{inputenc}
\usepackage[T1]{fontenc}
\usepackage{lmodern}
\usepackage[spanish]{babel}
\usepackage{booktabs}
\usepackage{amsmath}
\usepackage{forest}
\usepackage{float}
\usepackage{listings}
\usepackage{xcolor}
\usepackage{tikz}

\definecolor{codegreen}{rgb}{0,0.6,0}
\definecolor{codegray}{rgb}{0.5,0.5,0.5}
\definecolor{codepurple}{rgb}{0.58,0,0.82}
\definecolor{backcolour}{rgb}{0.95,0.95,0.92}

\lstdefinestyle{mystyle}{
    backgroundcolor=\color{backcolour},   
    commentstyle=\color{codegreen},
    keywordstyle=\color{magenta},
    numberstyle=\tiny\color{codegray},
    stringstyle=\color{codepurple},
    basicstyle=\ttfamily\footnotesize,
    breakatwhitespace=false,         
    breaklines=true,                 
    captionpos=b,                    
    keepspaces=true,                 
    numbers=left,                    
    numbersep=5pt,                  
    showspaces=false,                
    showstringspaces=false,
    showtabs=false,                  
    tabsize=2
}

\lstset{style=mystyle}

\sloppy
\setlength{\parindent}{0pt}

\begin{document}

\begin{center}
  {\LARGE \textbf{SQL Avanzado}}\\[0.5em]
  {Gestión y Arquitectura de Datos, Universidad de San Andrés}
\end{center}

Si encuentran algún error en el documento o hay alguna duda, mandenmé un mail a rodriguezf@udesa.edu.ar y lo revisamos.

\section{Cláusula HAVING}
HAVING se usa para filtrar resultados de grupos, a diferencia de WHERE que filtra registros individuales.

\begin{lstlisting}[language=SQL]
SELECT departamento, COUNT(*) as cantidad_empleados
FROM empleados
GROUP BY departamento
HAVING COUNT(*) > 2;
\end{lstlisting}

\begin{center}
\begin{tabular}{cc}
\toprule
departamento & cantidad\_empleados \\
\midrule
Ventas & 3 \\
IT & 4 \\
\bottomrule
\end{tabular}
\end{center}

\section{Subconsultas}
Las subconsultas son consultas anidadas dentro de otras consultas.

\subsection{Subconsulta en WHERE}
\begin{lstlisting}[language=SQL]
SELECT nombre, salario
FROM empleados
WHERE salario > (
    SELECT AVG(salario)
    FROM empleados
);
\end{lstlisting}

\begin{center}
\begin{tabular}{cc}
\toprule
nombre & salario \\
\midrule
María & 60000 \\
Juan & 55000 \\
\bottomrule
\end{tabular}
\end{center}

\subsection{Subconsulta en FROM}
\begin{lstlisting}[language=SQL]
SELECT dept_info.departamento, 
       dept_info.promedio_salario
FROM (
    SELECT departamento, 
           AVG(salario) as promedio_salario
    FROM empleados
    GROUP BY departamento
) as dept_info
WHERE dept_info.promedio_salario > 50000;
\end{lstlisting}

\begin{center}
\begin{tabular}{cc}
\toprule
departamento & promedio\_salario \\
\midrule
IT & 55000 \\
Ventas & 52000 \\
\bottomrule
\end{tabular}
\end{center}

\subsection{Subconsulta con EXISTS}
\begin{lstlisting}[language=SQL]
SELECT nombre
FROM empleados e
WHERE EXISTS (
    SELECT 1 
    FROM ventas v
    WHERE v.id_empleado = e.id 
    AND v.monto > 10000
);
\end{lstlisting}

\begin{center}
\begin{tabular}{c}
\toprule
nombre \\
\midrule
Juan \\
María \\
\bottomrule
\end{tabular}
\end{center}

\section{Funciones de Ventana (Window Functions)}
Las funciones de ventana permiten realizar cálculos a través de un conjunto de filas relacionadas con la fila actual.

\subsection{RANK()}
La función RANK() asigna un número de rango a cada fila dentro de una partición, saltando números cuando hay empates. En este ejemplo, ordenamos por salario descendente y asignamos rangos, donde el salario más alto recibe el rango 1.

\begin{lstlisting}[language=SQL]
SELECT 
    nombre,
    salario,
    RANK() OVER (ORDER BY salario DESC) as ranking
FROM empleados;
\end{lstlisting}

\begin{center}
\begin{tabular}{ccc}
\toprule
nombre & salario & ranking \\
\midrule
María & 60000 & 1 \\
Juan & 55000 & 2 \\
Carlos & 55000 & 2 \\
Pedro & 45000 & 4 \\
\bottomrule
\end{tabular}
\end{center}

\subsection{PARTITION BY}
PARTITION BY divide el conjunto de resultados en grupos o particiones. En este caso, calculamos el promedio de salario para cada departamento por separado, mostrando el resultado en cada fila correspondiente a ese departamento.

\begin{lstlisting}[language=SQL]
SELECT 
    nombre,
    departamento,
    salario,
    AVG(salario) OVER (
        PARTITION BY departamento
    ) as promedio_dept
FROM empleados;
\end{lstlisting}

\begin{center}
\begin{tabular}{cccc}
\toprule
nombre & departamento & salario & promedio\_dept \\
\midrule
María & IT & 60000 & 55000 \\
Juan & IT & 50000 & 55000 \\
Carlos & Ventas & 55000 & 50000 \\
Pedro & Ventas & 45000 & 50000 \\
\bottomrule
\end{tabular}
\end{center}

\subsection{ROW\_NUMBER()}
% ROW\_NUMBER() 
ROW\_NUMBER() asigna un número secuencial único a cada fila dentro de una partición. A diferencia de RANK(), no hay empates - cada fila recibe un número único. En este ejemplo, numeramos los empleados por departamento según su salario.

\begin{lstlisting}[language=SQL]
SELECT 
    nombre,
    departamento,
    salario,
    ROW_NUMBER() OVER (
        PARTITION BY departamento 
        ORDER BY salario DESC
    ) as num_fila
FROM empleados;
\end{lstlisting}

\begin{center}
\begin{tabular}{cccc}
\toprule
nombre & departamento & salario & num\_fila \\
\midrule
María & IT & 60000 & 1 \\
Juan & IT & 50000 & 2 \\
Carlos & Ventas & 55000 & 1 \\
Pedro & Ventas & 45000 & 2 \\
\bottomrule
\end{tabular}
\end{center}

\subsection{LAG() y LEAD()}
LAG() y LEAD() permiten acceder a datos de filas anteriores o siguientes respectivamente. LAG() obtiene el valor de la fila anterior, mientras que LEAD() obtiene el valor de la fila siguiente. En este ejemplo, mostramos el salario anterior y siguiente para cada empleado.

\begin{lstlisting}[language=SQL]
SELECT 
    nombre,
    salario,
    LAG(salario) OVER (ORDER BY salario) as salario_anterior,
    LEAD(salario) OVER (ORDER BY salario) as salario_siguiente
FROM empleados;
\end{lstlisting}

\begin{center}
\begin{tabular}{cccc}
\toprule
nombre & salario & salario\_anterior & salario\_siguiente \\
\midrule
Pedro & 45000 & NULL & 50000 \\
Juan & 50000 & 45000 & 55000 \\
Carlos & 55000 & 50000 & 60000 \\
María & 60000 & 55000 & NULL \\
\bottomrule
\end{tabular}
\end{center}

\section{Common Table Expressions (CTE)}
Las CTEs proporcionan una forma más legible de escribir subconsultas complejas.

\begin{lstlisting}[language=SQL]
WITH salarios_dept AS (
    SELECT 
        departamento,
        AVG(salario) as promedio_salario
    FROM empleados
    GROUP BY departamento
)
SELECT e.nombre,
       e.salario,
       s.promedio_salario
FROM empleados e
JOIN salarios_dept s 
    ON e.departamento = s.departamento
WHERE e.salario > s.promedio_salario;
\end{lstlisting}

\begin{center}
\begin{tabular}{ccc}
\toprule
nombre & salario & promedio\_salario \\
\midrule
María & 60000 & 55000 \\
Carlos & 55000 & 50000 \\
\bottomrule
\end{tabular}
\end{center}

\section{Optimización de Consultas}
\begin{itemize}
    \item Usar índices apropiadamente
    \item Evitar SELECT *
    \item Limitar resultados con WHERE antes de GROUP BY
    \item Usar EXPLAIN para analizar el plan de ejecución
    \item Preferir JOINs sobre subconsultas cuando sea posible
    \item Usar CTEs para mejorar la legibilidad
    \item Considerar el uso de índices compuestos para consultas frecuentes
\end{itemize}

\end{document}
