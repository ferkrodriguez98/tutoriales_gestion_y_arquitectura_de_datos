\documentclass[12pt]{article}

\usepackage[utf8]{inputenc}
\usepackage[T1]{fontenc}
\usepackage{lmodern}
\usepackage[spanish]{babel}
\usepackage{booktabs}
\usepackage{amsmath}
\usepackage{forest}
\usepackage{float}
\usepackage{listings}
\usepackage{xcolor}
\usepackage{tikz}
\usetikzlibrary{positioning,shapes.multipart,calc,arrows,shapes.geometric}

% Definicion de estilos para entidades
\tikzset{
    entity/.style={
        rectangle split,
        rectangle split parts=2,
        draw,
        align=center,
        minimum height=1cm
    },
    arrow/.style={
        ->,
        >=latex
    }
}

\sloppy
\setlength{\parindent}{0pt}

\begin{document}

\begin{center}
  {\LARGE \textbf{Guía de Ejercicios \\ Diagramas Entidad-Relación}}\\[0.5em]
  {Gestión y Arquitectura de Datos, Universidad de San Andrés}
\end{center}

Si encuentran algún error en el documento o hay alguna duda, mandenmé un mail a rodriguezf@udesa.edu.ar y lo revisamos.

\vspace{1em}

Para cada uno de los siguientes escenarios, diseñe un Diagrama Entidad-Relación (DER) que represente adecuadamente la situación planteada. Identifique entidades, atributos, relaciones y cardinalidades. Las soluciones se encuentran al final del documento.

\section{Ejercicios}

\subsection{Sistema de Biblioteca Personal}
Juan quiere organizar su biblioteca personal. Cada libro tiene un título, autor principal, año de publicación, editorial y género. Juan organiza sus libros en estantes, donde cada estante tiene un número y una descripción (por ejemplo, ``Ficción'', ``Historia''). Un libro solo puede estar en un estante a la vez, y cada estante puede contener múltiples libros.

\subsection{Gestión de Recetas}
Una chef necesita digitalizar su recetario. Cada receta tiene un nombre único, tiempo de preparación, nivel de dificultad y porciones. Las recetas utilizan ingredientes, y de cada ingrediente necesita registrar nombre, unidad de medida (gramos, unidades, etc.) y cantidad necesaria. Un mismo ingrediente puede aparecer en múltiples recetas, y cada receta requiere al menos un ingrediente.

\subsection{Club Deportivo}
Un club deportivo necesita gestionar sus clases. El club ofrece diferentes deportes (tenis, natación, básquet, etc.), cada uno con un nombre y descripción. Los profesores del club pueden enseñar uno o más deportes, y de cada profesor se registra su DNI, nombre, teléfono y fecha de ingreso. Los socios se inscriben en las clases, y de cada socio se guarda DNI, nombre, dirección y teléfono. Un profesor puede dar clases a múltiples socios, y un socio puede tomar clases con diferentes profesores.

\subsection{Sistema de Pedidos Online}
Una tienda online necesita gestionar sus pedidos. Los clientes realizan pedidos de productos. De cada cliente se registra su email (que lo identifica), nombre, dirección de envío y teléfono. Los productos tienen un código único, nombre, descripción, precio y stock disponible. Cada pedido tiene un número único, fecha, estado (pendiente, enviado, entregado) y puede incluir varios productos en diferentes cantidades. También se registra el precio unitario al momento de la compra.

\subsection{Sistema de Universidad}
Una universidad necesita gestionar su sistema académico. Los profesores (identificados por legajo) dictan materias y los alumnos (identificados por número de estudiante) se inscriben en ellas. De los profesores se registra nombre, departamento y título. De los alumnos se guarda nombre, carrera y año de ingreso. Las materias tienen un código único, nombre, departamento y créditos. Cada materia puede ser dictada por varios profesores en diferentes cuatrimestres, y un profesor puede dictar varias materias. Los alumnos reciben una nota final en cada materia que cursan.

\subsection{Gestión de Proyectos}
Una empresa necesita gestionar sus proyectos de software. Cada proyecto tiene empleados asignados con diferentes roles (desarrollador, tester, líder). De los empleados se registra legajo, nombre, email y fecha de ingreso. Los proyectos tienen un código, nombre, fecha de inicio y fecha estimada de fin. Cada proyecto se divide en tareas, donde cada tarea tiene un ID, descripción, estado y fecha límite. Las tareas pueden depender de otras tareas (una tarea no puede empezar hasta que sus dependencias estén completas).

\subsection{Sistema de Hospital}
Un hospital necesita gestionar sus consultas médicas. Los pacientes (identificados por DNI) pueden ser atendidos por médicos en diferentes especialidades. De los médicos se registra matrícula, nombre, especialidad y horarios de atención. Las consultas tienen fecha, hora, consultorio y diagnóstico. Los pacientes tienen un historial médico que incluye alergias y enfermedades crónicas. Además, en cada consulta se pueden recetar medicamentos, de los cuales se registra código, nombre, laboratorio y dosis recomendada.

\subsection{Plataforma de Streaming}
Una plataforma de streaming necesita modelar su sistema. Los usuarios pueden crear perfiles (máximo 4 por cuenta), y cada perfil puede tener su propia lista de contenido para ver más tarde. El contenido puede ser película o serie, donde las series tienen temporadas y episodios. De cada contenido se guarda ID, título, género, año, clasificación por edad y duración. Los usuarios pueden dejar reseñas (con puntuación y comentario) en cualquier contenido, pero solo una reseña por perfil por contenido.

\subsection{Sistema de Aerolínea}
Una aerolínea necesita gestionar sus vuelos y reservas. Los vuelos tienen un número único, aeropuerto de origen, aeropuerto de destino, fecha y hora de salida y llegada. Los aeropuertos tienen código IATA (único), nombre, ciudad y país. Los pasajeros realizan reservas para vuelos específicos, y de cada pasajero se guarda DNI, nombre, pasaporte y contacto de emergencia. Cada avión (identificado por matrícula) tiene un modelo, capacidad y fecha del último mantenimiento. Los vuelos son operados por tripulaciones, donde cada miembro (piloto, copiloto, auxiliares) tiene diferentes roles y certificaciones.

\newpage
\section{Anexo: Respuestas}

\subsection{Sistema de Biblioteca Personal}
\begin{center}
\begin{tikzpicture}[node distance=2cm]
    \node[entity] (libro) {Libro\nodepart{second}\textbf{id\_libro (PK)}\\ \underline{id\_estante (FK)} \\ titulo\\ autor\\ anio\\ editorial\\ genero};
    \node[entity] (estante) [right=of libro] {Estante\nodepart{second}\textbf{numero (PK)}\\ descripcion};
    
    \coordinate (fk_estante) at ($(libro.east)+(0,1)$);
    \coordinate (pk_estante) at ($(estante.west)+(0,0.4)$);
    \draw[-latex] (fk_estante) to[out=0,in=180] 
        node[pos=0.2,above] {1..N} 
        node[pos=0.8,above] {1..1} 
        (pk_estante);
\end{tikzpicture}
\end{center}

\subsection{Gestión de Recetas}
\begin{center}
\resizebox{1\textwidth}{!}{
\begin{tikzpicture}[node distance=2cm]
    \node[entity] (receta) {Receta\nodepart{second}\textbf{id\_receta (PK)}\\ nombre\\ tiempo\_prep\\ dificultad\\ porciones};
    \node[entity] (detalle) [right=of receta] {Detalle\_Receta\nodepart{second}\textbf{id\_detalle (PK)}\\ \underline{id\_receta (FK)}\\ \underline{id\_ingrediente (FK)}\\ cantidad\\ unidad\_medida};
    \node[entity] (ingrediente) [right=of detalle] {Ingrediente\nodepart{second}\textbf{id\_ingrediente (PK)}\\ nombre};
    
    \coordinate (fk_receta) at ($(detalle.west)+(0,0.1)$);
    \coordinate (pk_receta) at ($(receta.east)+(0,0.4)$);
    \draw[-latex] (fk_receta) to[out=180,in=0] 
        node[pos=0.2,above] {1..N} 
        node[pos=0.8,above] {1..1} 
        (pk_receta);
        
    \coordinate (fk_ingrediente) at ($(detalle.east)+(0,-0.5)$);
    \coordinate (pk_ingrediente) at ($(ingrediente.west)+(0,0.4)$);
    \draw[-latex] (fk_ingrediente) to[out=0,in=180] 
        node[pos=0.2,above] {N..1} 
        node[pos=0.8,above] {1..1} 
        (pk_ingrediente);
\end{tikzpicture}
}
\end{center}

\subsection{Club Deportivo}
\begin{center}
\resizebox{1\textwidth}{!}{
\begin{tikzpicture}[node distance=2cm]
    \node[entity] (profesor) {Profesor\nodepart{second}\textbf{dni (PK)}\\ nombre\\ telefono\\ fecha\_ingreso};
    \node[entity] (clase) [right=of profesor] {Clase\nodepart{second}\textbf{id\_clase (PK)}\\ \underline{dni\_profesor (FK)}\\ \underline{id\_deporte (FK)}\\ horario\\ dia};
    \node[entity] (deporte) [right=of clase] {Deporte\nodepart{second}\textbf{id\_deporte (PK)}\\ nombre\\ descripcion};
    \node[entity] (inscripcion) [below=of clase] {Inscripcion\nodepart{second}\textbf{id\_inscripcion (PK)}\\ \underline{id\_clase (FK)}\\ \underline{dni\_socio (FK)}\\ fecha};
    \node[entity] (socio) [right=of inscripcion] {Socio\nodepart{second}\textbf{dni (PK)}\\ nombre\\ direccion\\ telefono};
    
    % Conexiones profesor-clase
    \coordinate (fk_profesor) at ($(clase.west)+(0,0.1)$);
    \coordinate (pk_profesor) at ($(profesor.east)+(0,0.4)$);
    \draw[-latex] (fk_profesor) to[out=180,in=0] 
        node[pos=0.2,above] {1..N} 
        node[pos=0.8,above] {1..1} 
        (pk_profesor);
    
    % Conexiones clase-deporte    
    \coordinate (fk_deporte) at ($(clase.east)+(0,-0.5)$);
    \coordinate (pk_deporte) at ($(deporte.west)+(0,0.4)$);
    \draw[-latex] (fk_deporte) to[out=0,in=180] 
        node[pos=0.2,above] {N..1} 
        node[pos=0.8,above] {1..1} 
        (pk_deporte);
        
    % Conexiones clase-inscripcion
    \coordinate (fk_clase) at ($(inscripcion.north)+(0,0.1)$);
    \coordinate (pk_clase) at ($(clase.south)+(0,-0.1)$);
    \draw[-latex] (fk_clase) to[out=90,in=-90] 
        node[pos=0.2,right] {1..N} 
        node[pos=0.8,right] {1..1} 
        (pk_clase);
        
    % Conexiones inscripcion-socio
    \coordinate (fk_socio) at ($(inscripcion.east)+(0,-0.5)$);
    \coordinate (pk_socio) at ($(socio.west)+(0,0.4)$);
    \draw[-latex] (fk_socio) to[out=0,in=180] 
        node[pos=0.2,above] {N..1} 
        node[pos=0.8,above] {1..1} 
        (pk_socio);
\end{tikzpicture}
}
\end{center}

\subsection{Sistema de Pedidos Online}
\begin{center}
\resizebox{1\textwidth}{!}{
\begin{tikzpicture}[node distance=2cm]
    \node[entity] (cliente) {Cliente\nodepart{second}\textbf{email (PK)}\\ nombre\\ direccion\\ telefono};
    \node[entity] (pedido) [right=of cliente] {Pedido\nodepart{second}\textbf{numero (PK)}\\ \underline{email\_cliente (FK)}\\ fecha\\ estado};
    \node[entity] (detalle) [right=of pedido] {Detalle\_Pedido\nodepart{second}\textbf{id\_detalle (PK)}\\ \underline{numero\_pedido (FK)}\\ \underline{codigo\_producto (FK)}\\ cantidad\\ precio\_unitario};
    \node[entity] (producto) [right=of detalle] {Producto\nodepart{second}\textbf{codigo (PK)}\\ nombre\\ descripcion\\ precio\\ stock};
    
    % Conexiones cliente-pedido
    \coordinate (fk_cliente) at ($(pedido.west)+(0,0.1)$);
    \coordinate (pk_cliente) at ($(cliente.east)+(0,0.4)$);
    \draw[-latex] (fk_cliente) to[out=180,in=0] 
        node[pos=0.2,above] {1..N} 
        node[pos=0.8,above] {1..1} 
        (pk_cliente);
    
    % Conexiones pedido-detalle
    \coordinate (fk_pedido) at ($(detalle.west)+(0,0.1)$);
    \coordinate (pk_pedido) at ($(pedido.east)+(0,0.4)$);
    \draw[-latex] (fk_pedido) to[out=180,in=0] 
        node[pos=0.2,above] {1..N} 
        node[pos=0.8,above] {1..1} 
        (pk_pedido);
        
    % Conexiones detalle-producto
    \coordinate (fk_producto) at ($(detalle.east)+(0,-0.5)$);
    \coordinate (pk_producto) at ($(producto.west)+(0,0.4)$);
    \draw[-latex] (fk_producto) to[out=0,in=180] 
        node[pos=0.2,above] {N..1} 
        node[pos=0.8,above] {1..1} 
        (pk_producto);
\end{tikzpicture}
}
\end{center}

\subsection{Sistema de Universidad}
\begin{center}
\resizebox{1\textwidth}{!}{
\begin{tikzpicture}[node distance=2cm]
    \node[entity] (profesor) {Profesor\nodepart{second}\textbf{legajo (PK)}\\ nombre\\ departamento\\ titulo};
    \node[entity] (materia) [right=of profesor] {Materia\nodepart{second}\textbf{codigo (PK)}\\ nombre\\ departamento\\ creditos};
    \node[entity] (dictado) [below=of materia] {Dictado\nodepart{second}\textbf{id\_dictado (PK)}\\ \underline{legajo\_profesor (FK)}\\ \underline{codigo\_materia (FK)}\\ anio\\ cuatrimestre};
    \node[entity] (alumno) [right=of dictado] {Alumno\nodepart{second}\textbf{numero\_estudiante (PK)}\\ nombre\\ carrera\\ anio\_ingreso};
    \node[entity] (inscripcion) [below=of dictado] {Inscripcion\nodepart{second}\textbf{id\_inscripcion (PK)}\\ \underline{id\_dictado (FK)}\\ \underline{numero\_estudiante (FK)}\\ nota\_final};
    
    % Conexiones profesor-dictado
    \coordinate (fk_profesor) at ($(dictado.west)+(0,0.1)$);
    \coordinate (pk_profesor) at ($(profesor.east)+(0,0.4)$);
    \draw[-latex] (fk_profesor) to[out=180,in=0] 
        node[pos=0.2,left] {1..N} 
        node[pos=0.8,right] {1..1} 
        (pk_profesor);
    
    % Conexiones materia-dictado
    \coordinate (fk_materia) at ($(dictado.east)+(0,0.1)$);
    \coordinate (pk_materia) at ($(materia.east)+(0,-0.1)$);
    \draw[-latex] (fk_materia) to[out=0,in=0] 
        node[pos=0.2,right] {1..N} 
        node[pos=0.8,right] {1..1} 
        (pk_materia);
        
    % Conexiones dictado-inscripcion
    \coordinate (fk_dictado) at ($(inscripcion.west)+(0,0.1)$);
    \coordinate (pk_dictado) at ($(dictado.west)+(0,-0.1)$);
    \draw[-latex] (fk_dictado) to[out=180,in=180] 
        node[pos=0.2,left] {1..N} 
        node[pos=0.8,left] {1..1} 
        (pk_dictado);
        
    % Conexiones alumno-inscripcion
    \coordinate (fk_alumno) at ($(inscripcion.east)+(0,-0.5)$);
    \coordinate (pk_alumno) at ($(alumno.west)+(0,0.4)$);
    \draw[-latex] (fk_alumno) to[out=0,in=180] 
        node[pos=0.2,right] {N..1} 
        node[pos=0.8,left] {1..1} 
        (pk_alumno);
\end{tikzpicture}
}
\end{center}

\subsection{Gestión de Proyectos}
\begin{center}
\resizebox{1\textwidth}{!}{
\begin{tikzpicture}[node distance=2cm]
    \node[entity] (empleado) {Empleado\nodepart{second}\textbf{legajo (PK)}\\ nombre\\ email\\ fecha\_ingreso};
    \node[entity] (proyecto) [right=of empleado] {Proyecto\nodepart{second}\textbf{codigo (PK)}\\ nombre\\ fecha\_inicio\\ fecha\_fin\_estimada};
    \node[entity] (asignacion) [below=of proyecto] {Asignacion\nodepart{second}\textbf{id\_asignacion (PK)}\\ \underline{legajo\_empleado (FK)}\\ \underline{codigo\_proyecto (FK)}\\ rol\\ fecha\_asignacion};
    \node[entity] (tarea) [right=of asignacion] {Tarea\nodepart{second}\textbf{id (PK)}\\ \underline{codigo\_proyecto (FK)}\\ descripcion\\ estado\\ fecha\_limite};
    \node[entity] (dependencia) [below=of tarea] {Dependencia\nodepart{second}\textbf{id\_dependencia (PK)}\\ \underline{id\_tarea\_dependiente (FK)}\\ \underline{id\_tarea\_requerida (FK)}};
    
    % Conexiones empleado-asignacion
    \coordinate (fk_empleado) at ($(asignacion.west)+(0,0.1)$);
    \coordinate (pk_empleado) at ($(empleado.east)+(0,0.4)$);
    \draw[-latex] (fk_empleado) to[out=180,in=0] 
        node[pos=0.2,left] {1..N} 
        node[pos=0.8,right] {1..1} 
        (pk_empleado);
    
    % Conexiones proyecto-asignacion
    \coordinate (fk_proyecto_asignacion) at ($(asignacion.east)+(0,0.1)$);
    \coordinate (pk_proyecto_asignacion) at ($(proyecto.east)+(0,0.3)$);
    \draw[-latex] (fk_proyecto_asignacion) to[out=0,in=0] 
        node[pos=0.2,left] {1..N} 
        node[pos=0.8,left] {1..1} 
        (pk_proyecto_asignacion);
        
    % Conexiones proyecto-tarea
    \coordinate (fk_proyecto_tarea) at ($(tarea.west)+(0,0.1)$);
    \coordinate (pk_proyecto_tarea) at ($(proyecto.east)+(0,0.4)$);
    \draw[-latex] (fk_proyecto_tarea) to[out=180,in=0] 
        node[pos=0.2,right] {1..N} 
        node[pos=0.9,right] {1..1} 
        (pk_proyecto_tarea);
        
    % Conexiones tarea-dependencia
    \coordinate (fk_tarea_dependiente) at ($(dependencia.west)+(0,-0.7)$);
    \coordinate (pk_tarea_dependiente) at ($(tarea.west)+(0,0.4)$);
    \draw[-latex] (fk_tarea_dependiente) to[out=-180,in=-180] 
        node[pos=0.2,left] {1..N} 
        node[pos=0.8,right] {1..1} 
        (pk_tarea_dependiente);
        
    % Conexiones tarea-dependencia (requerida)
    \coordinate (fk_tarea_requerida) at ($(dependencia.east)+(0,-0.5)$);
    \coordinate (pk_tarea_requerida) at ($(tarea.east)+(0,0.6)$);
    \draw[-latex] (fk_tarea_requerida) to[out=0,in=0] 
        node[pos=0.2,right] {N..1} 
        node[pos=0.8,right] {1..1} 
        (pk_tarea_requerida);
\end{tikzpicture}
}
\end{center}

\subsection{Sistema de Hospital}
\begin{center}
\resizebox{1\textwidth}{!}{
\begin{tikzpicture}[node distance=2cm]
    \node[entity] (medico) {Medico\nodepart{second}\textbf{matricula (PK)}\\ nombre\\ especialidad\\ horarios};
    \node[entity] (consulta) [right=of medico] {Consulta\nodepart{second}\textbf{id\_consulta (PK)}\\ \underline{dni\_paciente (FK)}\\ \underline{matricula\_medico (FK)}\\ fecha\\ hora\\ consultorio\\ diagnostico};
    \node[entity] (paciente) [right=of consulta] {Paciente\nodepart{second}\textbf{dni (PK)}\\ nombre\\ alergias\\ enfermedades\_cronicas};
    \node[entity] (receta) [below=of consulta] {Receta\nodepart{second}\textbf{id\_receta (PK)}\\ \underline{id\_consulta (FK)}\\ \underline{codigo\_medicamento (FK)}\\ dosis\\ instrucciones};
    \node[entity] (medicamento) [right=of receta] {Medicamento\nodepart{second}\textbf{codigo (PK)}\\ nombre\\ laboratorio\\ dosis\_recomendada};
    
    % Conexiones paciente-consulta
    \coordinate (fk_paciente) at ($(consulta.east)+(0,0.6)$);
    \coordinate (pk_paciente) at ($(paciente.west)+(0,0.4)$);
    \draw[-latex] (fk_paciente) to[out=0,in=180] 
        node[pos=0.2,above] {1..N} 
        node[pos=0.8,above] {1..1} 
        (pk_paciente);
    
    % Conexiones medico-consulta
    \coordinate (fk_medico) at ($(consulta.west)+(0,0.1)$);
    \coordinate (pk_medico) at ($(medico.east)+(0,-0.1)$);
    \draw[-latex] (fk_medico) to[out=180,in=0] 
        node[pos=0.2,above] {1..N} 
        node[pos=0.8,above] {1..1} 
        (pk_medico);
        
    % Conexiones consulta-receta
    \coordinate (fk_consulta) at ($(receta.west)+(0,0.1)$);
    \coordinate (pk_consulta) at ($(consulta.west)+(0,1.2)$);
    \draw[-latex] (fk_consulta) to[out=-210,in=210] 
        node[pos=0.1,left] {1..N} 
        node[pos=0.9,left] {1..1} 
        (pk_consulta);
        
    % Conexiones receta-medicamento
    \coordinate (fk_medicamento) at ($(receta.east)+(0,-0.5)$);
    \coordinate (pk_medicamento) at ($(medicamento.west)+(0,0.4)$);
    \draw[-latex] (fk_medicamento) to[out=0,in=180] 
        node[pos=0.2,above] {N..1} 
        node[pos=0.8,above] {1..1} 
        (pk_medicamento);
\end{tikzpicture}
}
\end{center}

\subsection{Plataforma de Streaming}
\begin{center}
\resizebox{1\textwidth}{!}{
\begin{tikzpicture}[node distance=2cm]
    \node[entity] (usuario) {Usuario\nodepart{second}\textbf{email (PK)}\\ username\\ password\\ fecha\_registro};
    \node[entity] (perfil) [right=of usuario] {Perfil\nodepart{second}\textbf{id\_perfil (PK)}\\ \underline{email\_usuario (FK)}\\ nombre};
    \node[entity] (contenido) [below=of perfil] {Contenido\nodepart{second}\textbf{id\_contenido (PK)}\\ titulo\\ genero\\ anio\\ clasificacion\\ duracion\\ tipo};
    \node[entity] (resena) [left=of contenido] {Resena\nodepart{second}\textbf{id\_resena (PK)}\\ \underline{id\_perfil (FK)}\\ \underline{id\_contenido (FK)}\\ puntuacion\\ comentario};
    \node[entity] (lista) [right=of perfil] {Lista\nodepart{second}\textbf{id\_lista (PK)}\\ \underline{id\_perfil (FK)}\\ nombre};
    \node[entity] (contenido_lista) [right=of contenido] {Contenido\_Lista\nodepart{second}\textbf{id\_contenido\_lista (PK)}\\ \underline{id\_lista (FK)}\\ \underline{id\_contenido (FK)}};

    % usuario-perfil
    \coordinate (fk_usuario) at ($(perfil.west)+(0,0.1)$);
    \coordinate (pk_usuario) at ($(usuario.east)+(0,0.4)$);
    \draw[-latex] (fk_usuario) to[out=180,in=0]
        node[pos=0.2,above] {1..4}
        node[pos=0.8,above] {1..1}
        (pk_usuario);

    % perfil-resena
    \coordinate (fk_perfil_resena) at ($(resena.east)+(0,0.1)$);
    \coordinate (pk_perfil_resena) at ($(perfil.west)+(0,-0.1)$);
    \draw[-latex] (fk_perfil_resena) to[out=0,in=-180]
        node[pos=0.2,left] {1..N}
        node[pos=0.8,left] {1..1}
        (pk_perfil_resena);

    % contenido-resena
    \coordinate (fk_contenido_resena) at ($(resena.east)+(0,-0.3)$);
    \coordinate (pk_contenido_resena) at ($(contenido.west)+(0,1.2)$);
    \draw[-latex] (fk_contenido_resena) to[out=0,in=180]
        node[pos=0.2,below] {1..N}
        node[pos=0.8,above] {1..1}
        (pk_contenido_resena);

    % perfil-lista
    \coordinate (fk_perfil_lista) at ($(lista.west)+(0,0.1)$);
    \coordinate (pk_perfil_lista) at ($(perfil.east)+(0,-0.7)$);
    \draw[-latex] (fk_perfil_lista) to[out=-180,in=0]
        node[pos=0.2,above] {1..N}
        node[pos=0.8,above] {1..1}
        (pk_perfil_lista);

    % lista-contenido_lista
    \coordinate (fk_lista) at ($(contenido_lista.east)+(0,-0.2)$);
    \coordinate (pk_lista) at ($(lista.east)+(0,0.4)$);
    \draw[-latex] (fk_lista) to[out=70,in=0]
        node[pos=0.2,left] {1..N}
        node[pos=0.9,right] {1..1}
        (pk_lista);

    % contenido-contenido_lista
    \coordinate (fk_contenido_lista) at ($(contenido_lista.west)+(0,-0.7)$);
    \coordinate (pk_contenido_lista) at ($(contenido.east)+(0,1.2)$);
    \draw[-latex] (fk_contenido_lista) to[out=-180,in=0]
        node[pos=0.5,right] {1..N}
        node[pos=0.8,right] {1..1}
        (pk_contenido_lista);
\end{tikzpicture}
}
\end{center}

\subsection{Sistema de Aerolínea}
\begin{center}
\resizebox{1\textwidth}{!}{
\begin{tikzpicture}[node distance=2cm]
    \node[entity] (aeropuerto) {Aeropuerto\nodepart{second}\textbf{codigo (PK)}\\ nombre\\ ciudad\\ pais};
    
    \node[entity] (vuelo) [right=of aeropuerto] {Vuelo\nodepart{second}\textbf{numero (PK)}\\ \underline{codigo\_origen (FK)}\\ \underline{codigo\_destino (FK)}\\ \underline{matricula\_avion (FK)}\\ fecha\\ hora\_salida\\ hora\_llegada};
    
    \node[entity] (avion) [right=of vuelo] {Avion\nodepart{second}\textbf{matricula (PK)}\\ modelo\\ capacidad\\ fecha\_ultimo\_mantenimiento};
    
    \node[entity] (pasajero) [below=of vuelo] {Pasajero\nodepart{second}\textbf{dni (PK)}\\ nombre\\ pasaporte\\ contacto\_emergencia};
    
    \node[entity] (reserva) [right=of pasajero] {Reserva\nodepart{second}\textbf{id\_reserva (PK)}\\ \underline{numero\_vuelo (FK)}\\ \underline{dni\_pasajero (FK)}\\ fecha\_reserva\\ asiento};
    
    \node[entity] (tripulante) [below=of asignacion, xshift=-3cm] {Tripulante\nodepart{second}\textbf{id (PK)}\\ nombre\\ rol};
    
    \node[entity] (asignacion) [left=of pasajero] {Vuelo\_Tripulante\nodepart{second}\textbf{id\_asignacion (PK)}\\ \underline{id\_tripulante (FK)}\\ \underline{numero\_vuelo (FK)}};
    
    \node[entity] (posee) [right=of tripulante] {Tripulante\_Certificacion\nodepart{second}\textbf{id (PK)}\\ \underline{id\_tripulante (FK)}\\ \underline{id\_certificacion (FK)}};

    \node[entity] (certificacion) [right=of posee] {Certificacion\nodepart{second}\textbf{id (PK)}\\ descripcion};

    % Conexiones aeropuerto-vuelo (origen)
    \coordinate (fk_origen) at ($(vuelo.west)+(0,0.1)$);
    \coordinate (pk_origen) at ($(aeropuerto.east)+(0,0.4)$);
    \draw[-latex] (fk_origen) to[out=180,in=0] 
        node[pos=0.2,below] {1..N} 
        node[pos=0.8,below] {1..1} 
        (pk_origen);
    
    % Conexiones aeropuerto-vuelo (destino)
    \coordinate (fk_destino) at ($(vuelo.west)+(0,0.6)$);
    \coordinate (pk_destino) at ($(aeropuerto.east)+(0,0.4)$);
    \draw[-latex] (fk_destino) to[out=180,in=0] 
        node[pos=0.2,above] {1..N} 
        node[pos=0.8,above] {1..1} 
        (pk_destino);
        
    % Conexiones avion-vuelo
    \coordinate (fk_avion) at ($(vuelo.east)+(0,-0.5)$);
    \coordinate (pk_avion) at ($(avion.west)+(0,0.4)$);
    \draw[-latex] (fk_avion) to[out=0,in=180] 
        node[pos=0.2,above] {N..1} 
        node[pos=0.8,above] {1..1} 
        (pk_avion);
        
    % Conexiones vuelo-reserva
    \coordinate (fk_vuelo) at ($(reserva.west)+(0,0.5)$);
    \coordinate (pk_vuelo) at ($(vuelo.east)+(0,1.2)$);
    \draw[-latex] (fk_vuelo) to[out=120,in=0] 
        node[pos=0.2,right] {1..N} 
        node[pos=0.9,right] {1..1} 
        (pk_vuelo);
        
    % Conexiones pasajero-reserva
    \coordinate (fk_pasajero) at ($(reserva.west)+(0,0.1)$);
    \coordinate (pk_pasajero) at ($(pasajero.east)+(0,0.4)$);
    \draw[-latex] (fk_pasajero) to[out=180,in=0] 
        node[pos=0.2,above] {1..N} 
        node[pos=0.8,above] {1..1} 
        (pk_pasajero);

    % Conexión tripulante-asignación
    \coordinate (fk_tripulante) at ($(asignacion.east)+(0,-0.1)$);
    \coordinate (pk_tripulante) at ($(tripulante.west)+(0,0.4)$);
    \draw[-latex] (fk_tripulante) to[out=0,in=180] 
        node[pos=0.3,right] {1..N} 
        node[pos=0.8,left] {1..1} 
        (pk_tripulante);

    % Conexión vuelo-asignación
    \coordinate (fk_vuelo) at ($(asignacion.east)+(0,-0.7)$);
    \coordinate (pk_vuelo) at ($(vuelo.west)+(0,1.3)$);
    \draw[-latex] (fk_vuelo) to[out=0,in=180] 
        node[pos=0.2,right] {1..N} 
        node[pos=0.9,above] {1..1} 
        (pk_vuelo);

    % Conexión tripulante-posee
    \coordinate (fk_posee_trip) at ($(posee.west)+(0,-0.5)$);
    \coordinate (pk_posee_trip) at ($(tripulante.east)+(0,0.1)$);
    \draw[-latex] (fk_posee_trip) to[out=-180,in=0] 
        node[pos=0.2,above] {1..N} 
        node[pos=0.8,below] {1..1} 
        (pk_posee_trip);

    % Conexión certificacion-posee
    \coordinate (fk_posee_cert) at ($(posee.east)+(0,-0.8)$);
    \coordinate (pk_posee_cert) at ($(certificacion.west)+(0,-0.1)$);
    \draw[-latex] (fk_posee_cert) to[out=0,in=180] 
        node[pos=0.2,below] {N..1} 
        node[pos=0.8,above] {1..1} 
        (pk_posee_cert);
\end{tikzpicture}
}
\end{center}

\end{document}
