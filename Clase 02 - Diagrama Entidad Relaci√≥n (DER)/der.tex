\documentclass[12pt]{article}

\usepackage[utf8]{inputenc}
\usepackage[T1]{fontenc}
\usepackage{lmodern}
\usepackage[spanish]{babel}
\usepackage{booktabs}
\usepackage{amsmath}
\usepackage{forest}
\usepackage{float}
\usepackage{listings}
\usepackage{xcolor}
\usepackage{tikz}
\usetikzlibrary{positioning,shapes.multipart,calc,arrows,shapes.geometric}

% Definicion de estilos para entidades
\tikzset{
    entity/.style={
        rectangle split,
        rectangle split parts=2,
        draw,
        align=center,
        minimum height=1cm
    },
    arrow/.style={
        ->,
        >=latex
    }
}

\definecolor{codegreen}{rgb}{0,0.6,0}
\definecolor{codegray}{rgb}{0.5,0.5,0.5}
\definecolor{codepurple}{rgb}{0.58,0,0.82}
\definecolor{backcolour}{rgb}{0.95,0.95,0.92}

\lstdefinestyle{mystyle}{
    backgroundcolor=\color{backcolour},   
    commentstyle=\color{codegreen},
    keywordstyle=\color{magenta},
    numberstyle=\tiny\color{codegray},
    stringstyle=\color{codepurple},
    basicstyle=\ttfamily\footnotesize,
    breakatwhitespace=false,         
    breaklines=true,                 
    captionpos=b,                    
    keepspaces=true,                 
    numbers=left,                    
    numbersep=5pt,                  
    showspaces=false,                
    showstringspaces=false,
    showtabs=false,                  
    tabsize=2
}

\lstset{style=mystyle}

\sloppy
\setlength{\parindent}{0pt}

\begin{document}

\begin{center}
  {\LARGE \textbf{Diagramas Entidad-Relación (DER)}}\\[0.5em]
  {Gestión y Arquitectura de Datos, Universidad de San Andrés}
\end{center}

Si encuentran algún error en el documento o hay alguna duda, mandenmé un mail a rodriguezf@udesa.edu.ar y lo revisamos.

\section{Conceptos Básicos}
Un Diagrama Entidad-Relación (DER) es una herramienta para el modelado de datos que permite representar las entidades relevantes de un sistema de información así como sus interrelaciones.

\subsection{Elementos Básicos}
\begin{itemize}
    \item \textbf{Entidad}: Objeto o concepto del mundo real
    \item \textbf{Atributos}: Propiedades que describen a la entidad
    \item \textbf{Relación}: Asociación entre entidades
    \item \textbf{Cardinalidad}: Indica cuántas instancias de una entidad pueden estar relacionadas con otra
\end{itemize}

\section{Relación Uno a Muchos}
Consideremos un sistema donde cada cliente puede tener múltiples pedidos, pero cada pedido pertenece a un único cliente.

\begin{center}
\begin{tikzpicture}[node distance=4cm]
    % Entidades
    \node[entity] (cliente) {Cliente\nodepart{second}\textbf{id\_cliente (PK)}\\ nombre\\ email\\ telefono};
    \node[entity] (pedido) [right=of cliente] {Pedido\nodepart{second}\textbf{id\_pedido (PK)}\\ \underline{id\_cliente (FK)}\\ fecha\\ total\\ estado};
    
    % Conexión con cardinalidades y flecha FK->PK
    \coordinate (fk_cliente) at ($(pedido.west)+(0,0.1)$);
    \coordinate (pk_cliente) at ($(cliente.east)+(0,0.4)$);
    \draw[-latex] (fk_cliente) to[out=180,in=0] 
        node[pos=0.1,above] {1..N} 
        node[pos=0.9,above] {1..1} 
        (pk_cliente);
\end{tikzpicture}
\end{center}

En este ejemplo simple podemos ver:
\begin{itemize}
    \item Dos entidades con sus atributos
    \item Una relación uno a muchos (1..1 : 1..N)
    \item Claves primarias en negrita con (PK)
    \item Clave foránea subrayada con (FK)
    \item Flecha que conecta la FK con su PK correspondiente
\end{itemize}

\section{Relación Muchos a Muchos}
Ahora veamos un sistema donde los productos pueden estar en múltiples pedidos y cada pedido puede tener múltiples productos.

\begin{center}
\begin{tikzpicture}[node distance=2cm]
    % Entidades principales
    \node[entity] (pedido) {Pedido\nodepart{second}\textbf{id\_pedido (PK)}\\ fecha\\ total};
    \node[entity] (detalle) [right=of pedido] {Detalle\_Pedido\nodepart{second}\textbf{id\_detalle (PK)}\\ \underline{id\_pedido (FK)}\\ \underline{id\_producto (FK)}\\ cantidad\\ precio\_unitario};
    \node[entity] (producto) [right=of detalle] {Producto\nodepart{second}\textbf{id\_producto (PK)}\\ nombre\\ precio\\ stock};
    
    % Conexiones con flechas FK->PK
    \coordinate (fk_pedido) at ($(detalle.west)+(0,0.1)$);
    \coordinate (pk_pedido) at ($(pedido.east)+(0,0.1)$);
    \draw[-latex] (fk_pedido) to[out=180,in=0] 
        node[pos=0.2,above] {1..N} 
        node[pos=0.8,above] {1..1} 
        (pk_pedido);
        
    \coordinate (fk_producto) at ($(detalle.east)+(0,-0.5)$);
    \coordinate (pk_producto) at ($(producto.west)+(0,0.5)$);
    \draw[-latex] (fk_producto) to[out=0,in=180] 
        node[pos=0.2,above] {N..1} 
        node[pos=0.8,above] {1..1} 
        (pk_producto);
\end{tikzpicture}
\end{center}

Este ejemplo muestra:
\begin{itemize}
    \item Una relación muchos a muchos (N:M)
    \item Una tabla pivot con sus propios atributos
    \item Múltiples FKs en la tabla pivot
    \item Flechas conectando cada FK con su PK correspondiente
\end{itemize}

\section{Herencia (Subtipo-Supertipo)}
Veamos un sistema donde tenemos diferentes tipos de usuarios.

\begin{center}
\begin{tikzpicture}[node distance=4cm]
    % Entidad padre
    \node[entity] (usuario) {Usuario\nodepart{second}\textbf{id\_usuario (PK)}\\ nombre\\ email\\ password};
    
    % Entidades hijas
    \node[entity] (cliente) [below left=2cm of usuario] {Cliente\nodepart{second}\underline{\textbf{id\_usuario (PK,FK)}}\\ direccion\\ telefono\\ nivel\_membresia};
    \node[entity] (empleado) [below right=2cm of usuario] {Empleado\nodepart{second}\underline{\textbf{id\_usuario (PK,FK)}}\\ salario\\ departamento\\ fecha\_ingreso};
    
    % Relación de herencia con flecha compartida
    \coordinate (pk_usuario) at ($(usuario.south)+(0,-0.1)$);
    \coordinate (fk_cliente) at ($(cliente.north)+(0,0.1)$);
    \coordinate (fk_empleado) at ($(empleado.north)+(0,0.1)$);
    \coordinate (union) at ($(fk_cliente)!0.5!(fk_empleado)$);
    
    \draw[arrow] (union) -- (pk_usuario);
    \draw (fk_cliente) -- (union);
    \draw (fk_empleado) -- (union);
\end{tikzpicture}
\end{center}

Este ejemplo ilustra:
\begin{itemize}
    \item Una jerarquía de herencia con flecha unificada
    \item Atributos heredados (la PK del padre se convierte en PK y FK en los hijos)
    \item Atributos específicos en cada subtipo
\end{itemize}

\section{Relación Autoreferencial TEST}
Por último, veamos un ejemplo de una estructura jerárquica donde los empleados tienen supervisores.

\begin{center}
\begin{tikzpicture}[node distance=4cm]
    % Entidad
    \node[entity] (empleado) {Empleado\nodepart{second}\textbf{id\_empleado (PK)}\\ \underline{id\_supervisor (FK)}\\ nombre\\ cargo\\ departamento};
    
    \coordinate (salida) at ($(empleado.north)+(0,0)$);         
    \coordinate (p1) at ($(salida)+(0,1)$);                      
    \coordinate (p2) at ($(p1)+(3.5,0)$);                          
    \coordinate (p3) at ($(p2)+(0,-2.7)$);                         
    \coordinate (entrada) at ($(empleado.east)+(0,0)$);           
    \coordinate (p4) at ($(entrada)+(0.8,0)$);                    

    \draw[-latex, thick] (salida) -- (p1) -- (p2) -- (p3) -- (p4) -- (entrada);

    % Cardinalidades
    \node at ($(p2)+(-3,0.4)$) {0..N};
    \node at ($(p4)+(-0.4,0.3)$) {0..1};

\end{tikzpicture}
\end{center}

Este ejemplo muestra:
\begin{itemize}
    \item Una relación de una entidad consigo misma
    \item FK que referencia a la PK de la misma tabla
    \item Cardinalidad 0..N porque un empleado puede no tener subordinados
    \item Cardinalidad 1..1 porque un empleado tiene exactamente un supervisor
\end{itemize}

\section{Consideraciones Importantes}
\begin{itemize}
    \item \textbf{Claves Primarias (PK)}: Identificador único, mostrado en negrita
    \item \textbf{Claves Foráneas (FK)}: Referencias a PKs, mostradas subrayadas
    \item \textbf{Cardinalidades}: Siempre indicar en ambos extremos (min..max)
    \item \textbf{Herencia}: Las PKs se heredan como PK,FK en las tablas hijas
    \item \textbf{Relaciones N:M}: Requieren tabla pivot con sus propias FKs
\end{itemize}

\end{document}
