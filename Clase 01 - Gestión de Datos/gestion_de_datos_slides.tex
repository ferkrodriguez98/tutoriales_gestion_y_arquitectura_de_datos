% Diapositivas Gestión de Datos - UdeSA

\documentclass{beamer}
\usetheme{metropolis}

\usepackage[spanish]{babel}
\usepackage[utf8]{inputenc}
\usepackage{graphicx}
\usepackage{tikz}
\usepackage{xcolor}
\usepackage{amsmath}
\usepackage{fontawesome}

\usetikzlibrary{positioning,shapes.multipart,calc,arrows,shapes.geometric}

% Colores y configuración del template
\definecolor{primary}{RGB}{46, 204, 113}
\definecolor{secondary}{RGB}{52, 152, 219}
\definecolor{accent}{RGB}{231, 76, 60}
\definecolor{background}{RGB}{236, 240, 241}
\setbeamercolor{normal text}{fg=black,bg=background}
\setbeamercolor{structure}{fg=primary}
\setbeamercolor{alerted text}{fg=accent}

% Título
\title{Gestión de Datos}
\author{Gestión y Arquitectura de Datos}
\date{}

\begin{document}

% Portada
\begin{frame}
    \titlepage
    \begin{tikzpicture}[remember picture,overlay]
        \node[anchor=south west,inner sep=30pt] at (current page.south west) {
            \includegraphics[height=1cm]{../misc/UdeSA.png}
        };
    \end{tikzpicture}
\end{frame}

% Introducción
\begin{frame}{Introducción}
    \begin{center}
        La gestión de datos dentro de una organización es el proceso de:
        
        \vspace{0.8cm}
        
        \begin{tabular}{cccc}
            \begin{minipage}{2.2cm}
                \centering
                {\Large \faSitemap}\\[0.2cm]
                \textbf{Organizar}\\[0.1cm]
            \end{minipage} &
            \begin{minipage}{2.2cm}
                \centering
                {\Large \faShield}\\[0.2cm]
                \textbf{Proteger}\\[0.1cm]
            \end{minipage} &
            \begin{minipage}{2.2cm}
                \centering
                {\Large \faSearch}\\[0.2cm]
                \textbf{Recuperar}\\[0.1cm]
            \end{minipage} &
            \begin{minipage}{2.2cm}
                \centering
                {\Large \faWrench}\\[0.2cm]
                \textbf{Mantener}\\[0.1cm]
            \end{minipage}
        \end{tabular}
    \end{center}
\end{frame}

% Ciclo de Vida de los Datos (slide general)
\begin{frame}{Ciclo de vida de los datos}
    \begin{center}
    \begin{tikzpicture}[node distance=0.6cm]
        \node (c1) [rectangle, draw, fill=white, minimum width=4cm, minimum height=0.8cm, rounded corners] {\only<1->{Creación y captura}};
        \node (c2) [rectangle, draw, fill=white, below=of c1, minimum width=4cm, minimum height=0.8cm, rounded corners] {\only<2->{Almacenamiento}};
        \node (c3) [rectangle, draw, fill=white, below=of c2, minimum width=4cm, minimum height=0.8cm, rounded corners] {\only<3->{Procesamiento}};
        \node (c4) [rectangle, draw, fill=white, below=of c3, minimum width=4cm, minimum height=0.8cm, rounded corners] {\only<4->{Análisis}};
        \node (c5) [rectangle, draw, fill=white, below=of c4, minimum width=4cm, minimum height=0.8cm, rounded corners] {\only<5->{Archivado y eliminación}};
        \draw[->, thick] (c1) -- (c2);
        \draw[->, thick] (c2) -- (c3);
        \draw[->, thick] (c3) -- (c4);
        \draw[->, thick] (c4) -- (c5);
    \end{tikzpicture}
    \end{center}
    % FOTO: ciclo de vida de los datos (poner diagrama o imagen)
\end{frame}

% Creación y Captura
\begin{frame}{Creación y captura de datos}
    \begin{itemize}
        \item<1->[{\makebox[1em][c]{\textcolor{black}{\faDesktop}}}] Ingreso manual de datos
        \item<2->[{\makebox[1em][c]{\textcolor{black}{\faCog}}}] Captura automática
        \item<3->[{\makebox[1em][c]{\textcolor{black}{\faLink}}}] Integración de sistemas
        \item<4->[{\makebox[1em][c]{\textcolor{black}{\faWifi}}}] Sensores y dispositivos IoT
    \end{itemize}
    % ICONO: entrada de datos
\end{frame}

% Almacenamiento
\begin{frame}{Almacenamiento de datos}
    \begin{itemize}
        \raggedleft
        \item<1->[{\makebox[1em][c]{\textcolor{black}{\faDatabase}}}] Bases de datos relacionales
        \item<2->[{\makebox[1em][c]{\textcolor{black}{\faServer}}}] Bases de datos NoSQL
        \item<3->[{\makebox[1em][c]{\textcolor{black}{\faCloud}}}] Data Lakes
        \item<4->[{\makebox[1em][c]{\textcolor{black}{\faArchive}}}] Sistemas de archivos distribuidos
    \end{itemize}
    % ICONO: base de datos o almacenamiento
\end{frame}

% Procesamiento
\begin{frame}{Procesamiento de datos}
    \begin{itemize}
        \item<1->[{\makebox[1em][c]{\textcolor{black}{\faCogs}}}] ETL (Extract, Transform, Load)
        \item<2->[{\makebox[1em][c]{\textcolor{black}{\faBug}}}] Limpieza de datos
        \item<3->[{\makebox[1em][c]{\textcolor{black}{\faCheck}}}] Validación
        \item<4->[{\makebox[1em][c]{\textcolor{black}{\faPlus}}}] Enriquecimiento
    \end{itemize}
    % ICONO: engranaje o proceso
\end{frame}

% Análisis
\begin{frame}{Análisis de datos}
    \begin{itemize}
        \raggedleft
        \item<1->[{\makebox[1em][c]{\textcolor{black}{\faBarChart}}}] Business Intelligence
        \item<2->[{\makebox[1em][c]{\textcolor{black}{\faLineChart}}}] Analytics
        \item<3->[{\makebox[1em][c]{\textcolor{black}{\faCogs}}}] Machine Learning
        \item<4->[{\makebox[1em][c]{\textcolor{black}{\faPieChart}}}] Visualización
    \end{itemize}
    % ICONO: grafico o dashboard
\end{frame}

% Archivado y Eliminación
\begin{frame}{Archivado y eliminación de datos}
    \begin{itemize}
        \item<1->[{\makebox[1em][c]{\textcolor{black}{\faArchive}}}] Políticas de retención
        \item<2->[{\makebox[1em][c]{\textcolor{black}{\faGavel}}}] Cumplimiento regulatorio
        \item<3->[{\makebox[1em][c]{\textcolor{black}{\faSave}}}] Backup y recuperación
        \item<4->[{\makebox[1em][c]{\textcolor{black}{\faTrash}}}] Eliminación segura
    \end{itemize}
    % ICONO: caja de archivo o papelera
\end{frame}

% Calidad de Datos - Dimensiones (1)
\begin{frame}{Dimensiones de la calidad de los datos (I)}
    \begin{itemize}
        \item Precisión: grado en que los datos son exactos y libres de errores.
        \item Completitud: grado en que los datos cubren todas las partes relevantes de la información.
        \item Consistencia: grado en que los datos son coherentes y no contienen contradicciones.
        \item Actualidad: grado en que los datos reflejan la situación actual.
    \end{itemize}
    % ICONO: check o medalla
\end{frame}

% Calidad de Datos - Dimensiones (2)
\begin{frame}{Dimensiones de la calidad de los datos (II)}
    \begin{itemize}
        \item Unicidad: grado en que los datos son únicos y no duplicados.
        \item Accesibilidad: grado en que los datos son faciles de encontrar y usar.
        \item Seguridad: grado en que los datos estan protegidos contra accesos no autorizados.
    \end{itemize}
    % ICONO: check o medalla
\end{frame}

% Calidad de Datos - Métricas
\begin{frame}{Métricas de la calidad de los datos}
    \begin{itemize}
        \item [{\makebox[1em][c]{\textcolor{black}{\faExclamationTriangle}}}] Tasa de error: proporción de datos que contienen errores.
        \item [{\makebox[1em][c]{\textcolor{black}{\faClone}}}] Tasa de duplicación: proporción de datos que son duplicados.
        \item [{\makebox[1em][c]{\textcolor{black}{\faRefresh}}}] Tasa de actualización: proporción de datos que se actualizan regularmente.
    \end{itemize}
    % ICONO: estadistica o metrica
\end{frame}

% Gobierno de Datos - Roles
\begin{frame}{Gobierno de datos: roles y responsabilidades}
    \begin{center}
        \begin{tabular}{ccc}
            \begin{minipage}{2.8cm}
                \centering
                \fbox{\begin{minipage}[t][3.5cm]{2.6cm}
                    \centering
                    \vspace{0.2cm}
                    {\Large \faUserSecret}\\[0.3cm]
                    \textbf{CDO}\\[0.2cm]
                    {\scriptsize Dirección y gestión de todos los datos de la organización}
                \end{minipage}}
            \end{minipage} &
            \begin{minipage}{2.8cm}
                \centering
                \fbox{\begin{minipage}[t][3.5cm]{2.6cm}
                    \centering
                    \vspace{0.2cm}
                    {\Large \faUser}\\[0.3cm]
                    \textbf{Data Owner}\\[0.2cm]
                    {\scriptsize Responsable de un conjunto de datos específico}
                \end{minipage}}
            \end{minipage} &
            \begin{minipage}{2.8cm}
                \centering
                \fbox{\begin{minipage}[t][3.5cm]{2.6cm}
                    \centering
                    \vspace{0.2cm}
                    {\Large \faWrench}\\[0.3cm]
                    \textbf{Data Steward}\\[0.2cm]
                    {\scriptsize Responsable operativo, mantiene datos actualizados}
                \end{minipage}}
            \end{minipage}
        \end{tabular}
    \end{center}
    % ICONO: personas o jerarquía
\end{frame}

% Gobierno de Datos - Políticas
\begin{frame}{Gobierno de datos: políticas y procedimientos}
    \begin{center}
        \begin{tabular}{ccc}
            \begin{minipage}{2.5cm}
                \centering
                {\Large \faShield}\\[0.1cm]
                \textbf{Seguridad}\\[0.1cm]
                {\scriptsize Protección de datos}
            \end{minipage} &
            \begin{minipage}{2.5cm}
                \centering
                {\Large \faEyeSlash}\\[0.1cm]
                \textbf{Privacidad}\\[0.1cm]
                {\scriptsize Control de acceso}
            \end{minipage} &
            \begin{minipage}{2.5cm}
                \centering
                {\Large \faCalendar}\\[0.1cm]
                \textbf{Retención}\\[0.1cm]
                {\scriptsize Tiempo de vida}
            \end{minipage}
        \end{tabular}
        
        \vspace{0.8cm}
        
        \begin{tabular}{cc}
            \begin{minipage}{2.5cm}
                \centering
                {\Large \faUsers}\\[0.1cm]
                \textbf{Uso}\\[0.1cm]
                {\scriptsize Normas}
            \end{minipage} &
            \begin{minipage}{2.5cm}
                \centering
                {\Large \faKey}\\[0.1cm]
                \textbf{Acceso}\\[0.1cm]
                {\scriptsize Permisos y roles}
            \end{minipage}
        \end{tabular}
    \end{center}
    % ICONO: documento o candado
\end{frame}

% Arquitectura de Datos - General
\begin{frame}{Arquitectura de datos}
    \begin{center}
        \begin{tabular}{c}
            \fbox{\begin{minipage}[c][1.2cm]{8cm}
                \centering
                \textbf{Capa de Fuentes}\\
                {\scriptsize Sistemas operacionales, APIs externas, archivos}
            \end{minipage}}\\[0.2cm]

            \fbox{\begin{minipage}[c][1.2cm]{8cm}
                \centering
                \textbf{Capa de Procesamiento}\\
                {\scriptsize ETL, transformaciones, análisis}
            \end{minipage}}\\[0.2cm]
            
            \fbox{\begin{minipage}[c][1.2cm]{8cm}
                \centering
                \textbf{Capa de Almacenamiento}\\
                {\scriptsize Data warehouses, data lakes, bases de datos}
            \end{minipage}}\\[0.2cm]

            \fbox{\begin{minipage}[c][1.2cm]{8cm}
                \centering
                \textbf{Capa de Consumo}\\
                {\scriptsize Reportes, dashboards, APIs}
            \end{minipage}}
        \end{tabular}
    \end{center}
    % FOTO: arquitectura de datos (poner diagrama de capas)
\end{frame}

% Arquitectura de Datos - Fuentes
\begin{frame}{Arquitectura de datos: fuentes de datos}
    \begin{columns}
        \begin{column}{0.6\textwidth}
            \textbf{Fuentes de datos:}
            \begin{itemize}
                \item Sistemas operacionales (ERP, CRM)
                \item APIs externas
                \item Datos no estructurados (documentos, imagenes, videos)
            \end{itemize}
        \end{column}
        \begin{column}{0.4\textwidth}
            \centering
            {\fontsize{60}{60}\selectfont \faCubes}
        \end{column}
    \end{columns}
\end{frame}

% Arquitectura de Datos - Almacenamiento
\begin{frame}{Arquitectura de datos: almacenamiento}
    \begin{columns}
        \begin{column}{0.6\textwidth}
            \textbf{Almacenamiento:}
            \begin{itemize}
                \item Data Warehouse: datos estructurados y análisis histórico
                \item Data Lake: datos en formato original
        \item Data Mart: subconjuntos especializados
            \end{itemize}
        \end{column}
        \begin{column}{0.4\textwidth}
            \centering
            {\fontsize{60}{60}\selectfont \faDatabase}
        \end{column}
    \end{columns}
\end{frame}

% Arquitectura de Datos - Procesamiento
\begin{frame}{Arquitectura de datos: procesamiento}
    \begin{columns}
        \begin{column}{0.6\textwidth}
            \textbf{Procesamiento:}
            \begin{itemize}
                \item Procesamiento por lotes (Batch)
        \item Procesamiento en streaming
        \item Arquitectura Lambda (combinación de ambos)
            \end{itemize}
        \end{column}
        \begin{column}{0.4\textwidth}
            \centering
            {\fontsize{60}{60}\selectfont \faCogs}
        \end{column}
    \end{columns}
\end{frame}

% Arquitectura de Datos - Consumo
\begin{frame}{Arquitectura de datos: consumo}
    \begin{columns}
        \begin{column}{0.6\textwidth}
            \textbf{Consumo:}
            \begin{itemize}
                \item Reportes estructurados
                \item Dashboards interactivos
                \item APIs para integración
            \end{itemize}
        \end{column}
        \begin{column}{0.4\textwidth}
            \centering
            {\fontsize{60}{60}\selectfont \faDesktop}
        \end{column}
    \end{columns}
\end{frame}

% Cierre
\begin{frame}{Terminamos}
    \begin{center}
        \Large{\textbf{¿Dudas?\\¿Consultas?}}
    \end{center}
    \begin{tikzpicture}[remember picture,overlay]
        \node[anchor=south,inner sep=30pt] at (current page.south) {
            \includegraphics[height=1cm]{../misc/UdeSA.png}
        };
    \end{tikzpicture}
\end{frame}

\end{document}
