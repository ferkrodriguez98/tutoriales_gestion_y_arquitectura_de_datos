\documentclass[12pt]{article}

\usepackage[utf8]{inputenc}
\usepackage[T1]{fontenc}
\usepackage[spanish]{babel}
\usepackage{enumitem}
\usepackage{amsmath}
\usepackage{booktabs}

\begin{document}

\begin{center}
  {\LARGE \textbf{Guía de Preguntas - Gestión de Datos}}\\[0.5em]
  {Gestión y Arquitectura de Datos, Universidad de San Andrés}
\end{center}

Si encuentran algún error en el documento o hay alguna duda, mandenmé un mail a rodriguezf@udesa.edu.ar y lo revisamos.

\section*{Preguntas}

\begin{enumerate}[label=\arabic*.]

\item En el contexto de calidad de datos, ¿cuál de las siguientes afirmaciones sobre la dimensión de ``Consistencia'' es correcta?
\begin{enumerate}
    \item Significa que los datos son coherentes a través de diferentes sistemas y representaciones
    \item Se refiere únicamente a que los datos estén completos
    \item Implica que los datos sean coherentes solo dentro de una misma base de datos
    \item Solo se aplica a datos numéricos
\end{enumerate}

\item Una empresa implementa una arquitectura de datos moderna. ¿Cuál de las siguientes combinaciones de componentes sería la más apropiada para procesamiento en tiempo real y batch?
\begin{enumerate}
    \item Data Lake + Data Warehouse
    \item Solo Data Warehouse
    \item Lambda Architecture + Stream Processing
    \item Data Mart + Batch Processing
\end{enumerate}

\item En el rol de Data Steward, ¿cuál de las siguientes NO es una responsabilidad principal?
\begin{enumerate}
    \item Mantener los metadatos actualizados
    \item Asegurar la calidad de los datos
    \item Verificar el cumplimiento de políticas de datos
    \item Definir la estrategia general de datos de la empresa
\end{enumerate}

\item ¿Qué desafío principal resuelve la arquitectura Data Mesh?
\begin{enumerate}
    \item La centralización excesiva en equipos de datos
    \item La necesidad de procesamiento batch exclusivamente
    \item La falta de seguridad en los datos
    \item La imposibilidad de usar machine learning
\end{enumerate}

\item En el ciclo de vida de los datos, ¿qué fase debería incluir necesariamente validación de calidad?
\begin{enumerate}
    \item Solo en la fase de Archivado
    \item En la fase de Procesamiento (ETL)
    \item Solo en la fase de Análisis
    \item Solo en la fase de Creación
\end{enumerate}

\item Para implementar una política efectiva de gobierno de datos, ¿qué combinación de elementos es más crítica?
\begin{enumerate}
    \item Solo tecnología y herramientas
    \item Únicamente automatización
    \item Roles definidos + Políticas claras + Procesos documentados
    \item Solo documentación técnica
\end{enumerate}

\item En el contexto de protección de datos, ¿qué estrategia es más efectiva para datos sensibles en uso?
\begin{enumerate}
    \item Solo encriptación en reposo
    \item Únicamente control de acceso
    \item Backup diario
    \item Encriptación en tránsito + Enmascaramiento dinámico
\end{enumerate}

\item Para mejorar la calidad de datos en tiempo real, ¿qué enfoque es más efectivo?
\begin{enumerate}
    \item Reglas automatizadas + Monitoreo continuo + Alertas
    \item Validación manual periódica
    \item Solo documentación
    \item Revisión mensual
\end{enumerate}

\item En una arquitectura moderna de datos, ¿qué característica es esencial para garantizar el linaje de datos?
\begin{enumerate}
    \item Solo logs de acceso
    \item Metadata activa + Tracking de transformaciones
    \item Únicamente documentación
    \item Backup semanal
\end{enumerate}

\item Para implementar DataOps efectivamente, ¿qué conjunto de prácticas es más importante?
\begin{enumerate}
    \item Solo testing manual
    \item Únicamente documentación
    \item Automatización + CI/CD + Monitoreo + Colaboración
    \item Reuniones diarias
\end{enumerate}

\end{enumerate}

\newpage
\section*{Respuestas}

\begin{enumerate}[label=\arabic*.]

\item En el contexto de calidad de datos, ¿cuál de las siguientes afirmaciones sobre la dimensión de ``Consistencia'' es correcta?
\begin{enumerate}
    \item \textbf{Significa que los datos son coherentes a través de diferentes sistemas y representaciones}
    \item Se refiere únicamente a que los datos estén completos
    \item Implica que los datos sean coherentes solo dentro de una misma base de datos
    \item Solo se aplica a datos numéricos
\end{enumerate}
La consistencia NO es completitud (b) ni se limita a una base (c). Requiere coherencia entre múltiples sistemas y representaciones, no solo datos numéricos (d).

\item Una empresa implementa una arquitectura de datos moderna. ¿Cuál de las siguientes combinaciones de componentes sería la más apropiada para procesamiento en tiempo real y batch?
\begin{enumerate}
    \item Data Lake + Data Warehouse
    \item Solo Data Warehouse
    \item \textbf{Lambda Architecture + Stream Processing}
    \item Data Mart + Batch Processing
\end{enumerate}
Data Lake/Warehouse (a) o solo Warehouse (b) NO manejan tiempo real eficientemente. Lambda Architecture sí combina ambos procesamientos, no solo batch (d).

\item En el rol de Data Steward, ¿cuál de las siguientes NO es una responsabilidad principal?
\begin{enumerate}
    \item Mantener los metadatos actualizados
    \item Asegurar la calidad de los datos
    \item Verificar el cumplimiento de políticas de datos
    \item \textbf{Definir la estrategia general de datos de la empresa}
\end{enumerate}
Data Steward SÍ hace metadatos (a), calidad (b) y cumplimiento (c). Estrategia general (d) es responsabilidad del CDO, no del Steward.

\item ¿Qué desafío principal resuelve la arquitectura Data Mesh?
\begin{enumerate}
    \item \textbf{La centralización excesiva en equipos de datos}
    \item La necesidad de procesamiento batch exclusivamente
    \item La falta de seguridad en los datos
    \item La imposibilidad de usar machine learning
\end{enumerate}
Data Mesh NO resuelve batch (b), seguridad (c) o ML (d). Su foco es descentralizar equipos de datos mediante dominios autónomos.

\item En el ciclo de vida de los datos, ¿qué fase debería incluir necesariamente validación de calidad?
\begin{enumerate}
    \item Solo en la fase de Archivado
    \item \textbf{En la fase de Procesamiento (ETL)}
    \item Solo en la fase de Análisis
    \item Solo en la fase de Creación
\end{enumerate}
Archivado (a), Análisis (c) y Creación (d) NO transforman datos. ETL es donde se valida y limpia antes del uso final.

\item Para implementar una política efectiva de gobierno de datos, ¿qué combinación de elementos es más crítica?
\begin{enumerate}
    \item Solo tecnología y herramientas
    \item Únicamente automatización
    \item \textbf{Roles definidos + Políticas claras + Procesos documentados}
    \item Solo documentación técnica
\end{enumerate}
Solo tecnología (a), automatización (b) o documentación (d) NO bastan. Gobierno necesita roles, políticas y procesos integrados para funcionar.

\item En el contexto de protección de datos, ¿qué estrategia es más efectiva para datos sensibles en uso?
\begin{enumerate}
    \item Solo encriptación en reposo
    \item Únicamente control de acceso
    \item Backup diario
    \item \textbf{Encriptación en tránsito + Enmascaramiento dinámico}
\end{enumerate}
Encriptación en reposo (a), control de acceso (b) o backup (c) NO protegen datos EN USO. Tránsito + enmascaramiento sí.

\item Para mejorar la calidad de datos en tiempo real, ¿qué enfoque es más efectivo?
\begin{enumerate}
    \item \textbf{Reglas automatizadas + Monitoreo continuo + Alertas}
    \item Validación manual periódica
    \item Solo documentación
    \item Revisión mensual
\end{enumerate}
Validación manual (b), documentación (c) o revisión mensual (d) NO detectan problemas en tiempo real. Automatización + monitoreo sí.

\item En una arquitectura moderna de datos, ¿qué característica es esencial para garantizar el linaje de datos?
\begin{enumerate}
    \item Solo logs de acceso
    \item \textbf{Metadata activa + Tracking de transformaciones}
    \item Únicamente documentación
    \item Backup semanal
\end{enumerate}
Logs (a), documentación (c) o backup (d) NO rastrean transformaciones. Metadata activa + tracking sí registra origen y cambios.

\item Para implementar DataOps efectivamente, ¿qué conjunto de prácticas es más importante?
\begin{enumerate}
    \item Solo testing manual
    \item Únicamente documentación
    \item \textbf{Automatización + CI/CD + Monitoreo + Colaboración}
    \item Reuniones diarias
\end{enumerate}
Testing manual (a), documentación (b) o reuniones (d) NO escalan. DataOps requiere automatización, CI/CD y monitoreo para agilidad.

\end{enumerate}

\end{document}
